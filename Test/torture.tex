%% -*- latex  encoding: iso-8859-1 -*-
%% This is a torture test for tralics 
% The next line indicates that the document class to use
% by tralics should be torture
%\documentclass{torture1}
%ok
%this is a comment \def}}}
% $Id: torture.tex,v 1.181 2017/05/29 06:22:57 grimm Exp $

% Define the class via a command
\def\myclass{article}
% Define some macros. 
\def\entity#1{\xmllatex{\&#1;}{}}
\def\xentity#1{\xmllatex{\&\#x#1;}{}}
\def\strok#1{\entity{#1strok}}
\def\Han{H�n Th\xmllatex{&\#x1ebf;}{\'{\^e}} Th�nh}

%% will be used at end of file. 
\def\thisistheend{END OF FILE}
\def\discardcounter#1{\expandafter\let\csname c@#1\endcsname \relax}

\def\xshow#1{} % Could be \show

\AtBeginDocument{\myprintdate}
\AtBeginDocument{DOC}\AtBeginDocument{S\AtBeginDocument{TART}}
\AtEndDocument{\AtEndDocument{AT\_}}
\AtEndDocument{DOC}\AtEndDocument{\_END}



% This is the real document class
\documentclass[useallsizes]{\myclass}
\usepackage{calc,fp,ifthen,fancyvrb}
\usepackage{fancyhdr,curves}
\usepackage[catperso]{ra} 
\usepackage[T1]{fontenc}
\usepackage[frenchb,english]{babel}
\usepackage{tralics,fink}
\usepackage{textcase}\usepackage{soul}\usepackage{amsmath}
% tralics.sty file (C) INRIA 2004
% This file is part of Tralics. Copyright Inria/ Jose Grimm/ Apics. 2004
% You can use this file in order to conditionnally compile
% a latex file with and without tralics

% This is how you can check that tralics is loaded
\ifx\tralicsversion\undefined\else\endinput\fi

% here we are outside tralics

\newcommand\xmllatex[2]{#2}


\endinput

%% Tralics knows these commands
\providecommand\IJ{IJ}
\providecommand\ij{ij}
\providecommand\xmlelement{}
\providecommand\AddAttToLast{}
\providecommand\AddAttToCurrent{}
\providecommand\xbox[1]{\hbox}
\providecommand\numberedverbatim{}
\providecommand\unnumberedverbatim{}


\begin{filecontents}{taux2.tex}
% aux file  for testing tralics
% this file contains nothing useful
\mytypeout{in file taux2.tex}
\endinput
The file should finish with a \endinput, but not on the last line. 
\end{filecontents}

\newcounter{hours}\newcounter{minutes}
\expandafter\let\expandafter\Twodigits\csname two@digits\endcsname

% Important note: this command prints the date and time of the current run
% when testing tralics, the result and evrything before it ignored; so this
% must be at the very start of the XML file. 
% In particular, in order to check the begin document hook, it is the start of
% the hook.
\newcommand{\printtime}{%
   \setcounter{hours}{\time/60}%
   \setcounter{minutes}{\time-(\value{hours}*60)}%
   \Twodigits\thehours h \Twodigits\theminutes min}
 \def\today{\ifcase\day\or
      1st\or 2nd\or 3rd\or 4th\or 5th\or
      6th\or 7th\or 8th\or 9th\or 10th\or
      11th\or 12th\or 13th\or 14th\or 15th\or
      16th\or 17th\or 18th\or 19th\or 20th\or
      21st\or 22nd\or 23rd\or 24th\or 25th\or
      26th\or 27th\or 28th\or 29th\or 30th\or
      31st\fi~\ifcase\month\or
      January\or February\or March\or April\or May\or June\or
      July\or August\or September\or October\or November\or
      December\fi\space \number\year}
\def\myprintdate{The time is \printtime, \today.\par}



\begin{document}

%% must reset
\setcounter{hours}{0}\setcounter{minutes}{0}


\chapter{First}

%
%
\iftrue %%%%%%%%%%%%%%%%%  \showthe \the
         %%%%%%%% replace \iffalse by \iftrue, will print a lot of things
%\def\Show#1{\the#1\showthe#1}
\def\Show#1{\the#1}
%\the parameter
\widowpenalty=3 \Show\widowpenalty
\parindent1.5pt \Show\parindent
\leftskip = 1pt plus 2fil minus 4fill \Show\leftskip
\thinmuskip = 3mu plus -2fil minus 4fill \Show\thinmuskip
% \the register
\count0=17 \Show{\count0}
\dimen0=17pt \Show{\dimen0}
\skip0=17pt plus 1 pt minus 2pt \Show{\skip0}
\muskip0=17mu plus 1 mu minus 2mu \Show{\muskip0}
\muskip3=3mu plus 4 fil minus \muskip0 \Show{\muskip3}
\setbox0=\hbox{$\global\muskip0=\lastskip $}\Show{\muskip0}
%\the codename 8bit-number
\Show{\catcode`\A}
\Show{\lccode`\B}
%\the special register
\Show\inputlineno
%\the  with fonts
\font\xa=cmr10 at 11truept
\fontdimen6\xa = 11pt \hyphenchar\xa=`\-
\Show{\fontdimen6\xa}
\Show{\hyphenchar\xa}
% \the\lastpenalty...
% \the\ defined character
\chardef\foo25 
\Show\foo
\Show\xa
\toks0={\foo = \foo} \def\foo{foo}
\Show{\toks0}
\fi %%%%%%%%%%%%%%%%%%%%%%%%%%%%%


%\recurse\def\recurse{\newcounter{toto}\recurse\recurse}\recurse
%
\makeatletter
\let\gobble\@gobble % Then we might say: \let\show\gobble.
\toksdef\toks@=0 \let\thezero\z@
\long\def\addto@hook#1#2{#1\expandafter{\the#1#2}}
\def\addtohookz#1{\addto@hook\toks@{#1}}
\def\copytomac#1{\xdef#1{\the\toks@}\toks@{}}
\def\execlist#1#2{\def\do##1{#1}\expandafter\exec#2\do\done}
\def\exec#1#2\done{\ifx\do#1\else\do#1\exec#2\do\done\fi}
\def\ifbracket{\@ifnextchar[}
\let\FirstOfOne\@firstofone
\let\kFirstOfTwo\@firstoftwo
\let\kSecondOfTwo\@secondoftwo 
%% These two are defined in the kernel
%\def\@firstoftwo#1#2{#1}%\def\@secondoftwo#1#2{#2}
\let\@xp=\expandafter
\long\def\@zifempty#1{\ifx @#1@\@xp\@firstoftwo\else\@xp\@secondoftwo\fi}
\long\def\@yifempty#1{\@xifempty#1@@..\@nil}
\long\def\@xifempty#1#2@#3#4#5\@nil{%
  \ifx#3#4\@xp\@firstoftwo\else\@xp\@secondoftwo\fi}

\let\ifempty\@iftempty\let\yifempty\@yifempty\let\bifempty\@ifbempty 
\let\zifempty\@zifempty
\let\Scanupdown\@scanupdown
\makeatother

\newcounter{nc} \setcounter{nc}{3} \ifnum\value{nc}=3 \else \bad\fi
\addtocounter{nc}{1} \ifnum\value{nc}=4 \else \bad\fi


\long\def\testeq#1#2{\def\tmp{#2}\ifx#1\tmp\else \toks0={#1wantd: ->#2.}
\typeout{\the\toks0}\show #1\uerror\fi}
\def\ytesteq#1#2{\def\tmp{#2}\ifx#1#2\else 
\typeout{Wanted}\show #2\show #1\uerror\fi}
\def\xtesteq#1#2{\xdef\tmp{#2}\ifx#1\tmp\else \toks0={#1wantd: ->}
\typeout{\the\toks0 #2.}\show #1\uerror\fi}

%\chapter{W}
\paragraph{Macroexpansion}
Version\tralicsversion.
\def\foo{foo}\foo=foo.
\def\foo{{foo}}\foo=foo.
\def\foo{\bf foo}{\foo=bold-foo} non-bold foo.
\def\foo{{\bf foo}}{bold-\foo=nonbold-foo}
\def\foo#1{#1#1}\foo{bar}=barbar
\def\foo#1{#1#1}{bold-\foo{\bf bar}=bold-barbar}
\def\foo#1{#1#1}{bold-\foo{{\bf bar}}=nonbold-barbar}
\def\foo#1#2{#1#2#1#2}\foo bar=babar
\def\foo#1#2{#1#2#1#2}\foo {ba}r=barbar

{\tracingall
firstofone \def\ABC#1{#1}
\ABC{\let\xfoo\foo}\ifx\xfoo\foo\else\uerror\fi
\FirstOfOne{\let\xfoo\foo}\ifx\xfoo\foo\else\uerror\fi
\relax{\let\xfoo\relax}\ifx\xfoo\relax\uerror\fi
}
Testing \symbol{`a}\symbol{`\d}dtohook.\newfont{\unusedfont}{somefontspecs}

\_\_\toks17{}\toks17{12345}\the\toks17
\_\_\toks1{}\toks1{12345}\the\toks1
{\_\_\makeatletter\global\toks@{}\toks@{12345}\the\toks@
\_\_\toks@\expandafter{\the\toks@ abc}\the\toks@}
empty{\makeatletter\the\toks@{}}empty
\addtohookz{a}
\addtohookz{b}
\addtohookz{c}
\copytomac\foo\foo=abc
\execlist\relax\foo % This does nothing
\execlist{+#1+}\foo % This generates +a++b++c+

=\hbox{Texte\par texte}=

aaaa~bbbb\verb+aaaa bbbb+ $a~b$
\def\myempty{}\ifx\myempty\empty\relax\else\uerror\fi

% This gives only cmr10
\edef\foo{\the\font\the\tenrm}
{\lccode`+=`c \lccode`-=`m \lccode`/=`r\lowercase{\def\test{+-/10+-/10}}
\ifx\foo\test\else\bad\fi}
\edef\foo{\meaning\tenrm}
\edef\yoo{\detokenize{select font cmr10 at 10.0pt}}
\xtesteq\foo\yoo

\makeatletter
\newcommand{\smaller}[1][1]{\larger[-#1]}
\newcommand{\larger}[1][1]{%
  \count@\@xfontsize \advance\count@#1\relax
  \@xfontsize \count@}
\testeq\sloppypar{\par\sloppy}
\testeq\endsloppypar{\par}
\testeq\null{\hbox{}}
\testeq\hb@xt@{\hbox to}
\testeq\thepage{\@arabic\c@page}
\testeq\labelitemi{\textbullet}
\testeq\labelitemii{\textendash}
\testeq\labelitemiii{\textasteriskcentered}
\testeq\labelitemiv{\textperiodcentered}
\testeq\endguillemets{\guillemotright}
\testeq\guillemets{\guillemotleft}
\testeq\@nnil{\@nil}
\testeq\@sanitize{\@makeother\ \@makeother\\\@makeother\$\@makeother\&%
   \@makeother\#\@makeother\^\@makeother\_\@makeother\%\@makeother\~}
\testeq\dospecials{\do\ \do\\\do\$\do\&%
   \do\#\do\^\do\_\do\%\do\~\do\{\do\}}
\testeq\obeyspaces{\@makeactive\ }
{\xdef\fooA{    }\obeyspaces\xdef\fooB{    }
\expandafter\gdef\expandafter\fooC\expandafter{ }}
\edef\foob{\space\space\space\space}\def\fooc{ }
\ifx\fooA\foob\bada\fi\ifx\fooB\foob \else \badb\fi\ifx\fooC\fooc \else \badc\fi
{\catcode`\_=13 \catcode`\#=13 \catcode`\&=13 \xdef\foo{_#&}}
=\foo\url{a~&#& _}=
{\@makeactive\ \gdef\foo{\@makeactive\ \let \nobreakspace}}
\ifx\foo\@vobeyspaces\else \bad \fi
\makeatother

%% test that newline is the same as space
\def\fooa #1 #2{\def\foob{#1 #2}}\fooa 12 3\testeq\foob{12 3}\fooa 12 
4\testeq\foob{12 4}\fooa 12 {3 }\testeq\foob{12 3 }\fooa 12 {3
}\testeq\foob{12 3 }
\let\foob \fooa\def\fooa #1
#2{\def\foob{#1
#2}}\ifx\fooa\foob \else\bad\fi
\fooa 12 3\testeq\foob{12 3}\fooa 12 
4\testeq\foob{12 4}\fooa 12 {3 }\testeq\foob{12 3 }
\def\foo #1fooa{\def\foob{#1}}
\foo fooa\testeq\foob{}
\foo 1fooa\testeq\foob{1}
\foo {2 2}fooa\testeq\foob{2 2}
\foo {fooa}fooa\testeq\foob{fooa}
\foo {fooa} fooa\testeq\foob{{fooa} }
\foo {2 2}foofooa\testeq\foob{{2 2}foo}
{\def~x y#1{\def\foob{#1}}~x
y 0\testeq\foob{0}~x y1\testeq\foob{1}\let\fooc~
\def~ y#1{\def\foob{#1}}~
y 0\testeq\foob{0}~ y1\testeq\foob{1}
\def~x
 y#1{\def\foob{#1}}\ifx\fooc~\else\show\fooc\show~\bad\fi~x
y 0\testeq\foob{0}~x y1\testeq\foob{1}
\def~ 
y#1{\def\foob{#1}}~
y 0\testeq\foob{0}~ y1\testeq\foob{1}}

%This is a test. An \unknown command. A math formula $a^$.
%\def\test#1#2?{#1,#2.}\test ok?\typeout{A message}

27$^{i\grave {e}me}$
27$^{ji\grave {e}me}$

\paragraph{hathat}
{1^^{^^ab2^^5e^ab3^^5e^5e^ab4\def\Abc{ok}\def\bAc{OK}\^^41bc\b^^41c}
{\catcode `\�=7 ��ab $x�2$ %next line should produce M
��
%$1^�=^^��T$ %% hat hat control-A 
$1^�=^^$ %% hat hat control-A 
}\def\msg{a message.^^J}

\def\foo#1#2#3{#1=#2=#3=}
^^5c^^66o^^6f^^^^0153^^^^0152^^^^0178
 ^^^^017b^^8?
Compare:\oe\OE\"Y\.Z

\def\foo{^^�}\def\Bar#1#2#3{\def\gee{#3}}
\expandafter\Bar\foo\testeq\gee{�}
{\catcode`\A=9 \def\foo{BAC}\testeq\foo{BC}}
{\catcode`^^ab=7 \let\xy^^ab�!b\foo \ifx\xyab\foo\else\bad\fi} % � 

{ %% ## in a group
\def\Ma#1{\def\Mb ##1{\xdef\xbar{#1##1}}}\Ma a\Mb b
\testeq\xbar{ab}
\catcode`A6 \def\fooA1A2{\xdef\xbar{A2A1}}\foo23
\testeq\xbar{32}
}


{\language=2 
Test for German thing
 OK: "a"o"u"e"i"A"O"U"I"E"s"z"S"Z"c"C"f"F"l"L"m"M"n"N"p"P"r"R"t"T
 Ok: "-"~"|""""="`"'"<">
}

{% Check that csname is locally relax 
\csname somecommand\endcsname
\csname par\endcsname
\ifx\somecommand\relax\else\bad\fi\ifx\par\relax\bad\fi
}\ifx\somecommand\relax\bad\fi

\paragraph{accents}
\label{1x}
\paragraph{latin extended-A}
\label{1y}
ligne 010x: \=A\=a\u A\u a\k A\k a \'C\'c\^C\^i \.C\.c\v C\v c\v D\v d
\noindent ligne 011x: ??\=E\=e\u E\u e\.E\.e\k E\k e\v E\v e\^G\^g\u G\u g
\noindent ligne 013x: \.I\i\IJ\ij\^J\^j\c K\c k?\'L\'l\c L\c l\v L\v l?
\noindent ligne 014x: ?\L\l\'N\'n\c N\c n\v N\v n?\NG\ng\=O\=o\u O\u o
\noindent ligne 015x: \H O\H o\OE\oe\'R\'r\c R\c r\v R\v r\'S\'s\^S\^s\c S\c s
\noindent ligne 016x: \v S\v s\c T\c t\v T\v t??\~U\~u\=U\=u\u U\u u\r U\r u
\noindent ligne 017x: \H U\H u\k U\k u\^W\^w\^Y\^y\"Y\'Z\'z\.Z\.z\v Z\v z?

testpar


\medskip

\noindent a paragraph


\begin{figure} \includegraphics{x} \end{figure}

\begin{figure}\includegraphics{y.ps}\label{xpd} \caption{A\par B}\end{figure}

\begin{figure}
\centerline{\includegraphics{x}}

\caption{Caption of the figure}
\end{figure}

\begin{figure*}
\def\IC#1{\includegraphics[width=#1pt]{x}}
\begin{tabular}{cc}
\IC{1}&\IC{2}\\
\IC{3}&\IC{4}\\
\end{tabular}
\end{figure*}

\begin{table*}
\def\IC#1{\includegraphics[width=#1pt]{x}}
\begin{tabular}{cc}
\IC{1}&\IC{2}\\
\IC{3}&\IC{4}\\
\end{tabular}
\caption{a table with a caption}
\end{table*}

\begin{glossaire} \glo{x1}{y1}\glo{x2}{y2}\end{glossaire}
\def\X{\pers{a}{b}} \def\Y{\pers{c}{d}}
\begin{participant} \X\Y \end{participant}
\begin{participante} \X\Y \end{participante}
\begin{participants} \X\Y \end{participants}
\begin{participantes} \X\Y \end{participantes}

\begin{participant} \pers{1}{2}[3], \pers{1}{2}, \end{participant}

\newenvironment{abstract}{}{}
\newenvironment{body}{}{}
\begin{abstract}abstract\end{abstract}
\begin{body}a body \end{body}
\begin{moreinfo}additional information\end{moreinfo}

% Accents
\def\xx#1{\xmllatex{\&\#x#1;}{}}
%\Large
ok

\begin{xmlelement*}{preview}
\begin{equation}
\begin{aligned}
x^2+y^2&=1   & 1    &=X^2+Y^2\\
x      &=0.01& 0.001&=X
\end{aligned}
\end{equation}
\end{xmlelement*}

\begingroup
\def\shoveleft{\multicolumn{1}{l}}
\def\shoveright{\multicolumn{1}{r}}
\begin{multline}
\sin=\cos\\
\shoveright{\cos}\\
\shoveleft{\sin}\\
1+2
\end{multline}
% The \endgroup should be here, but we want a \par token after
% the equation




ABCABCABC

\endgroup
\def\foo{X}
\output={\foo}\the\output\the\output

{\everyhbox= \csname non-existent\endcsname {aa}%
 \everyvbox={bb}\everyxbox={cc} \everymath{\alpha}
 \everydisplay{\beta} \tracingall
\xbox{foo}{\hbox{A\vbox{B}}\hbox{}}
$$ A\hbox{B $C$} D$$
}\xbox{foo}{\hbox{A\vbox{B}}\hbox{}}
$$ A\hbox{B $C$} D$$

{\def\:{\global\let\sptoken= } \:  % this makes \sptoken a space token
\gdef\space{ }
}

\count0=0 \count1=0 \count100=0 \count11=0 \count120=0  \count121=0 
{
  \count0=1 \global\count1=2
  \globaldefs=1
  \count100=3 \global\count11=4
  \globaldefs=-1
  \count120=5 \global\count121=6
}
\the\count0 \the\count1 \the\count100 \the\count11 \the\count120 \the\count121 
%\showthe\globaldefs
\globaldefs=0

{ASSTEST
\everymath \space \space \sptoken= \relax \space\relax\sptoken{foo} 
\everydisplay\relax\sptoken\everymath
 \pretolerance=1 \tolerance=2  \hbadness=3 \vbadness = 4
 \linepenalty =5 \hyphenpenalty=6  \exhyphenpenalty=7
 \binoppenalty=8 \relpenalty=9 \clubpenalty=10 \widowpenalty=11
 \displaywidowpenalty=12  \brokenpenalty=13 \predisplaypenalty=14
 \postdisplaypenalty=15 \interlinepenalty=16 \floatingpenalty=17
 \outputpenalty=18 \doublehyphendemerits=19 \finalhyphendemerits=20
 \adjdemerits = 21 \looseness =22 \pausing=23 \holdinginserts=24
 \tracingonline=25 \tracingmacros=26 \tracingstats=27 \tracingparagraphs=28
 \tracingpages=29 \tracingoutput=30 \tracinglostchars=31
 \tracingcommands=32 \tracingrestores=33 \language=34 \uchyph=35
 \lefthyphenmin=36 \righthyphenmin=37
 \globaldefs=-38 \defaulthyphenchar =39 \defaultskewchar=40
 \escapechar41 \endlinechar`A \newlinechar43 \maxdeadcycles44 %
 \hangafter 45 \fam46 \mag47 \delimiterfactor48 \time49 %
 \day50 \month51 \year52 \showboxbreadth53 \showboxdepth54 \errorcontextlines55.}%
{ \hfuzz5pt \vfuzz5pt \overfullrule=5pt \emergencystretch5pt
 \hsize10cm \vsize10cm \maxdepth3mm \splitmaxdepth=4mm \boxmaxdepth=5mm
 \delimitershortfall =5pt  \lineskiplimit-3mm \delimitershortfall=1mm
 \nulldelimiterspace=11mm \scriptspace=1cm \mathsurround=2mm
 \predisplaysize=1.2pt \displaywidth12cm \displayindent3pc
 \parindent3mm \hangindent2mm \hoffset1in \voffset1in


 \baselineskip =1pt plus 2pt minus 1cm
 \lineskip =2pt plus 2pt minus 1cm
 \parskip =3pt plus 2pt minus 1cm
 \abovedisplayskip =4pt plus 2pt minus 1cm
 \abovedisplayshortskip =5pt plus 2pt minus 1cm
 \belowdisplayskip =6pt plus 2pt minus 1cm
 \belowdisplayshortskip =7pt plus 2pt minus 1cm
 \leftskip =8pt plus 2pt minus 1cm
 \rightskip =9pt plus 2pt minus 1cm
 \topskip =10pt plus 2pt minus 1cm
 \splittopskip =11pt plus 2pt minus 1cm
 \tabskip =12pt plus 2pt minus 1cm
 \spaceskip =13pt plus 2pt minus 1cm
 \xspaceskip =14pt plus 2pt minus 1cm
 \parfillskip =15pt plus 2pt minus 1cm
}

$a+b$ $a\mathbin+b$


\begin{verbatim}
{\let\rm\bf \bf toto}
<!--this is a comment -->
&dollar; not &equals; EURO &excl;
  \end{verbatim}
%
\begin{rawxml}
{\let\rm\bf \bf toto}
<!--this is a comment -->
&dollar; not &equals; EURO &excl;
\end{rawxml}


$\hat{a} \acute{b} \bar{c} \dot{d} \breve{e} \check{f} \grave{g} \vec{h}
\ddot{k} \tilde{l}a\rightarrow b\longrightarrow c$





%\xmtext{nrarrc           "&#x02933;&#x00338}
%\xmtext{nrarrw           "&#x0219D;&#x00338}   




$\acute{x} \bar{x} \breve{x} \check{x} 
\ddddot{x} \dddot{x} \ddot{x} \dot{x} 
\grave{x} \hat{x} \mathring{x} \tilde{x}
\vec{x} \widehat{xyz} \widetilde{xyz}$







\begin{math}\alpha=\beta\end{math} and  \(\gamma=\delta\) and $\phi=\psi$;
and \ensuremath{0\neq\infty}

\begin{equation} 0=0 \end{equation}
\begin{equation*} 0=0 \end{equation*}
\begin{displaymath} 0=0 \end{displaymath}
\[ 0=0 \]
$$ 0 = 0 $$

\begin{equation}
\begin{split}
(a+b)^4 &= (a+b)^ 2 (a+b)^2           \\
        &= (a^2+2ab+b^2)(a^2+2ab+b^2) \\
        &= a^4+4a^3b+6a^2b^2+4ab^3+b^4 \\
\end{split}
\end{equation}


\begin{equation}
\begin{aligned}
(a+b)^4 &= (a+b)^ 2 (a+b)^2           \\
        &= (a^2+2ab+b^2)(a^2+2ab+b^2) \\
        &= a^4+4a^3b+6a^2b^2+4ab^3+b^4 \\
\end{aligned}
\end{equation}


\begin{align}
x^2+y^2&=1\\ x&=\sqrt{1-y^2}
\end{align}

\begin{eqnarray*}
x & \ll & y_{1} + \cdots + y_{n} \\
  & \leq &z
\end{eqnarray*}

\begin{eqnarray}
x & = &17y \\
y & > & a + b + c+d+e+f+g+h+i+j+ \\ % \nonumber semantic changed
  &   & k+l+m+n+o+p
\end{eqnarray}

$$EQ1$$ etc
$$EQ2$$ \par etc
$$EQ3$$ 

etc




\def\ok{4} \leavevmode
\begin{rawxml}
&#x31;
\end{rawxml}
\begin{latexonly}\def\ok{2}\ok\end{latexonly}
%\begin{htmlonly}
%\def\ok{ ?
%\end{htmlonly}
\begin{comment}
\def\ok{ ?
\end{comment}
%\begin{rawhtml}
%<p>\def\ok{ ?</p>
%\end{rawhtml}
\begin{LaTeXonly}
\def\ok{ ?
\end{LaTeXonly}
\begin{xmlonly}\def\ok{3}\ok\end{xmlonly}
\ok



%\Large
\xx{1e08} \xx{1e09} \xx{1e16} \xx{1e17} \xx{1e2e} \xx{1e2f}
ok


{\large X\textordmasculine}
\begin{align*}
u& \equiv v+1 \mod{n^2} \\
u& \equiv v+1 \bmod{n^2} \\
u&   =    v+1 \pmod{n^2} \\
u&   =    v+1 \pod{n^2} \\
\end{align*}


$$\frac12 \dfrac 12 \tfrac12 \qquad
{}^{\frac12 \dfrac 12 \tfrac12}$$
\def\N{\mathchoice{a}{b}{c}{d}}
$$\frac{\N\frac{\N+m}{2}} \N \quad
\tfrac{\N\frac{\N+m}{2}} \N \quad
\dfrac{\N\frac{\N+m}{2}} \N
%\quad \genfrac {}{}{}0{\N\frac{\N+m}{2}} \N
$$


\persA 1 2 3 
\pers {Rose} {Dieng-Kuntz} {Scientist} {Inria} [1956-2008, Research Director] [yes],
\pers {Rose} {Dieng-Kuntz} [Sophia]{Scientist} {Inria} [1956-2008, Research Director]. 



{\Huge 1\xx{1F4}\xx{1F5}
\xx{1CD}\xx{1CE}}
%\end{xmlelement*}
\'g\'G

$\bar{\dot {\rm o}}$
a - b -- c --- d

\def\List{{abcABC\oe\OE\o\O\ae\AE\dh\DH\dj\DJ\l\L\ng\NG\ss\SS\th\TH}}
%% There was a \expandafter here, not needed any more
\MakeUppercase\List\\
\MakeLowercase\List\\
Check:\MakeUppercase{\MakeLowercase{Foo\o\O}}=foo\o\o{}
Check:\MakeLowercase{\MakeUppercase{Foo\o\O}}=FOO\O\O{}


%\halign{#\global\let\xxx\cr\relax\cr\xxx}
%\show\xxx

\ignorespaces
1\penalty2 3

\def\thiswillfail{
$\radical -1 x
\radical 134217727 w
\radical 100 y$
\vadjust\relax\space {c}
\vadjust\relax\space ?
\vcenter to 5pt {toto}
XXXXX\vcenter to 5pt {\bf toto}x!
\insert 2000 !
}



{\def\n{\mathchoice{a}{b}{c}{d}}
$$\binom{\frac{\n+m}{2}} \n \quad
\tbinom{\frac{\n+m}{2}} \n \quad
\dbinom{\frac{\n+m}{2}} \n\qquad
{}^{\binom{\frac{\n+m}{2}} \n \quad
\tbinom{\frac{\n+m}{2}} \n \quad
\dbinom{\frac{\n+m}{2}} \n}$$
}



\def\XX{\nonscript\,}
$a\XX b_{c\XX d} a\XX b_{\textstyle c\XX d} $

$$\genfrac{}{}{}{}12
\genfrac[]{2pt}{2}{aaa}{bbb} 
\genfrac(){0cm}{0}nm
$$


\def\mychoice{\mathchoice{a}{b}{c}{d}}
$$\mychoice = \frac{\mychoice}{\mychoice} = 
 1_{\mychoice\frac{\mychoice}{\mychoice}} = 2^ {\mychoice\frac{\mychoice}{\mychoice}}$$
\begin{center}
$\mychoice = \frac{\mychoice}{\mychoice} = 
 1_{\mychoice\frac{\mychoice}{\mychoice}} = 2^ {\mychoice\frac{\mychoice}{\mychoice}}$
\end{center}



${\frac12}{x\displaystyle\frac 12} 
{x\textstyle\frac 12}{x\scriptstyle\frac 12}{x\scriptscriptstyle\frac 12} \qquad
{x\displaystyle a}+{x\textstyle a}+{x\scriptstyle a}+{x\scriptscriptstyle a}\qquad
\frac{x\displaystyle a}{x\textstyle b}+\frac{x\scriptstyle a}{x\scriptscriptstyle b}$



 $$a\over{b\over c}
$$

y\textasciiacute$x' x'' x''' x'''' u_2' v'^3_4$
 
\subparagraph{t}
\label{1z} \ref{1x}\ref{1y}\ref{1z}c
\xbox{bar}{\xdef\nbar{\the\XMLcurrentid}}
\xbox{foo}{\XMLref{1x}\XMLref[\nbar]{1y}}
Testing raw XML.
\begin{xmlelement}{foo}
\begin{xmlelement}{subelt1}
texte1
\end{xmlelement}
\begin{xmlelement}{subelt2}
texte2
\end{xmlelement}
\AddAttToLast{sb2-att}{value1}%
\AddAttToLast{sb3-att}{}%
\AddAttToCurrent{foo-att}{att-value'{}'}% in 2.13, this gives &#x201D; without braces
\end{xmlelement}


\def\Dmin{\operatorname{dmin}}
$\min _xf(x) >\Dmin _xf(x)$.

\newcommand\Operatorname[1]{%
  \mathmo{#1}%
  \mathattribute{form}{prefix}%
  \mathattribute{movablelimits}{true}%
}
\def\Dmin{\Operatorname{dmin}}
$\min _xf(x) >\Dmin _xf(x)$.


\begin{xmlelement}{foo}
\AddAttToCurrent{x}{1} % ignored
\AddAttToCurrent{y}{2} % ok
\AddAttToCurrent*{x}{3} % ok
\AddAttToCurrent{y}{4} % ignored
\end{xmlelement}

\begin{xmlelement*}{bar}
\AddAttToCurrent{x}{1}
\AddAttToCurrent{y}{2}
\end{xmlelement*}
code starts here\par

ok
\begin{xmlelement*}{bar}
\AddAttToCurrent{x}{1}
ok %
\AddAttToCurrent{x}{2}
\end{xmlelement*}


ok
\begin{xmlelement}{bar}
\AddAttToCurrent{x}{1}
ok %
\AddAttToCurrent{x}{2}
\end{xmlelement}

\expandafter\a\rq{e}=\'e %this tests \rq and expandafter
\expandafter\a\lq{e}=\`e %this tests \rq and expandafter

% ca chie
%\xbox{foo}{\let\FOO=xA\xbox{bar\let\FOO=y\foo}{B} C\foo}
\xbox{foo}{\let\FOO=xA\xbox{bar}{\let\FOO=yB\FOO} C\FOO}

\def\foo{}
\xbox{mybox\foo}{box contains foo}
\par

Optional argument trick.
\let\foo\undef
\newcommand\foo[2][\fi\if]{#2\iffalse#1\else[#1]\fi}
\foo{no-opt} and \foo[optional]{opt-bar-arg}

\newcommand\NCA{nca}
\newcommand{\NCB}[0]{ncb}
\newcommand{\NCC}[2]{ncc#1#2}
\newcommand{\NCD}[2][x]{ncd#1#2}
\NCA and \NCB and \NCC{arg1}{arg2} and 
\NCD[arg1]{arg2} and \NCD{arg2}


Another idea (heiko)
\let\foo\undef   %
\newcommand{\foo}{}%
\def\foo#1#{\Xexch{#1}} %
\newcommand{\Xexch}[2]{#2#1}  %
\foo{no-opt} and \foo[optional]{opt-bar-arg}

\countdef\foo 0 \xshow\foo


{\large x}{\Large x}
{\large x\smaller y\smaller z}{\Large x\smaller y\smaller z}

{ \let\show\gobble  \let\showthe\gobble
\makeatletter
\tracingall
  \newcounter{toto}
  \newcounter{titi}[toto]
  \newcounter{tata}[titi]
  \newcounter{tutu}[toto]
\show\cl@toto
\setcounter{toto}{10}
\setcounter{titi}{20}
\setcounter{tata}{30}
\setcounter{tutu}{40}
\expandafter\show\value{toto}\expandafter\showthe\value{toto}
\expandafter\show\value{titi}\expandafter\showthe\value{titi}
\expandafter\show\value{tata}\expandafter\showthe\value{tata}
\expandafter\show\value{tutu}\expandafter\showthe\value{tutu}
\stepcounter{toto} %  kills titi, tutu
\stepcounter{tata}\thetata=31, 
\stepcounter{titi}\thetata=0 % titi=1
\stepcounter{tutu} 
\thetoto\thetiti\thetata\thetutu=11101

\show\cl@toto
\show\cl@titi
\show\cl@tata
\show\cl@tutu
}%
\paragraph{Test verbatim}\setcounter{FancyVerbLine}{4}

Essai de verbatim 1) \verb*+ x y+ 2) \verb*! x y!
\numberedverbatim
\begin{verbatim}
test : !@#$%^&*()_$
test : {\foo\} et zxcvbnm,./
test ZXVBNM<>? ~
test \verb+\verb-xx-+
test ligatures: <<>>``''-- et --- !
! \end{verbatim} Not this one
\end{notverbatim}
\end {verbatim}

\gobble{
\end{verbatim}
}
\gobble{
\end{verbatim} %% make emacs happy
}

Fin du verbatim. \verb+\toto+ et \verb-\verb+ { } $$-.
\verb+\begin{verbatim}+ doit passer % avec ca \end{verbatim}
\unnumberedverbatim
Autre verbatim (sans num�ro)
\begin{verbatim}
test : !@#$%^&*()_$
test : {\foo\} et zxcvbnm,./
test ZXVBNM<>? ~
test \verb+\verb-xx-+
test ligatures: <<>>``''-- et --- !
test BL : \\738! et \\838!.
\end{verbatim}

 \begin {Verbatim}
Verbatim_line 
 \end {Verbatim}
 \begin {Verbatim}
Verbatim_line 
 \end {Verbatim} % this is a comment


Verbatim env with line numbers
\begin{Verbatim}                   [numbers=true]
TEST
\end{Verbatim}
and without
\begin{Verbatim}
[ok]TEST
\end{Verbatim}
\begin{Verbatim} %
[ok] ceci est en commentaire
TEST
\end{Verbatim}

\iffalse % emacs hack
\end{Verbatim}
\fi

\def\verbatimfont#1{{\it #1}}
\def\verbatimnumberfont{\large}
%\tracingall
\count3=4
\begin{Verbatim}[counter=3]
5,one line
\end{Verbatim}
\begin{Verbatim}[counter=03]
6,one line
\end{Verbatim}
\newcounter{vbcounter}
\setcounter{vbcounter}8
\begin{Verbatim}[counter=vbcounter]
9,one line
\end{Verbatim}
\begin{Verbatim}[counter=vbcounter]
10,one line
\end{Verbatim}
\let\verbatimfont\tt


\DefineShortVerb{\|}
Test of |\DefineShortVerb| and |\UndefineShortVerb|.
\DefineShortVerb{\+}
test 1 |toto| +x+ |+x-| +|t|+
\UndefineShortVerb{\+}
test 2 |toto| +x+ |+x-| +|t|+
espace: |+ +|
\DefineShortVerb{\*}
Verbatimfoo: *+ foo +*\verb+*foo*+\verb*+foo*+
Verbatimfoo: \verb|+ foo +*foo*foo*|.

\SaveVerb{FE}|}|\def\FE{\UseVerb{FE}}
\DefineShortVerb{\+}
\SaveVerb{VE}+|+\def\VE{\UseVerb{VE}}
\SaveVerb{DU}|$_|\def\DU{\UseVerb{DU}} %$
\UndefineShortVerb{\+}
\UndefineShortVerb{\|}
\UndefineShortVerb{\*}

\UseVerb{FE}
Test \FE,\VE, \DU.

\DefineShortVerb{\|}
\fvset{showspaces=true}|+ +|
\begin{verbatim}
Some Verbatim text.
\end{verbatim}
\fvset{showspaces=false}|+ +|, \verb+ +, \verb*+ +.
\begin{verbatim}
More Verbatim text.
\end{verbatim}
\UndefineShortVerb{\|}

\DefineVerbatimEnvironment{verbat�m}{Verbatim}{}
\begin{verbat�m}
Some Verbatim text.
\end{verbat�m}


\[\text{The sign function: \ } 
\mathcal{S}(x) = \begin{cases}
-1&x<0 \\
0&x=0 \\
1&x>0
\end{cases}
\]


%\tracingall
Verbatim test

\verb-\verb+ { } $$-.
\verb*-\verb+ { } $$-.
\let\verbatimfont\it
\verb +x+ !




\paragraph{ifthenelse}
{
\def\truefals {tru}
\newboolean{foo}
\def\Foo{fo}
\iffoo Bug1 \fi
\setboolean{\Foo o}{\truefals e}
\iffoo\else Bug2\fi

\ifthenelse{2=3}{bug}{ok1}
\ifthenelse{\isodd{2}}{bug2}{ok2}
\ifthenelse{\isodd{3}}{ok3}{bug3}
\ifthenelse{\boolean{foo}}{ok4}{bug54}
\setboolean{foo}{False}\ifthenelse{\boolean{foo}}{bug5}{ok5}
\ifthenelse{\lengthtest{1cm<1mm}}{bug6}{ok6}
\dimen0=1cm \dimen1=0.01pt \dimen2=1.01pt \dimen3=9876.54321533pt 
\ifthenelse{\lengthtest{\dimen0>\dimen1}}{ok7}{bug7}
\ifthenelse{2>1}{ok8}{bug8}
\newcommand{\BB}{\CC}\newcommand{\CC}{\DD}
\newcommand{\DD}{AA}\newcommand{\EE}{EE}
\ifthenelse{\equal{\BB}{\EE}}{bug9}{ok9}
\ifthenelse{\equal{\BB}{\CC}}{ok10}{bug10}
\ifthenelse{\equal{\DD}{\BB}}{ok11}{bug11}
\ifthenelse{\not\boolean{true}}{bug12}{ok12}
\ifthenelse{\not\not\boolean{true}}{ok13}{bug13}
\ifthenelse{\not\boolean{true}\and\boolean{true}}{bug}{ok14}
\ifthenelse{\not\boolean{true}\and\boolean{false}}{bug}{ok15}
\ifthenelse{\not\boolean{false}\and\boolean{true}}{ok16}{bug16}
\ifthenelse{\not\boolean{false}\and\boolean{false}}{bug}{ok17}
\ifthenelse{\not\(\boolean{true}\and\boolean{true}\)}{bug}{ok18}
\ifthenelse{\not\(\boolean{true}\and\boolean{false}\)}{ok19}{bug}
\ifthenelse{\not\(\boolean{false}\and\boolean{true}\)}{ok20}{bug}
\ifthenelse{\not\(\boolean{false}\and\boolean{false} good ?\)}{ok21}{bug}

\newcommand{\IX}[2][R]{\ifthenelse{\equal{#1}{D}}{d}{r}#2}
\IX[D]{d} et \IX{r}.

\newcommand{\QU}[2]{%
\ifthenelse {
  \(\equal{#1}{ENG} \and \equal{#2}{yes} \)
     \or
  \(\equal{#1}{FRE} \and \equal{#2}{oui} \)
}{``OK''}{``not OK''}}
You agree \QU{ENG}{yes} or don't \QU{ENG}{no}. 
D'accord \QU{FRE}{oui} ou pas \QU{FRE}{non}. 


\paragraph{CALC}

\newcounter{Ac}
\setcounter{Ac}{(1+2)*(3+4)-20}\theAc=1
\addtocounter{Ac}{(1*2)+(3*-4)+(34/7)}\theAc=-5

\newlength{\Bc}
\setlength{\Bc}{(1cm+2cm)*(3+4)-200mm} \the\Bc=28.4526pt
\setlength\Bc{\the\Bc*\ratio{25.4pt}{722.7pt}} \the\Bc=0.99985pt
\Bc=1in \setlength\Bc{\the\Bc * 100 / 7227} \the\Bc=0.99998pt
\Bc=1in \setlength\Bc{\the\Bc * \real{ 0.01383700013837}} \the\Bc=1.00018pt
\Bc=1cm \setlength\Bc{\the\Bc / \real{28.452755}} \the\Bc=0.99985pt
\Bc=1cm \setlength\Bc{\the\Bc * \ratio{254pt}{7227pt}} \the\Bc=0.99985pt
\Bc=1in \setlength\Bc{\the\Bc / \ratio{7227pt}{100pt}} \the\Bc=1.00018pt
\Bc=1IN \setlength\Bc{\the\Bc / \ratio{7227PT}{100pT}} \the\Bc=1.00018pt

{\catcode `\p=12\catcode`\t=12\gdef\ZPT{0.0pt}}
\Bc=1in \setlength\Bc{\the\Bc-72.26999pt}\edef\foo{\the\Bc}\xtesteq\foo\ZPT
\Bc=1pc \setlength\Bc{\the\Bc-12pt}\edef\foo{\the\Bc}\xtesteq\foo\ZPT
\Bc=1cm \setlength\Bc{\the\Bc-28.45274pt}\edef\foo{\the\Bc}\xtesteq\foo\ZPT
\Bc=1mm \setlength\Bc{\the\Bc-2.84526pt}\edef\foo{\the\Bc}\xtesteq\foo\ZPT
\Bc=1bp \setlength\Bc{\the\Bc-1.00374pt}\edef\foo{\the\Bc}\xtesteq\foo\ZPT
\Bc=1dd \setlength\Bc{\the\Bc-1.07pt}\edef\foo{\the\Bc}\xtesteq\foo\ZPT
\Bc=1cc \setlength\Bc{\the\Bc-12.8401pt}\edef\foo{\the\Bc}\xtesteq\foo\ZPT
\Bc=1sp \setlength\Bc{\the\Bc-0.00002pt}\edef\foo{\the\Bc}\xtesteq\foo\ZPT


\def\a{\b}\expandafter\let\a\c

\setcounter{hours}{3* \real{1.6} *\real{1.7}} \thehours
}
\count0 =2  \dimen0=3pt \dimen1=5pt
\dimen0 = \count0 \dimen1 \the\dimen0=10pt
\count0=123 \def\toto#1#2#3{#3#2#1} \expandafter\toto\the\count0=321

{
\IfFileExists{nohope}{\errmessage{bad1}}{}
\IfFileExists{\jobname}{}{\errmessage{bad2}}
\IfFileExists{X.tex}{\errmessage{bad3}}{}
\def\bad{\errmessage{BAD}}\let\ybad\bad
\def\mytypeout#1{\def\bad{\xbad}}
\def\foo{\ifx\bad\ybad\else\let\xbad\relax\fi\let\bad\ybad}
\def\IIFE#1{\InputIfFileExists{#1}{}{}}
\IIFE{taux2}\IIFE{nohope}\foo\IIFE{taux2}
\bad
}


{\def\mytypeout#1{ok\ignorespaces}
\def\IIFE#1{\InputIfFileExists{#1}{a}{b}}
\IIFE{taux2}\IIFE{nohope}\IIFE{taux2}
}

{\small\it\toks4={}\dimen0=1sp \count0=0
\setbox13=\xbox{foo}{\large x}a
\box13 \let\foo\relax\let\bar\relax\let\gee\relax}

\def\Test-#174{#1}
\count3=-27
\expandafter\Test\number\count3 4
\def\catletter#1{\catcode`#1=12 }
\def\somecats{\catletter\m\catletter\c\catletter\x\catletter\i
  \catletter\v\catletter\l}
{\somecats \gdef\Testmcmlxxxiv{}}
\expandafter\Test\romannumeral 1984
\def\catletter#1{\catcode`#1=12 }
\def\somecats{\catletter\M\catletter\C\catletter\X\catletter\I
  \catletter\V\catletter\L}
{\somecats \gdef\TestMCMLXXXIV{}}
\expandafter\Test\Romannumeral 1984



\setbox12=\xbox{foo}{ok}\setbox14=\xbox{foo}{ok}
\count3=12
\copy12 , and \copy 12 , and \box \count3 , and \box 12!
\usebox{14}\usebox{14}3

\def\dupbox{\setbox0=\lastbox \copy0 \box0 }
\xbox{foo}{ok} and \dupbox; \xbox{bar}{ok}\dupbox.
X\setbox0=\vsplit 25 to 4cm \box0Y.
X\setbox0=\vtop to 2cm{x}\box0Y.
X\setbox0=\vtop spread 2cm{x}\box0Y.
X\setbox0=\vbox to 2cm{x}\box0Y.
X\setbox0=\vbox spread 2cm{x}\box0Y.
X\setbox0=\hbox to 2cm{x}\box0Y.
X\setbox0=\hbox spread 2cm{x}\box0Y.


% Keep the dot. Otherwise, \the is expanded, 
Test{\count0=1000\countdef\spacefactor0\relax \skip0="DD DDPLUS2,5 \spacefactor\space.\the\skip0=236.47191pt plus 0.03815pt.}

{\large{\hbox{\normalsize x}}}
{\large a\hbox{\small x}b}
{\large a\hbox{\normalsize x}b}

test hbox A\hbox{a\par b}B.
{\tracingall \parbox[foo][bar][gee]{2cmPlus3mm}{some \it box content}.}

UPPER\uppercase \noexpand\testeq\relax\space \bgroup }\lowercase \relax\space \bgroup }CASE

{\makeatletter
\begingroup
  \catcode`P=12
  \catcode`T=12
  \lowercase{
    \def\x{\def\rem@pt##1.##2PT{##1\ifnum##2>\z@.##2\fi}}}
  \expandafter\endgroup\x
\def\strip@pt{\expandafter\rem@pt\the}
\dimen0=3pt
\dimen1=3.4pt
\strip@pt\dimen0
\strip@pt\dimen1

\def\do@noligs#1{%
  \catcode`#1\active
  \begingroup
     \lccode`\~`#1\relax
     \lowercase{\endgroup\def~{\leavevmode\kern\z@\char`#1}}}
\do@noligs A
}

{\makeatletter
\endlinechar-1
\count@0
\catcode0=13
\gdef\XML@tempa{
 \begingroup
   \uccode0\count@
  \uppercase{\endgroup
    \edef^^@{
      \ifnum\catcode\count@=11 %
        \noexpand\utfeightay\else\noexpand\utfeightax\fi
      \noexpand^^@}
    \expandafter\edef\csname 8:\string^^@\endcsname{\string^^@}}
 \ifnum\count@<127\advance\count@1 \expandafter\XML@tempa\fi}
\XML@tempa
}

\outer\def\ofoo{}\edef\foo{\noexpand\ofoo}

\includegraphics[angle=0,width=3cm,clip=]{Logo-INRIA-couleur.ducon}
\includegraphics[angle=0,width  =   %
                .5\textwidth,height=.3\textheight]{Logo-INRIA-couleur.ducon}
\includegraphics[angle=0,width=\textwidth,height=\textheight]{Logo-INRIA-couleur.ducon}
\includegraphics[angle=0, =foo,,width=3cm,clip]{Logo-INRIA-couleur.ducon}

{ \let\show\gobble \let\showthe\gobble
\def\test{\a\-\b}
  \show\test\showthe\escapechar\escapechar-1\showthe\escapechar\show\test}
\def\optionalequals{ \space=}
{\catcode`A\optionalequals2 A% } the A closes the brace
{\escapechar=-1 \edef\foo{\string\=}\testeq\foo{=}}
{\escapechar=-1 \edef\foo{\expandafter\string\csname\endcsname}
\edef\yoo{\detokenize{csnameendcsname}}\xtesteq\foo\yoo}
{\escapechar=0 \edef\foo{\string\=}\edef\yoo{\string^\string^@=}
\xtesteq\foo\yoo}
\xdef\foo{\string^^^^1234}\testeq\foo{^^^^1234}
{\escapechar=-1 \xdef\foo{\string\}}}
{\escapechar=`\+ \edef\foo{\string\=}\testeq\foo{+=}}
{\catcode0=12 \edef\foo{\meaning^^00} \tracingall
  \edef\yoo{\detokenize{the character }\string ^\string^\string@}\xtesteq\foo\yoo 
  \catcode0=11 \edef\foo{\meaning ^^00} 
  %% Note: \meaning^^00}  is interpreted as \meaning} which is wrong
  \edef\yoo{\detokenize{the letter }\string ^\string^\string@}\xtesteq\foo\yoo 
  \edef\foo{\meaning^\meaning_\meaning$%$
  \meaning&\meaning#\meaning\sptoken}
  \edef\yoo{\detokenize{superscript character }\string   ^\detokenize{subscript character %
     _math shift character $%$
    alignment tab character  &macro parameter character }\string#%
  \detokenize{blank space  } }\xtesteq\foo\yoo
  \edef\foo{\meaning\egroup\meaning\bgroup} 
 \edef\yoo{\detokenize{end-group character }\string}\detokenize{begin-group character }\string{}\xtesteq\foo\yoo 
}
{\catcode`\A=12 \catcode`\B=12\catcode`\R=12\gdef\fooval{ABAR}}
{\escapechar=`\A \edef\foo{\string\BAR}\ifx\foo\fooval\else \uerror\fi}
\string{\string}
\char`{\char`}
\expandafter\def\csname\string{\string}\endcsname#1#2{#2#1}
\expandafter\csname\string{\string}\endcsname XY

\def\recur#1{\csname rn#1\recur}
\def\rn#1{}
\def\rnm#1{\endcsname{#1}#1}
\def\replicate#1{\csname rn\expandafter\recur 
  \romannumeral\number\number#1 000\endcsname\endcsname}
0{\tracingall
\dimen0=4sp
\replicate{\dimen0}{{{{\P}}}}
}

  \def\factorial#1{%
    \ifnum\numexpr#1>1
      \number \numexpr#1*\factorial{(#1-1)}\expandafter\relax
    \else
      \expandafter1\fi}
\edef\foo{\factorial{0}\factorial{0} \factorial{0}4}\testeq\foo{11 14}
\edef\foo{\factorial{1}4}\testeq\foo{14}
\edef\foo{\factorial{2}4}\testeq\foo{24}
\edef\foo{\factorial{3}4}\testeq\foo{64}
\edef\foo{\factorial{5}\factorial{5} \factorial{5}4}\testeq\foo{120120 1204}
\edef\foo{\factorial{6}4}\testeq\foo{7204}

%\newcommand{\ba}{OK}
%\newcommand\bab{OK}
%\newcommand{\ba\bb}{OK}
%\newcommand{1}{OK}
%\newcommand{\Athapascan0}{ok}

% testing optional chars via \ifbracket
\def\wopt{\ifbracket\xwopt\ywopt}
{\def\ywopt#1#2{}\let\xwopt\bad \wopt 12}
{\def\xwopt[#1]{}\let\ywopt\bad \wopt [12]}
{\let\=\wopt\def\ywopt#1#2{}\let\xwopt\bad \= 12}
{\let\=\wopt\def\ywopt#1#2{}\let\xwopt\bad \=12}
{\let\=\wopt\def\xwopt[#1]{}\let\ywopt\bad \=[12]}
{\let\=\wopt\def\xwopt[#1]{}\let\ywopt\bad \= [12]}
{\let\=\wopt\def\xwopt[#1]{}\let\ywopt\bad \= 
[12]}
{
\def\; {}\futurelet\SPtokena\; % comment required
\def\;#1{}\futurelet\SPtokenb\; 
0\xshow\SPtokena\xshow\SPtokenb
% This fails in the current version.
%\ifx\SPtokena\SPtokenb \else \bad\fi
}
\expandafter\ifx\csname @sptoken\endcsname\sptoken \else \oops\fi 
\expandafter\ifx\csname @@sptoken\endcsname\sptoken \oopps\fi 

\paragraph{Biblio}
\def\wopt{\ifbracket\xwopt\ywopt}
\def\xwopt[#1]#2{With opt #1#2}
\def\ywopt#1{Without opt #1}
This is some text
\footcite{toto,titi} \cite[text]{toto,titi} \cite{mathml2}
\cite[foo]{mathml2}
% Check all possibilities
\cite[foot][p32]{toto,titi}\cite[foot][p32]{toto}\cite{mathml2,mathml2}
\cite[text]{toto,titi}\cite[refer][]{toto,titi}
\footcite{toto,titi}\footcite[p3]{toto,titi}
\nocite[foot]{toto,titi}\nocite{mathml2}
%\bibliographystyle{Plain}
%\bibliographystyle{program:cat}
%\bibliographystyle{bibtex:plain}

\xbox{bibitem}{\XMLsolvecite[][foot]{toto}{key1}}
\xbox{bibitem}{\XMLsolvecite[][foot]{titi}{key2}}

MINIPAGE
\Han\ designed \xmllatex{\pdf\<TeX/\>}{}. \l\ and \strok{l}.\\
In Tralics version 2, you can say Th{\'{\^e}}

\def\test#1#2{\mathtt{#1#2}\mathcal{#1}\mathbf{#1#2}\mathrm{#1#2}
\mathit{#1#2}\mathbb{#1}\mathsf{#1#2}\mathnormal{#1#2}}
\begin{equation}\begin{split}
\test Aa &\test Bb\\
\test Cc &\test Dd\\
\test Ee &\test Ff\\
\test Gg &\test Hh\\
\test Ii &\test Jj\\
\test Kk &\test Ll\\
\test Mm &\test Nn\\
\test Oo &\test Pp\\
\test Qq &\test Rr\\
\test Ss &\test Tt\\
\test Uu &\test Vv\\
\test Ww &\test Xx\\
\test Yy &\test Zz\\
\end{split}
\end{equation}

${\cal ABCDEFGHIJKLMNOPQRSTUVWXYZ}$
\begin{center} xx

{ \begin{minipage}{2cm}abc\end{minipage} \par xx}
\end{center}
\begin{minipage}[t]{2cm}abc\end{minipage}
\begin{minipage}[b]{2cm}abc\end{minipage}
\begin{minipage}[c]{2cm}abc\end{minipage}
\begin{minipage}[c][xxx][s]{2cm}abc\end{minipage}


\makebox[1cm][r]{some text}
\framebox[1cm][r]{some text}



\wopt[AA]{BB}\wopt{CC}
\def\citenotext{\let\originalcite\cite \def\cite ##1##{\originalcite}}
\let\originalwopt\wopt \def\wopt #1#{\originalwopt}
\citenotext
\cite[foo]{mathml2}\wopt[AA]{BB}\wopt{CC}


{%\tracingall
   {\def\foo{\gdef\xbar{OK}}\catcode`x=0 xfoo}
   \testeq\xbar{OK}
   {\def\foo{\gdef\xbar{notOK}}\catcode`x=0xfoo}
   \testeq\xbar{OK}
}

{\catcode`A1\catcode`B2 
\def\fooA2B\testeq\fooA2B
\def\xbarA\bgroup\def\fooA3B\egroupB\xbar\testeq\foo{2}}

{\catcode`A6 \def\fooA1A2{\def\xbar{A2A1}}\foo23\testeq\xbar{32}}

\paragraph{Letter c}
testchar{\language=1
\char60\ \char0\ \char 10\ \char233\ \char 300\
\char126\ \char45- \char39' \char96` \char60< \char62>
\char59\char58\char33\char63\char171\char 187\char38
\verb@~--``xx''<<>>;:!?��&@
~--``<<>>;:!?��''\&
}
{\language=0
\char60\ \char0\ \char 10\ \char233\ \char 300\
\char126\ \char45- \char39' \char96` \char60< \char62>
\char59\char58\char33\char63\char171\char 187\char38
\verb@~--``<<>>;:!?��&''@
~--``<<>>;:!?��''\&
}
$ \bf a \mathrm{a \mathcal{c} d}$

$a^c_b$
$a\sb b\sp c$

%ERROR{\catcode`x7 \catcode`y=8 a^b_c xy\sp\sb}
{\def\xbar#1{\egroup\edef\FV{\meaning#1}\xshow\FV\FV}
\def\foo{\bgroup\catcode32=11\catcode`\%=11 \xbar}\foo \foo%\foo=}


$%
\widetilde{abc} \widehat{abc} \overleftarrow{abc} \overrightarrow{abc}
\overline{abc} \underline{abc} \overbrace{abc} \underbrace{abc}
\underleftarrow{abc} \underrightarrow{abc}
$ 

\wd0=3cm %\showthe\wd0
\spacefactor-12345
%\showthe\spacefactor
$x\Box$
\def\R{\mathbf{R}}

\def\MAT#1{\begin{array}{c#1}1&22\\3&4\end{array}}


\[\left|
\begin{array}[ignored]{lcr}
AAAAAAA&BBBBBCC&CCCCCCC\\
A&B&C\\
\multicolumn{1}{c}{A}&\multicolumn{1}{c}{B}&\multicolumn{1}{c}{C}\\
\multicolumn{1}{r}{A}&\multicolumn{1}{r}{B}&\multicolumn{r}{c}{C}\\
\multicolumn{1}{l}{A}&\multicolumn{1}{l}{B}&\multicolumn{l}{c}{C}\\
\MAT l&\MAT c&\MAT r\\
\multicolumn{2}{c}{0123456789abcdef}&C\\
A&\multicolumn{2}{c}{0123456789abcdef}\\
\multicolumn{2}{r}{0123456789abcdef}&C\\
A&\multicolumn{2}{l}{0123456789abcdef}\\
A&B&C\\
\end{array}
\right|\]



\begin{eqnarray*}
\left\{\begin{array}{lcl}
\dot{x} & = & Ax+g(x,u)\\
 y & = & Cx \\
 \multicolumn{3}{l}{x\in \R^n} 
\end{array}
    \right.
\end{eqnarray*}


\def\makeboxes{%
\setbox0=\hbox{123}%
\setbox1=\xbox{foo}{456}%
\setbox2=\xbox{foo}{789\xbox{bar}{abc}}}
1.\makeboxes \copy0 \copy1 \copy2 \copy0 \copy1 \copy2\par
2.\makeboxes \box0 \box1 \box2 \box0 \box1 \box2\par
3.\makeboxes \unhcopy0 \unhcopy1 \unhcopy2 \unhcopy0 \unhcopy1 \unhcopy2\par
4.\makeboxes \unhbox0 \unhbox1 \unhbox2 \unhbox0 \unhbox1 \unhbox2\par
5.\makeboxes \unvcopy0 \unvcopy1 \unvcopy2 \unvcopy0 \unvcopy1 \unvcopy2\par
6.\makeboxes \unvbox0 \unvbox1 \unvbox2 \unvbox0 \unvbox1 \unvbox2\par


FILL \hfil \hfill \hfilneg \hss \vfil  \vfill \vfilneg \vss

\def\foo{123}\afterassignment\foo\setbox0=\hbox{4}
\makeatletter
\def\openup{\afterassignment\@penup\dimen@=}
\def\@penup{\advance\lineskip\dimen@
  \advance\baselineskip\dimen@
  \advance\lineskiplimit\dimen@}
\makeatother
\openup1234pt
%\showthe \lineskip\showthe\baselineskip \showthe \lineskiplimit
\prevdepth=12345pt
\pagestretch=1pt
\pagefilstretch=1pt
\pagefillstretch=1pt
\pagefilllstretch=1pt
\pageshrink=1pt
\pagetotal=1pt
\pagegoal=1pt
\deadcycles-25
\insertpenalties=13

ABC\parshape3 1pt 102pt  3pt 104pt 5pt 106pt parshape
\count`\^^M=\catcode'53 \ifnum\count13=12 \else \bad\fi
\parindent=3pt
\bgroup %% keep this local since \z@ might be \count13
\dimen`\^^M=4\parindent \ifdim\dimen13=12pt \else \bad\fi
\egroup

{\let\show\xshow
{\def\B#1{a} \let \A\B \show\A}
{\catcode `A=13 \def A{b} \let\A A \show\A}
{\catcode `A=13 \def\B#1{c} \let A\B \showA}
{\catcode `A=13  \let A~ \showA}
{\catcode `A=13 \letA=1 \showA}
{\let\sp=^ \show\sp}
{\let\egroup=} \show\egroup}
}

{\let \show\gobble
\catcode `A=13 \letA=1 \showA \count0=0A1 \showthe\count0}
$a'^2$ $a'\sp2$

{\chardef\foo124 \count\foo13 }
\let\foo W
{\futurelet\bar\relax\relax \global\futurelet\foo\relax2}
\xshow\foo \xshow\bar

a) \begin{math} x1 \end{math}
b) \( x2 \)
c) \begin{displaymath} x3 \end{displaymath}
d) \[ x4 \]
e) \begin{equation} x5 \end{equation}

text A \\* text B \\[3mm] text C
\begin{center}
line one\\
this is the second line
\end{center}


text A \bigskip text B \smallskip text C \medskip text D
\begin{center}
short line \bigskip a longer line \smallskip a still longer line \medskip 
a very long line of text
\end{center}

{\rm Some text} {\sf Some text} {\tt Some text}
{\bf Some text} {\it Some text} {\sc Some text} {\sl Some text}

1\textsuperscript{st} x\textsuperscript{i�me} y\textsubscript{some text}
$1^e$\overline{foo}\underline{bar}.

\setcounter{enumi}{\value{enumii}}
\renewcommand\theenumiii{\alph{enumiv}}\stepcounter{enumiv}

$\notin$

1\textsuperscript{st} x\textsuperscript{i�me} y\textsubscript{some text}
$1^e$\overline{foo}\underline{bar}.
\lsc{Foo}\fsc{Bar}


\def\Test{
\textrm{Textrm} \textsf{Textsf} \texttt{Texttt}
\textbf{Testbf} \textit{Textit} \textsc{Textsc} \textsl{Textsl}
\textnormal{Textnormal} \textmd{Textmd} \emph{Emph} \textup{Textup}
\par\noindent}

{\rmfamily \mdseries \upshape rmmdup \Test}
{\sffamily \mdseries \upshape sfmdup \Test}
{\ttfamily \mdseries \upshape ttmdup \Test}
{\rmfamily \bfseries \upshape rmbfup \Test}
{\sffamily \bfseries \upshape sfbfup \Test}
{\ttfamily \bfseries \upshape ttbfup \Test}
%
{\rmfamily \mdseries \itshape rmmdit \Test}
{\sffamily \mdseries \itshape sfmdit \Test}
{\ttfamily \mdseries \itshape ttmdit \Test}
{\rmfamily \bfseries \itshape rmbfit \Test}
{\sffamily \bfseries \itshape sfbfit \Test}
{\ttfamily \bfseries \itshape ttbfit \Test}
%
{\rmfamily \mdseries \scshape rmmdsc \Test}
{\sffamily \mdseries \scshape sfmdsc \Test}
{\ttfamily \mdseries \scshape ttmdsc \Test}
{\rmfamily \bfseries \scshape rmbfsc \Test}
{\sffamily \bfseries \scshape sfbfsc \Test}
{\ttfamily \bfseries \scshape ttbfsc \Test}
%
{\rmfamily \mdseries \slshape rmmdsl \Test}
{\sffamily \mdseries \slshape sfmdsl \Test}
{\ttfamily \mdseries \slshape ttmdsl \Test}
{\rmfamily \bfseries \slshape rmbfsl \Test}
{\sffamily \bfseries \slshape sfbfsl \Test}
{\ttfamily \bfseries \slshape ttbfsl \Test}



\begin{center}
1234567890 \\
\oldstylenums{1234567890}
\end{center}


{Aa\tiny Bb  \scriptsize Cc \footnotesize Dd \small Ee \normalsize Ff 
\large Gg \Large Hh \LARGE Ii  \huge Jj  \Huge Kk}

\vfil \vfill \hfil \hfill

\begin{equation}
\begin{array}{lcl}
\dot{x} & = & Ax+g(x,u)\\[2mm]
 y & = & Cx \\
 \multicolumn{3}{l}{x\in R^n} 
\end{array}
\end{equation}

\begin{tabular}{|ll|rr|cc|}
\hline a&b&c&d&e&f\\
aaa&bbb&ccc&ddd&eee&fff\\[3mm]
\hline
A&\multicolumn{3}{l}{BCD}&E&F\\
\cline{1-3}\cline{6-6}
aaa&bbb&ccc&ddd&eee&fff\\\hline
\end{tabular}
 
\part{A} \label{label-A}
\chapter{B} \label{label-b}
\makeatletter 
\section{C\\c} \label{label-c}
\let\@headercr d
\subsection{D\\[2mm]d} \label{label-d}
\subsubsection{E} \label{label-e}
\makeatother
\paragraph{F} \label{label-f}
\subparagraph{G} \label{label-g}
Text \par more text
\ref{label-A} \ref{label-b} \ref{label-c}
\ref{label-d} \ref{label-e} \ref{label-f} \ref{label-g}

\paragraph{A}
\label{pa}
\ref{pa} \ref{pb} \ref{pc} \ref{pd} \ref{pe} \ref{pf}
\ref{pa} \ref{pb} \ref{pc} \ref{pd} \ref{pe} \ref{pf}
\paragraph{B}
\label{pb}
\begin{equation}x \label{pc} \end{equation}
\begin{figure} \includegraphics{x}\label{pd} \end{figure}
\begin{itemize} \item a \label{pe}\item[B] b \label{pf} \end{itemize}

\paragraph{xxxxx}123\subsubsection{xxxxx}456
\endsec{\subsubsection} ENDOFSUBSECTION
\endsec{}\xmlemptyelt{appendix}

Test tilde:~\verb=~=$a~b$.
\Href{\url{a~b\~n_d}}{a~b\~n_d}
\href{a~b\~n_d}{\url{a~b\~n_d}}
\href{url&-~}{text~a}
\url{a~b\~n_d}
\footnote{URL in footnote....
\Href{\url{a~b\~n_d}}{a~b\~n_d}
\href{a~b\~n_d}{\url{a~b\~n_d}}
\href{url&-~}{text~a}
\url{a~b\~n_d}}


$$ \arccos^2 (x),\,  \arcsin^2(x) ,\,  \arctan^2(x) ,\,  \arg^2(x),\,
   \cos^2(x) ,\,   \cosh^2(x)  ,\, \cot^2(x)$$

$$ \coth^2(x) ,\,  \csc^2(x),\,   \deg^2(x),\,   \dim^2(x),\, \exp^2(x),\,
   \hom^2(x),\,   \ker^2(x) ,\,   \lg^2(x)$$



$%
\stackrel{j}{\longrightarrow} \overset{*}{X} \underset{*}{X}
\sqrt{abc} \sqrt[n]{abc} \root n \of{abc}
\frac{abc}{xyz} \dfrac{abc}{xyz}
$



$$\binom{\frac{n+m}{2}} n \quad
\tbinom{\frac{n+m}{2}} n \quad
\dbinom{\frac{n+m}{2}} n\qquad
{}^{\binom{\frac{n+m}{2}} n \quad
\tbinom{\frac{n+m}{2}} n \quad
\dbinom{\frac{n+m}{2}} n}$$



$\Leftrightarrow$
$\sharp$ \sharp

1\textsuperscript{st} x\textsuperscript{i�me} y\textsubscript{some text}. 

$$\genfrac{}{}{}{}12
\genfrac[]{2pt}{2}{aaa}{bbb}
\genfrac(){0cm}{0}nm
$$



\newcommand\dean[2]{%
  \xbox{dean}{\xbox{first-name}{#1}\xbox{last-name}{#2}}}
\dean{Ingo}{Zschoch}

\newcommand\Date[2]{%
  \xbox{date}{\AddAttToCurrent{type}{#2}#1}}
\Date{01-01-2004}{defense}



$\genfrac{}{}{}0ab 
\genfrac{}{}{}1ab 
\genfrac{}{}{}2ab 
\genfrac{}{}{}3ab $

${a\over b}\quad {a\above1pt b}\quad {a\atop b}$
${a\overwithdelims() b}\quad {a\abovewithdelims()1pt b}\quad
{a\atopwithdelims() b}$
$\dimen0=2pt\genfrac(){\dimen0 }{}{foo}{bar}$


$$\frac{\frac{n+m}{2}} n \quad
\tfrac{\frac{n+m}{2}} n \quad
\dfrac{\frac{n+m}{2}} n$$


$$  \ln^2(x),\, \log^2(x),\,  \Pr^2(x),\, \sec^2(x),\, \sin^2(x) ,\,
 \sinh^2(x),\,  \tan^2(x),\, \tanh^2(x)$$

$$ \det_{x=1},\, \gcd_{x=1},\, \inf_{x=1},\, \injlim_{x=1},\,
 \lim_{x=1},\, \liminf_{x=1},\,\limsup_{x=1},\,
 \max_{x=1},\, \min_{x=1},\, \projlim_{x=1},\, \sup_{x=1}
$$



$$
\sup_{x=2}y~
\frac{1}{k}\log_2 c(f)\;
\tfrac{1}{k}\log_2 c(f)\;
\sqrt{\frac{1}{k}\log_2 c(f)}\;
\sqrt{\dfrac{1}{k}\log_2 c(f)}
$$





$\genfrac(){1mm}{3}{aaaaaaaaaaaaaaaaaa}{111111111}$
$\genfrac(){0pt}{3}{aaaaaaaaaaaaaaaaaa}{111111111}$
$\genfrac{}{}{}{0}{abc}{def}$
$\genfrac{}{}{}{1}{abc}{def}$
$\genfrac{}{}{}{2}{abc}{def}$
$\genfrac{}{}{}{3}{abc}{def}$
%$a\atop b$  $c\over d$


{
1\Href{a}{b}, 2\Href{a\~n}{b}, 3\Href{b}{a\~n},
4\Href{\url{a~b\~n_d}}{a~b\~n_d},
5\url{a~b\~n_d},\\
\let\urlfont\tt
1\Href{a}{b}, 2\Href{a\~n}{b}, 3\Href{b}{a\~n},
4\Href{\url{a~b\~n_d}}{a~b\~n_d},
5\url{a~b\~n_d}.
}



$\tilde a,\tilde z, \tilde{a+b}$

$\alpha \beta \gamma \delta \epsilon \varepsilon \zeta \eta
\theta \iota \kappa \lambda \mu \nu \xi \pi \rho \sigma \tau
\upsilon \phi \chi \psi \omega \varpi \varrho \varsigma \varphi
\varkappa \vartheta$
$\Gamma \Delta \Theta \Lambda \Xi \Sigma \Upsilon \Phi \Pi \Psi \Omega$





$\coprod \bigvee \bigwedge \biguplus \bigcap \bigcup \int \sum 
 \prod \bigotimes \bigoplus \oint \bigsqcup  \bigodot$


$ \pm \mp \times \div \ast \star \circ \bullet \cdot 
\cap \cup \uplus \sqcap \sqcup \vee \wedge \setminus \wr
\diamond \bigtriangleup \bigtriangledown \triangleleft \triangleright
\oplus \ominus \otimes \oslash \odot \bigcirc \dagger \ddagger \amalg
$


$\le \leq \succ \simeq \parallel \subseteq \sqsubset \doteq
\ge \geq \sim \mid \subset \subseteq \ne \neq \frown \vdash
\equiv \perp \ll \supset \cong \smile \in \dashv 
\models \preceq \gg \approx \Join \sqsubseteq \ni
\prec \succeq \asymp \bowtie \sqsupseteq \propto$


$\wp$
{
\setlength\unitlength{1mm}
\begin{picture}(50,-30)(-10,10)
%\linethickness{1pt}
\qbezier(0,0)(10,30)(50,30)\tracingall
\setlength\parindent{13sp}
\line(0,\parindent){25}
\end{picture}
}

\def\test#1{\csname f#1#1\endcsname}
\def\foo{bar}
\def\foobar{\def\result{ok}}
\csname foo\test o\endcsname
\testeq\result{ok}

{
\def\double#1{#1#1}
\let\startmacro\csname
\let\endmacro\endcsname
\def\foo{bar}
\def\foobar{ok}
\startmacro foo\startmacro f\double o\endmacro\endmacro
}

{
\def\double#1{#1#1}
\newboolean{g\double o}
\def\swap{\setboolean{g\double o}{\ifgoo fals \else TRU\fi e}}
\ifthenelse{\boolean{goo}}{\bad}{oka}
\swap 
\ifthenelse{\boolean{goo}}{okb}{\bad}

\ifthenelse{\boolean{goo} \bof !!}{okb}{\bad}
}

{
\newboolean{cA}\newboolean{cB}\newboolean{cC}\newboolean{cD}
\def\Test{ %
\ifthenelse{\( \boolean{cA} \and \boolean{cB} \) \or \( \boolean{cC} \and 
\boolean{cD} \)}{0}{1}}
\def\testa{\setboolean{cA}{true}\Test \setboolean{cA}{false}\Test}
\def\testb{\setboolean{cB}{true}\testa \setboolean{cB}{false}\testa}
\def\testc{\setboolean{cC}{true}\testb \setboolean{cC}{false}\testb}
\def\testd{\setboolean{cD}{true}\testc \setboolean{cD}{false}\testc}
\testd

\ifthenelse{\not 1<2 \or \NOT 3<4}{\bad}{ok}

\ifthenelse{\not 1<2 \or \NOT 3<4}{\bad}{ok}
\ifthenelse{\isundefined{\or}}{\bad}{ok}
\ifthenelse{\isundefined{\xor}}{ok}{\bad}
\ifthenelse{\isundefined{}}{\bad}{ok}
\ifthenelse{\isundefined{\relax \and \undef}}{\bad}{ok}
}

\countdef\n 3
\n=5
\begingroup\aftergroup\edef\aftergroup\asts\aftergroup{
\loop\ifnum\n>0 \aftergroup*\advance\n-1 \repeat
\aftergroup}\endgroup
\asts


{\makeatletter\catcode`\^^M=\active % these lines must end with %
  \ifx^^M\par\else\bad\fi
  \def\foo{\@makeactive\^^M\let^^M\par}% 
%\meaning\foo\meaning\obeylines % 
  \ifx\foo\obeylines\else\bad\fi % 
  \gdef\obeylines{\catcode`\^^M\active \let^^M\par}%
  \global\let^^M\par} % this is in case ^^M appears in a \write
\begingroup
  \obeylines%
  \gdef\VEE{\obeylines%
    \def^^M{\verbegroup \verberrorA}}%
\endgroup
\let\VBG\empty
\def\verbegroup{\global\let\VBG\empty\egroup}

\def\sverb{ \bgroup \VEE \tt \ssverb}
\def\ssverb#1{%
  \catcode`#1\active
  \lccode`\~`#1%
  \gdef\VBG{\verbegroup\verberrorB}
  \aftergroup\VBG
  \lowercase{\let~\verbegroup}}%

{\tracingall \sverb x y x 
%\def\test#1{{#1}}\test{\sverb x y x}
}

\the\lastskip\the\lastpenalty\the\lastkern

\toksdef\foo 11
\newtoks\Bar
\foo={123 456} \the\Bar

\toksdef\foo 11 \def\Thefoo{\the\foo}
\def\AAA{A}
\foo={\AAA} \def\Foo{\AAA}
\edef\test{\Thefoo\Foo}
\let\AAA\relax

{\makeatletter %\tracingall
\newcounter{foo}
\def\OO{oo}
\setcounter{f\OO}{17}
\def\setcounter#1#2{%
  \@ifundefined{c@#1}%
    {\@nocounterr{#1}}%
    {\global\csname c@#1\endcsname#2\relax}}
\setcounter{f\OO}{18}
}

\makeatletter
\@ifundefined{par}{\bad}{}\@ifundefined{mypar}{}{\bad}
\tralics@ifundefined{par}{\xbad}{}\tralics@ifundefined{myspar}{}{\bad}
\newif\ifok
\def\test{\@ifstar\oktrue\okfalse}
\test*\ifok\else\typeout{bug in ifstar1}\error\fi
\test\ifok\typeout{bug in ifstar2}\error\fi
\expandafter\test\space*\ifok\else\typeout{bug in ifstar3}\error\fi
\expandafter\test\space\ifok\typeout{bug in ifstar4}\error\fi

%\tracingall
\def\normal#1#2{=#1=#2=}
\def\withbrackets[#1]#2{\normal{#1}{#2}}
\def\withoutbrackets#1{\normal{}{#1}}
\def\foo{\@ifnextchar[\withbrackets\withoutbrackets}
\def\xfoo{\@testopt\withbrackets{25}}
\foo[1]{2}\foo34 \par
\xfoo[1]{2}\xfoo34

{
\def\@ifstar#1{\@ifnextcharacter*{\@firstoftwo{#1}}}
\def\@ifplus#1{\@ifnextcharacter+{\@firstoftwo{#1}}}
\newif\ifseenplus\newif\ifseenstar
\def\testa#1{\@ifstar{\seenstartrue\testb{#1}}{\seenstarfalse\testb{#1}}}
\def\testb#1{\@ifplus{\seenplustrue#1}{\seenplusfalse#1}}
\def\test{\testa{\testaux}}
\def\testaux#1{\edef\foo{\ifseenstar*\fi\ifseenplus+\fi#1}}
\def\resA{12}
\def\resB{*+12}
\test{12}  \ifx\foo\resA\else \bad\fi
\test*+{12} \ifx\foo\resB\else \bad\fi
\catcode`*=3 \catcode`+=3
\test *+{12} \ifx\foo\resB\else \bad\fi
\catcode`*=13 \catcode`+=13
\test *+{12} \ifx\foo\resB\else \bad\fi
}

$\overline {\tt P} P$

\def\looptest#1#2#3{\count0=#1\relax \loop #2\advance\count0by-1
  \ifnum\count0>0 #3\repeat}
\def\res{}
\looptest5{\edef\res{\the\count0 a\res}}{\edef\res{b\res}}
\testeq\res{1ab2ab3ab4ab5a}

%\tracingall
\let\olda\a %% \newcount defines \a globally...
\countdef\td 4 \td=0
\newif\ifprime \newif\ifunknown
\newcount\n \newcount\p \newcount\d \newcount\a
\def\primes#1{2,~3% assume that #1 is at least 3
  \n=#1 \advance\n by-2 % n more to go 
  \p=5 % odd primes starting with p
  \loop\ifnum\n>0 \printifprime\advance\p by2 \repeat}
\def\printp{, % we will invoke \printp if p is prime
  \ifnum\n=1 and~\fi % this precedes the last value
  \number\p \advance\n by -1 }
\def\printifprime{\testprimality \ifprime\printp\fi}
\def\testprimality{{\d=3 \global\primetrue
  \loop\trialdivision \ifunknown\advance\d by2 \repeat}}
\def\trialdivision{\a=\p \divide\a by\d
  \global\advance\td by 1 % okok
  \ifnum\a>\d \unknowntrue\else\unknownfalse\fi
  \multiply\a by\d
  \ifnum\a=\p \global\primefalse\unknownfalse\fi}


The first thirty prime numbers are \primes{30}. trial division macro was
expanded \the\td\ times. % Knuth says cpu time=4 sec, I see 0.02
\let\a\olda




\begin{tabular}{|ll|rr|cc|}
\hline a&b&c&d&e&f\\
aaa&bbb&ccc&ddd&eee&fff\\
\hline
A&\multicolumn{3}{l|}{BCD}&E&F\\
\multicolumn{2}{|l}{ab}&c&d&e&f\\
\cline{1-3}\cline{6-6}
aaa&bbb&ccc&ddd&eee&fff\\\hline
\end{tabular}




�  \char`\�  \char"E1  \'a  \a'a \xmllatex{\&\#xE1;}{}  \xmllatex{\&aacute;}{} 
$\hat{a} \acute{a}  \bar{a} \dot{a} \breve{a}
\check{a} \grave{a} \vec{a} \ddot{a} \tilde{a}
$ 
$$d=z^2=( I_t + \nabla I\vec{vvvv} +I div(\vec{vvvvv}) )^2,$$

$a\hbox{b=1} c$
$a\hbox{b~1}~c$ % both ~ do not give the same space. bad



{\LARGE
\textcelsius \textwon  \textnaira \textlira \textflorin
 \textdong \textperthousand   \textpertenthousand \textreferencemark
 \textdagger \\
\textdaggerdbl  \textgravedbl \textacutedbl \textasciigrave \textasciiacutex
\textlquill \textrquill \textohm \textmho \textnumero\\
 \textcircledP \textrecipe
  \textservicemark \textestimated \textbullet \textinterrobang
}




\texttrademark

IEME 1\ier 1\iere 1\iers 1\ieres 2\ieme 2\iemes \numero \Numero


      \begin{tabular}{|l|c|c|c|c|}
        \hline
        \multicolumn{1}{|l|}{} & \multicolumn{3}{c}{~~~~~~~~~~Propositional linear
          logic}& \\ \cline{2-4}
        \multicolumn{1}{|l|}{} & \multicolumn{3}{c|}{Rudimentary linear
          logic} &  \\ \cline{2-5}
        \multicolumn{1}{|l|}{} & Negation & Multiplicatives &
        Additives & Exponentials \\ \hline
        Negation & $A^\perp$ & & & \\ \hline
        Conjunction & & $A\otimes B$ & $A \& B$ & \\ \hline
        Disjunction & & $A\bindnasrepma B$ & $A \oplus B$ & \\ \hline
        Implication & & $A\multimap B$ &  & \\ \hline
        Constants & & $\mathbf{1}, \perp$ & $\top, \mathbf{0}$ & \\ \hline
        Modalities & & & & !A, ?A \\ \hline
      \end{tabular}



\newcounter{cms}
\setlength{\unitlength}{1mm}
\begin{picture}(50,39)
\put(0,7){\makebox(0,0)[bl]{cm}}
\multiput*(10,7)(10,0){5}{\addtocounter{cms}{1}\makebox(0,0)[b]{\arabic{cms}}}
\put(15,20){\circle{6}}
\put(30,20){\circle{6}}
\put(15,20){\circle*{2}}
\put(30,20){\circle*{2}}
\put(10,24){\framebox(25,8){car}}
\put(10,32){\vector(-2,1){10}}
\multiput(1,0)(1,0){49}{\line(0,1){2.5}}
\multiput(5,0)(10,0){5}{\line(0,1){3.5}}
\thicklines
\put(0,0){\line(1,0){50}}
\multiput(0,0)(10,0){6}{\line(0,1){5}}
\end{picture}
\url{http://www-sop.inria.fr/miaou/Jose&Grimm}
\setlength\unitlength{1pt}
\begin{picture}(120,100)(-5,0)
%\linethickness{.5pt}
 \qbezier [50](0,0)(0,100)(100,0)
 \qbezier     (0,0)(30,80)(100,0)
 \qbezier[150](0,0)(60,60)(100,0)
 \qbezier[200](0,0)(90,40)(100,0)
 \qbezier     (0,0)(120,20)(100,0)
% mark the end points
 \put(0,0){\circle*{3}}
 \put(100,0){\circle*{3}}
% mark the control points
 \multiput(0,100)(30,-20){5}{\circle{2}}
\end{picture}



\setlength{\unitlength}{1pt}
\sbox0{1A\bf b}
\savebox1{2A\bf b}
\savebox2(3,4){3A\bf b}
\savebox3(3,4)[c]{4A\bf b}
\savebox4[40pt]{5A\bf b}
\savebox{5}[40pt][c]{6A\bf b}
\newsavebox\Nsbox
\savebox\Nsbox[40pt][c]{7A\bf b}
\box0\box1\box2\box3\box4\box5\usebox\Nsbox


{\setlength\unitlength{10pt}
\begin{picture}(12,10)(-2,0)
%\linethickness{0.4pt}
\qbezier(2,6)(7,6)(9,3)
\qbezier(2,0)(7,0)(9,3)
\qbezier(2,6)(4,3)(2,0)
\qbezier(1,6)(3,3)(1,0)
\put(9.75,3.00){\circle{1.50}}
\put(10.5,3.00){\line(1,0){1.50}}
\put(0,5){\line(1,0){1.5}}
\put(0,1){\line(1,0){1.50}}
\end{picture}}
$x^{y^\epsilon_\varepsilon}$
\begin{equation}
\vec B(\vec r) = \frac{\mu_0}{4\pi}\left\{
\frac{3\vec m({\vec r}')\cdot(\vec r-{\vec r}')}{|\vec r-{\vec r}'|^5}
(\vec r-{\vec r}') - \frac{\vec m({\vec r}')}{|\vec r-{\vec r}'|^3}
\right\}. \label{W102}
\end{equation}
$\leftarrow\longleftarrow  \Leftarrow \Longleftarrow 
\rightarrow \longrightarrow \Rightarrow \Longrightarrow
\leftrightarrow \Leftrightarrow \longleftrightarrow\Longleftrightarrow
\mapsto \longmapsto \hookleftarrow \hookrightarrow
\leftharpoondown \leftharpoonup \rightharpoondown \rightharpoonup
\uparrow \Uparrow  \updownarrow \Updownarrow 
\downarrow \Downarrow \nearrow \searrow \swarrow \nwarrow
$

$\supseteq$

$\ldots \prime \exists \Diamond \top \bot \mho 
\cdots \forall \nabla \imath \flat \clubsuit \Re
\vdots \infty \surd \jmath \natural \diamondsuit \Im
\ddots \hbar \square \ell \sharp \heartsuit \angle
\aleph \emptyset \triangle \neg \wp \spadesuit \partial
$


$\hbar \ell \wp \Re \Im \partial \infty \emptyset \nabla
\surd \top \bottom \bot \angle \triangle$
%%  AFAIRE  smallint \land \lor  \Diamond


$\smallint \int \land \wedge \lor \owns \ni \notin
\neg \lnot \gets \leftarrow \to \rightarrow
\gtrless \geqslant \leqslant \iff \backslash$


$
\def\kernel{\frac{x^1_2}{y^3_4}}
\left\Vert\left|\left[\left(\left\{ \kernel \right\}\right)\right]\right|\right\Vert
\left\vert\left|\left<\left\langle \kernel \right\rangle\right>\right|\right\vert
\left\lbrace\left\lceil\left\lfloor \kernel \right\rfloor\right\rceil\right\rbrace
\lbrack\rbrack
a\mskip1mu b\mskip2mu c \mskip4mu d \mskip8mu e \mskip16mu f
$

$ 1\kern10pt b\mkern 18mu c \hspace{20pt} d\mskip36mu$


123
\begin{comment}
\begin{picture}(150,15)(0,0)
\thicklines
\dottedline{2}(0,10)(70,10)
\dottedline[$\bullet$]{3}(0,5)(70,5)
\dottedline[$\diamond$]{4}(0,0)(70,0)
\end{picture}
\begin{picture}(70,22)(0,-2)
\dashline{3}[0.7](0,18)(63,18)
\thicklines
\dashline{3}(0,13)(63,13)
\dashline[-30]{3}(0,8)(63,8)
\dashline[+15]{3}(0,4)(63,4)
\dashline[+30]{3}(0,0)(63,0)
\end{picture}
\end{comment}

\setlength\unitlength{1.5pt}
\begin{picture}(220,30)(0,0)
\put(0,0){\makebox(30,10){aaa}}
\put(40,0){\framebox(30,10)[lb]{bbb}}
\put(80,0){\dashbox{1}(30,10)[bl]{ccc}}
\put(120,0){\dashbox{2}(30,10)[rt]{ddd}}
\put(160,0){\dashbox{5}(30,10)[b]{eee}}
\put(192,1){\frame{W}}
\put(15,20){\oval(30,10)}
\put(55,20){\oval(30,10)[t]}
\put(95,20){\oval(30,10)}
\put(135,20){\oval(30,10)[bl]}
\put(175,20){\oval(30,10)[r]}
\end{picture}

\def\showcounter#1{%
\arabic{#1} \roman{#1} \Roman{#1} \alph{#1} \Alph{#1} \fnsymbol{#1}  $\fnsymbol{#1}$\\}
\newcounter{ctr}%\tracingall
\stepcounter{ctr}\showcounter{ctr}
\stepcounter{ctr}\showcounter{ctr}
\stepcounter{ctr}\showcounter{ctr}
\stepcounter{ctr}\showcounter{ctr}
\stepcounter{ctr}\showcounter{ctr}
\stepcounter{ctr}\showcounter{ctr}
\stepcounter{ctr}\showcounter{ctr}
\stepcounter{ctr}\showcounter{ctr}
\stepcounter{ctr}\showcounter{ctr}



\def\letterF{f}\def\letterO{o}
\def\double#1{#1#1}
\discardcounter{foo}
\newcounter{foo}\setcounter{foo}{10}
\the\value{\letterF\double\letterO}



\def\kernel{\frac{x^1_2}{y^3_4}}
\def\test{
\left\lmoustache\left\uparrow\left\Uparrow\left\updownarrow\left\lgroup
\kernel
\right\rgroup\right\Updownarrow\right\Downarrow\right\downarrow\right\rmoustache}
$\test$
$\let\left\relax\let\right\relax \test$
$\def\kernel{\int_0^{\frac{x^1_2}{y^3_4}}} \test$



{\tracingall \count206=3{
\catcode199=2
\mathcode 200=3
\lccode 201=4
\uccode 202=5
\sfcode 203=6
\delcode 204=7
\count 205=8
\global\count 206=9 \count 206=12
\muskip 207=10mu
\countdef\Xa=1 \Xa=2
\chardef \Xb=2
\mathchardef\Xc=3
\countdef\Xd=4 \Xd=5
\dimendef\Xe=5 \Xe=6pt \voffset=4pt
\skipdef\Xf=6 \Xf=7pt plus 8pt minus 9pt \lineskip=12pt \relax
\muskipdef\Xg=7 \Xg=10mu \relax \medmuskip=1mu 
\toksdef\Xh=8 \Xh={\relax} \output={ok}
}}


{\def\AAA{AAA}\ref{\char65 \AAA}\label{AAAA}}

{\def\par{!\endgraf}
Redefining par


}

\count0=0
{\def\par{\global\advance\count0by 1\endgraf}
Redefining par



\the\count0}


\begin{itemize}
\item[first item] This is the first item
\item[\it second item] This is the second item
\item[third] This is the last item
\begin{description}
\item[a] description of a.
\item[b] description of b.
\item[cde] description of cde.
\item[defgh etc] description of other letters
\begin{enumerate}
\item One
\item Two
\item Threee
\end{enumerate}
\end{description}
\end{itemize}

A:\mbox{aa}, B:\mbox{\it bb}, C:\mbox{\includegraphics{Logo-INRIA-couleur}}
D:\mbox{$a=y$}, E: $a=b\mbox{by definition}$. 

A:\makebox{aa} B:\makebox[1cm]{bb} C:\makebox[1cm][y]{cc}  C:\makebox[1cm][s]{cc}
D\makebox (1.2,4,5)[5]{text}
E\count3=4 \makebox ({1,2},\count3)[5]{text}
F\def\FF{1,2}\makebox(\FF,\count3)[5]{text}

\newcommand\Oarg[2][3]{\def\foo{#1=#2}}
\Oarg[5]{4}\testeq\foo{5=4}
\Oarg{4}\testeq\foo{3=4}
\renewcommand\Oarg[2][\@dblarg]{\def\foo{#1=#2}}
\Oarg[5]{4}\testeq\foo{5=4}
\Oarg{4}\testeq\foo{4=4}
\def\afoo#1[#2]#3{\def\foo{#1=#2=#3}}
\def\bfoo#1{\@dblarg{\afoo{#1}}}
\bfoo12 \testeq\foo{1=2=2}
\bfoo{12}{13} \testeq\foo{12=13=13}
\bfoo{12}[14]{13} \testeq\foo{12=14=13}

$\Longrightarrow$
\def\test{\ensuremath{\cal X_y}}
\test and $\test Z$
$\sin(x) + \cos^2(x) $ and $\overbrace{xyz} ^t $


$\bigl[\frac{1}{a^2}\bigr[$
$\left[\frac{1}{b^2}\right[$ 
$[\frac{1}{x^2}[$\\

$\int f\mathopen)a+b\mathclose(=c+d$
$\int f(a+b)=c+d$

$a\limits \displaylimits \nolimits b$


\def\cmd{a\dots b\ldots c\quad d\qquad e\space f\$g\%h\&i\!j\,k\{% 
l\}m\i n\sharp o\natural p\flat q\_}
wwwwwwwwwwww\cmd\\ $\cmd$


Math spacing\\
$xxxxx$\\
$x\,x\>x\;x\!x$\\
$a\,\,\,\,\,a\>\>\>\>\>a$\\
$a\;\;\;a\;\;\;\;a$\\
$!?\#\_.@\%\|/|$

% Testing footnote hack in \pers
\def\macA{some text}\def\macB{other text}
\tralics@fnhack\macA\macB
\testeq\macA{some text}\testeq\macB{other text}
\def\macA{some{\footnote{x}} text}
\tralics@fnhack\macA\macB
\testeq\macA{some{\footnote{x}} text}\testeq\macB{other text}
\def\macA{some\footnote{x} text}
\tralics@fnhack\macA\macB
\testeq\macA{some}\testeq\macB{{x} text, other text}
\def\macA{some\footnote{x} text}\def\macB{}
\tralics@fnhack\macA\macB
\testeq\macA{some}\testeq\macB{{x} text}



% This not test the new version
\begin{catperso}{List of Very Important Persons 1}
\let\pers\persA
\pers{Jean}[de la]{Fontaine},
\pers{Donald}{Knuth}[ author of \TeX]
\pers{Leslie}{Lamport } ,
\pers{Jose}{Grimm \footnote{No footnote here}}
\end{catperso}

\begin{catperso}{List of Very Important Persons 2}
\makeatletter
\renewcommand\@persB[7][]{\xbox{pers}{\xbox{firstname}{#2}\xbox{lastname}{#3}%
\xbox{localisation}{#1}\xbox{main-interest}{#4}\xbox{nationality}{#5}%
\xbox{info}{#6}\xbox{hdr}{#7}}\@addnl}
\pers{Jean}[de la]{Fontaine}{Tales}{French}[][phd]
\pers{Donald}{Knuth}{Math}{American}[ author of \TeX]\par
\pers{Leslie}{Lamport }{Computer Science}{American} 
\pers{Jose}{Grimm \footnote{No footnote here}}[Sophia]{Inria}{French}[funny text]
\end{catperso}


\begin{catperso}{List of Very Important Persons 3}
\pers{Jean}[de la]{Fontaine}{Other}{Other}[][phd],
\pers{Donald}{Knuth}{Other}{Other}[ author of \TeX]
\pers{Leslie}{Lamport }{Other}{Other}
\end{catperso}

\begin{catperso}{List of Very Important Persons 4}
\makeatletter
\def\@persA#1#2#3{\xbox{pers}{\xbox{firstname}{#1}\xbox{lastname}{#2}\xbox{info}{#3}}}
\let\pers\persA
\pers{Jean}[de la]{Fontaine}
\pers{Donald}{Knuth}[author  of \TeX]
\pers{   Leslie   }{Lamport}
\end{catperso}


\pers {Jean B}{Dupond}[Lyon]{these}{Dec}[Ok][not yet]
\pers {Jean A}{Dupond}[Sophia]{Chercheur}{CNRS}[Ok][not yet]
\pers {Jean B}{Dupond}[+Lyon]{Ingenieur}{INRIA}[Ok][not yet]
\pers {Jean C}{Dupond}[=Paris]{Other}{IBM}[Ok]
\pers {Jean D}{Dupond}{Enseignant}{Other}[Ok][not yet]

\def\declaretopic#1#2{\xbox{topic}{\XMLaddatt{num}{#1}\xbox{t\_titre}{#2}}\@addnl}

%% test de parametrisation
\begin{itemize} \item [A] B \end{itemize}
\begin{glossaire} \glo{x}{y}\end{glossaire}
\part{A}\chapter{B}\section{C} \subsection{D}
\subsubsection{E}\paragraph{F} \subparagraph{G}
\begin{motscle}first,second \end{motscle}
\mbox{$a$}$b$ ${c}$ $0$ xxxxxxxxxxxxxxxxxxxxxxxxxxxx
\declaretopic{10}{foo}\caption[opt]{ok}
\footnote{anote\label{from-one}} 
\footnote{par1\par pr2}
\footnote[ign]{last\ref{from-one}}
\includegraphics{x.ps}
\begin{figure}\includegraphics{y.ps} \caption{A\par B}\end{figure}
\fbox{\includegraphics{../../tralics/Test/a_b.eps}}
\includegraphics[scale=0.3]{x.ps}
\scalebox{0.3}{\includegraphics[scale=0.6]{x.ps}}
\scalebox{0.3}{\includegraphics{x.ps}}
\scalebox{0.3}[0.4]{\includegraphics{x.ps}}



\def\sample{This is some text explaining and showing how text is centered,
flushed left or flushed right or indented or not indented. \Env}
\def\Line{\sample\noindent\sample\indent\sample}
\def\Env{Mode is normal}\Line
\begin{center}\def\Env{Mode is center}\Line\end{center}
\begin{flushleft}\def\Env{Mode is left}\Line\end{flushleft}
\begin{flushright}\def\Env{Mode is right}\Line\end{flushright}



{
\openin 5=tortureaux
\endlinechar=-1
\ifeof5 \badifeofatentry\fi
\read 5 to \foo\testeq\foo{abc}
\read 5 to \foo\testeq\foo{}
\read 5 to \foo\testeq\foo{\a\b{\c{} \d} \e}
\global\read 5 to \foo
\closein5\relax
\ifeof5\else\badifeofatexit\fi
}\testeq\foo{123}
\ifeof3\else \badifeofnonexists\fi
\begin{tabular}{cc}
a&b&c\\
\multicolumn{14}{l}{s}&e
\end{tabular}

{\makeatletter
X\count@=2 \count@=+2 \count@= + + 2 \count@ = + + + 2
\count@=-2 \count@=+-2 \count@= -+ + 2 \count@ = + - + + 2 \count@=---2 X

\dimen0=.25pt \count@=-\dimen0
\skip0=-.125pt plus 4fill \count@=+-+\skip0
\count@=-\count@
\count@=-\count255
\count@=\catcode`\A
\count@=\inputlineno
\count@=\escapechar
\count@=\year0, \count@\the\year0

2294X\count@=1000 \count@='1750 \count@="3E8
\count@=`A \count@=`\A \count@ =`\~ X

\count@=3 \count@=222\ifodd\count@ 1 \else 2\fi

}
{\makeatletter% \tracingall
2298X\dimen@=2pt \dimen@=+2pt \dimen@= + + 2pt \dimen@ = + + + 2pt
\dimen@=-2pt \dimen@=+-2 Pt \dimen@= -+ + 2pT \dimen@ = + - + + 2PT \dimen@=---2 pt X
\skip0=2ptplus2fill \parskip=2pt minus 3pt
\dimen@=\skip0
\dimen@=\parskip

\dimen@=-\dimen@ 
\dimen@=-\dimen0
\dimen@=\hsize
\dimen@=\fontdimen0 \font

\dimen@=.10pt \dimen@=10.pt \dimen@=1.2pt \dimen@=.pt
\dimen@=,10pt \dimen@=10,pt \dimen@=1,2pt \dimen@=,pt

\dimen@=2pt \count@=5 \dimen@=\count@ pt \dimen@=\year pt 
\dimen@=\the\year,\the\month pt

\dimen@=2pt \count@=196608 \skip0=4pt plus 3fill
\dimen@=2 \dimen@ \dimen@=2 \count@ \dimen@=2 \skip0
\dimen@=.2 \dimen@ \dimen@=.2 \count@ \dimen@= .2 \skip0

X\dimen@=2truept \dimen@=2 TRUE PT \dimen@=2.0 True PtX

\dimen@=2ex \dimen@=4em
\ifdim \ifnum0=1 \count0 pt\else \skip 0\fi >1cm \errmessage{bad}\fi
\skip0 =1\skip0
\dimen@ = 12c\ifdim \the\ifnum0=1 \count0 pt\else \skip 0\fi <1cmm\else c\fi
\dimen@="Ccc \dimen@=-,sp
 X

\skip0=2pt
\skip0=-\skip0
\skip0=\parskip

2340X\skip0 = 1pt plus 2 pt minus 3pt 
\skip0 = 1pt PLUS -2 pt MINUS -3pt 
\skip0 = 1pt
\skip0=.33\skip0 plus \count0fill minus \parskip
\skip0 = 1pt minus 2pt
\skip0 = 1ptplus 3Filll minus 4fil X\xmlemptyelt{foo}\xmlelt{foo}{bar}
}

{\makeatletter \@namedef{1.0pt}{2}
\dimen0=1pt 
\ifnum123=1\csname\the\dimen0\endcsname3\relax \else\errmessage{bad}\fi
}
\par\xmlelt{foo}{barX}
\par\xbox{foo}{barY}
2340\lsc{Abc}\fsc{Abc}{\sc Abc}

{
\hrule \vrule
\hrule width 1pt depth 2pt height  3pt
\vrule width 1pt depth 2pt height  3pt
}

\begin{motscle}
latex, xml, translator
\end{motscle}


ligne2060
\begin{tabular*}{\textwidth}[i]{cc}
a&b&c\\
\multicolumn{14}{l}{s}&e
\end{tabular*}

\newcounter{ca}\newcounter{cb}
\newcommand{\printgcd}[2]{%
  \setcounter{ca}{#1}\setcounter{cb}{#2}%
  Gcd(#1,#2) = 
  \whiledo{\not\(\value{ca}=\value{cb}\)}%
    {\ifthenelse{\value{ca}>\value{cb}}%
      {\addtocounter{ca}{-\value{cb}}}%
      {\addtocounter{cb}{-\value{ca}}}%
    gcd(\arabic{ca},\arabic{cb}) = }%
  \arabic{ca}.}

\makeatletter
\newif\if@whiledo
\long\def\whiledo#1#2{%
  \ifthenelse{#1}%
    {\@whiledotrue
     \@whilesw\if@whiledo\fi
       {#2%
        \ifthenelse{#1}\@whiledotrue\@whiledofalse}}%
    {}%
  }
\printgcd{144}{272}
\let\printgcd\relax

\newif\iftest
\newcommand{\printgcd}[2]{%
  \setcounter{ca}{#1}\setcounter{cb}{#2}%
  \testtrue
  \@whilesw\iftest gcd(\arabic{ca},\arabic{cb}) = \fi{%
   \ifnum\value{ca}=\value{cb}\testfalse\else
    \ifnum\value{ca}>\value{cb}%
      \addtocounter{ca}{-\value{cb}}\else
      \addtocounter{cb}{-\value{ca}}\fi
\fi}%
  \arabic{ca}.}
\printgcd{144}{272}
\newcounter{cc}\newcounter{cd}
\def\assigncounter{%
 \ifnum\value{ca}<\value{cb}%
  \setcounter{cc}{\value{ca}}%
  \setcounter{cd}{\value{cb}}%
 \else 
  \setcounter{cc}{\value{cb}}%
  \setcounter{cd}{\value{ca}}%
  \fi}

\def\subtractcounter{%
 \setcounter{cb}{\value{cd}}%
 \setcounter{ca}{\value{cc}}%
 \addtocounter{cb}{-\value{cc}}}%
\let\printgcd\relax
\newcommand{\printgcd}[2]{%
  \setcounter{ca}{#1}\setcounter{cb}{#2}%
  \assigncounter
  \@whilenum \value{cc}<\value{cd} Gcd(\arabic{cc},\arabic{cd}) = \do{%
     \subtractcounter\assigncounter}%
  \arabic{ca}.}
\printgcd{144}{272}


\newlength\cC\newlength\cD
\def\assignlength{%
 \ifnum\value{ca}<\value{cb}%
  \setlength\cC{\value{ca}sp}%
  \setlength\cD{\value{cb}sp}%
 \else 
  \setlength\cC{\value{cb}sp}%
  \setlength\cD{\value{ca}sp}%
  \fi}

\def\subtractlength{%
 \setcounter{cb}{\cD}%
 \setcounter{ca}{\cC}%
 \addtocounter{cb}{-\cC}}%
\let\printgcd\relax
\newcommand{\printgcd}[2]{%
  \setcounter{ca}{#1}\setcounter{cb}{#2}%
  \assignlength
  \@whiledim \cC<\cD Xgcd(\arabic{ca},\arabic{cb}) = \do{%
     \subtractlength\assignlength}%
  \arabic{ca}.}
\printgcd{144}{272}

\makeatletter 
\ifnum\z@>0 \bad{positive zero}\fi
\ifdim\z@<0pt \bad{positive dim zero}\fi
\ifnum\z@<0 \bad{negative zero}\fi
\ifdim\z@<0pt \bad{negative dim zero}\fi
\makeatother

\setcounter{cb}{3}
\setcounter{ca}{4}
\edef\foo{\the\value{cb}}\testeq\foo{3}
\edef\foo{\the\value{ca}}\testeq\foo{4}
\numberwithin{cb}{ca}
\edef\foo{\the\value{cb}+\thecb}\testeq\foo{3+4.3}\tracingall
\stepcounter{ca}
\edef\foo{\the\value{cb}+\thecb}\testeq\foo{0+5.0}
\setcounter{cb}{4}
\edef\foo{\the\value{cb}+\thecb}\testeq\foo{4+5.4}
\numberwithin[\alph]{cb}{ca}
\edef\foo{\the\value{cb}+\thecb}\testeq\foo{4+5.d}

\newcounter{AA}
\setcounter{AA}{0}
\stepcounter{AA}\fnsymbol{AA}
\stepcounter{AA}\fnsymbol{AA}
\stepcounter{AA}\fnsymbol{AA}
\stepcounter{AA}\fnsymbol{AA}
\stepcounter{AA}\fnsymbol{AA}
\stepcounter{AA}\fnsymbol{AA}
\stepcounter{AA}\fnsymbol{AA}
\stepcounter{AA}\fnsymbol{AA}
\stepcounter{AA}\fnsymbol{AA}


{EOL\endlinechar=`\A \relax% 
\def\foo{a
b}\testeq\foo{aAb}%
\def\foo{\foo
b}\testeq\foo{\fooA b}%
\def\foo{a
b}\testeq\foo{aAb}%
\endlinechar=`\% %
\def\foo{a
b}\testeq\foo{ab}
\endlinechar=-1 %
\def\foo{a
b}\testeq\foo{ab}
y}


\def\empty{}
\def\lengthof#1{\def\lengthres{0}\ifx#1\empty\else\count0=1 \lengthaux #1\relax\edef\lengthres{\the\count0 }\fi}
\def\lengthaux#1#2\relax{\ifx#2\empty\else\advance\count0 by1 \lengthaux #2\relax\fi}

Foo\linebreak\linebreak[2]\nolinebreak\nolinebreak[2]Bar
Foo\pagebreak\pagebreak[2]\nopagebreak\nopagebreak[2]Bar


\let\foo\par
\renewcommand*\foo[2][toto]{#1#2}
\xdef\Bar{{\meaning\foo}}
\expandafter\lengthof\Bar\testeq\lengthres{21}

%macro:(latex mac with optional argument toto)#2->#1#2
\lengthof{abc}\testeq\lengthres 3
\lengthof{}\testeq\lengthres 0

\makeatletter
\def\strip@prefix#1>{}
\def\foo#1#2{OK OK}
\edef\Bar{\expandafter\strip@prefix\meaning\foo}
{\catcode `\O=12 \catcode `\K=12 \testeq\Bar{OK OK}
}
\count0=3
\skip1=10pt
\def\foo{\count} \def\xbar{\skip} \def\zero{0}\def\one{1}
\parindent=\foo\zero\xbar\one
%\showthe\parindent

\edef\toto{x\empty\empty\empty y}\testeq\toto{xy}
\def\zfoo#1\bar#2#{\def\X{{#1}{#2}}}
\zfoo 1\bar2{3} \xshow\X 
\zfoo1\bar {2}{3}\xshow\X
\let\foo#\def\A\foo1\foo2{\foo1} \let\foo\relax

%debug \AtBeginDocument{}

%\nocite{tor1,tor2,tor3,tor4,tor5,tor6,tor7,tor8,tor9,tor10,tor11}
%\nocite{tor12,tor13,tor14,tor15,tor16,tor17,tor18,tor19,tor20,tor21}
%\nocite{tor22,tor23,tor24,tor25,tor26,tor27}
%\nocite{tor31,tor32,tor33,tor34,tor53,tor52}
%\nocite{tor40,tor41,tor42,tor43,tor44,toto,titi}
\nocite{toto,titi}


\begingroup %\tracingall
\newenvironment{AAA}{1\begin{BBB}5}{8\end{BBB}c}
\newenvironment{BBB}{2\begin{CCC}4}{9\end{CCC}b}
\newenvironment{CCC}{3}{a}
\newenvironment{DDD}[2]{#2#1\def\foo{#1}}{\foo}
\newenvironment{EEE}[2][e]{#2#1\def\foo{#1}}{\foo}
\begin{AAA}67\end{AAA} 123456789abc
\begin{DDD}678\end{DDD} 7686
\begin{EEE}[6]78\end{EEE} 7686
\begin{EEE}78\end{EEE} 7e8e
\renewenvironment{EEE}[2][e]{#1#2\def\foo{#1}}{\foo}
\endgroup

\iffalse test de \renew... qui fait des erreurs
\renewenvironment{xxx}{before}{after}
\let\xx\renewenvironment
\xx{xxx}{before}{after}
\newenvironment{xxx}{before}{after}
\newenvironment{xxx}{before}{after}
\xx{xxx}{before}{after}
\fi

{\def\foo{} \providecommand\foo[2]{#1#2} \xshow\foo\foo}


\bibliography{torture,bo,bo}
\def\foo#1#{#1} \xshow\foo
\renewcommand\foo[2][toto]{#1#2} \xshow\foo
\let\foo\par \xshow\foo
\let\foo=1 \xshow\foo
\let\foo=\undef \xshow\foo
\count0=0
\loop \csname command number \the\count0\endcsname
  \advance\count0 by 1 
\iffalse\repeat


\let\mytypeout\typeout
\ifnum \tracingcommands=0 \def\mytypeout{\write -1 }\fi\tracingall
\input taux1.tex\input taux2.tex
% aux file  for testing tralics
% this file contains nothing useful
\mytypeout{in file taux1.tex}
\input taux2
% the file should finish with a comment
%% LaTeX2e file `taux2.tex' utf8-encoded
%% generated by the `filecontents' environment
%% from source `../Test/torture' on 2008/02/10.
%%
% aux file  for testing tralics
% this file contains nothing useful
\mytypeout{in file taux2.tex}
\endinput
The file should finish with a \endinput, but not on the last line.

\def\foo{\input taux1 \input taux2 }
\foo

\paragraph{a} aa\label{pax}
\paragraph{b} bb\label{pbx}

\makeatletter
\fancyhf{} % clear all
\let\XF\fancyinternal
\def\fancyinternal#1#2{\XF{\uppercase{#1}}{#2}}

\let\IP\inert@thepage
\def\inert@thepage{\IP\AddAttToLast{style}{arabic}}

\fancyhead[el,or]{righmark}
\fancyhead[er,ol]{leftmark}
\lhead[\thepage]{} 
\rhead[\emph{OK}]{\thepage}

\def\inert@thepage{\IP\AddAttToLast{style}{roman}}
\lfoot{\thepage}
\rfoot[\emph{OK}]{\thepage}
ok

\makeatother
\newcolumntype{L} {>{\large\bfseries 2}l <{y}|}
\newcolumntype{+} {>{B}l <{D}|}
\newcolumntype{_}{rlc<{x}}
\newcolumntype{x}{>{b}c<{a}}
\begin{tabular*}{10pc}[b]{lrc}
\hline
a&b&c\\[2pt]
\multicolumn{1}{l}{A}&B&C\\\hline
\end{tabular*}



\begin{tabular*}{5cm}[b]{*{4}{_c|}}
a1&a2&a3&a4  & b1&b2&b3&b4 & c1&c2&c3&c4& d1&d2&d3&d4\\
Wa1&Wa2&Wa3&Wa4  & Wb1&Wb2&Wb3&Wb4 & Wc1&Wc2&Wc3&Wc4& Wd1&Wd2&Wd3&Wd4\\
\end{tabular*}
\declaretopic{A}{B}

okok\par

%\end{xmlelement*}



\def\pararg#1#2{\begin{xmlelement+}{}#1\par #2\end{xmlelement+}}
\includegraphics[angle=90]{x}
\rotatebox{30}{\includegraphics{x}}
\rotatebox{45}{\it x}
\rotatebox{-10}{\pararg{xx}{yy}}
\fbox{\includegraphics{x}}
\fbox{\tt abc}
\fbox{\pararg{abc}{def}}
\fbox{ab}
\framebox{\includegraphics{x}}
\framebox{\tt abc}
\framebox{\pararg{abc}{def}}
\dimen0=20pt
\framebox[2\dimen0]{\includegraphics{x}}
\framebox[3\dimen0]{\tt abc}
\framebox[4\dimen0]{\pararg{abc}{def}}
\framebox[5\dimen0][l]{\includegraphics{x}}
\framebox[6\dimen0][r]{\tt abc}
\framebox[7\dimen0][s]{\pararg{abc}{def}}
\scalebox{xx}{\includegraphics{x}}
\scalebox{yy}{\tt abc}
\scalebox{yy}[zz]{\tt abc}
\scalebox{zz}{\pararg{abc}{def}}


\includegraphics[angle=90]{x}
\rotatebox{30}{\includegraphics{x}}
\rotatebox{40}{x}


{\centering a \indent a1 \noindent a2 \\ a3\par a4} \par
{\raggedleft b \indent b1 \noindent b2 \\ b3\par b4} \par
{\raggedright c \indent c1 \noindent c2 \\ c3\par c4} \par
{\centering\raggedright d \indent d1 \noindent d2 \\ d3\par d4} \par
{f\centering\raggedright e \indent e1 \noindent e2 \\ e3\par e4} \par
\begin{center}A \indent A1 \noindent A2 \\ A3 \par A4\end{center}
\begin{flushright}B \indent B1 \noindent B2 \\ B3 \par B4\end{flushright}
\begin{flushleft}C \indent C1 \noindent C2 \\ C3\par C4\end{flushleft}
\begin{center}\begin{flushleft}D \indent D1 \noindent D2 \\ D3\par D4\end{flushleft}\end{center}
F\begin{center}\begin{flushleft} E \indent E1 \noindent E2 \\ E3 \par E4\end{flushleft}\end{center}
\begin{quote}I \indent I1 \noindent I2 \\ I3 \par I4\end{quote}
\begin{quotation}J \indent J1 \noindent J2 \\ J3 \par J4\end{quotation}

\begin{tabular*}{5cm}{>{\bgroup\bf}c<{\egroup}c}
aaaaa}aaa{c&b\\c&d
\end{tabular*}

\begin{tabular}{|ll|rr|cc|}
\hline a&b&c&d&e&f\\
aaa&bbb&ccc&ddd&eee&fff\\
\hline
A&\multicolumn{3}{+}{C}&E&F\\
\multicolumn{2}{|l}{ab}&c&d&e&f\\
\cline{1-3}\cline{6-6}
aaa&bbb&ccc&ddd  &eee&fff\\\hline
\end{tabular}
%cline tests
\begin{tabular}{}a \\ \cline{2-2} \end{tabular}
\begin{tabular}{}b \\ \cline{4-7} \end{tabular}
\begin{tabular}{}c \\ \cline{1-7} \end{tabular}
\begin{tabular}{}\multicolumn{4}{}{d} \\ \cline{1-1} \end{tabular}
\begin{tabular}{}\multicolumn{4}{}{e} \\ \cline{1-1} \cline{4-4} \end{tabular}
\begin{tabular}{} \cline{1-1} \cline{4-4} f\end{tabular}
\begin{tabular}{} \cline{4-4} g\end{tabular}


\begin{tabular}{|ll|rr|cc|}
A&\multicolumn{3}{l}{BCD}&E&F\\\cline{1-3}\cline{6-6} 
A&\multicolumn{3}{l}{BCD}&E&F\\\cline{1-3}\\ \cline{6-6}  
z&t\\ \cline{3-3}
\end{tabular}

 
\begin{tabular} {| >{\large 1}c <{x}| L > {\large\itshape 3}x <{z}|}
\hline A&B&C\\\hline 100&10 &1
\end{tabular}

test de table
\newcolumntype{C}{>{$}c<{$}}
\newcolumntype{L}{>{$}l<{$}}
\newcolumntype{R}{>{$}r<{$}}
\newcolumntype{d}[1]{>{\rightdots{#1}}r<{\endrightdots}}
\newcolumntype{X}{CLR}
\begin{tabular}{*{3}{|c|}d{23}X}
\end{tabular}




\let\IC=\includegraphics
\def\FILE{Logo-INRIA-couleur}
\IC[angle=0,width=3cm,clip=]{\FILE}
\IC[angle=20,width=.5\textwidth,height=.3\textheight]{\FILE.ducon}
\IC[angle=0,width=\columnwidth,height=\textheight]{\FILE.foo_bar}
{\language=1 a:c
\IC[angle=0, =foo,,width=3cm,scale=1,scale=2,clip]{../../a_b:c}
}
\framebox{\includegraphics{x_.ps}}

{
  \setlength\epsfxsize{50pt}
  \setlength\epsfysize{60pt}
  \epsfbox{x.ps}
  \setlength\epsfysize{70pt}
  \epsfbox{x.eps}
  \epsfbox{x.epsf}
}


\fbox{\includegraphics{x.ps}}
\includegraphics[scale=0.3]{x.ps}
\scalebox{0.3}{\includegraphics[scale=0.6]{x.ps}}
\scalebox{0.3}{\includegraphics{x.ps}}
\includegraphics[angle=90]{x.ps}
\rotatebox{30}{\includegraphics{x.ps}}

\theorembodyfont{\sl}
\theoremstyle{break}
\newtheorem{Cor}{Corollary}
\theoremstyle{plain}
%\newcounter{section}
\setcounter{section}{17}
\newtheorem{Exa}{Example}[section]
{\theorembodyfont{\rmfamily}\newtheorem{Rem}{Remark}}
\theoremstyle{marginbreak}
\newtheorem{Lem}[Cor]{lemma}
\theoremstyle{change}
\theorembodyfont{\small\itshape} \newtheorem{Def}[Cor]{Definition}
\theoremheaderfont{\scshape}
\def\Lenv#1{\texttt{#1}}

\begin{Cor}
 This is a sentence typeset in the theorem environment \Lenv{Cor}.
\end{Cor}
\begin{Exa}
 This is a sentence typeset in the theorem environment \Lenv{Exa}.
\end{Exa}
\begin{Rem}
 This is a sentence typeset in the theorem environment \Lenv{Rem}.
\end{Rem}
\begin{Lem}[Ben User]
 This is a sentence typeset in the theorem environment \Lenv{Lem}.
\end{Lem}
\begin{Def}[Very Impressive definition]
 This is a sentence typeset in the theorem environment \Lenv{Def}.
\end{Def}



\[
\mathop {\mathop {arg\,min}\nolimits_\Psi }\limits_{\theta \in \Theta } 
\Delta \left( {\mathop \Phi \nolimits_\theta \left( {\Omega _s } 
\right)-\Omega _t } \right)
\qquad
x_{y_z} = {x_y}_z= 
x^{y_z} = {x^y}_z= 
x_{y^z} = {x_y}^z= 
x^{y^z} = {x^y}^z
%
\left\{\{\frac{2}{3}\}\right\}
\]



\newcommand{\RAFsixE}{
  \scaleput(1.25,1.25){\arc(0,-1.25){-135}}
  \scaleput(0,0){\curve(0.366,2.133, 1.25,3.19, 2.5,4.42,
    5.0,6.10, 7.5,7.24, 10,8.09, 15,9.28, 20,9.90, 30,10.3,
    40,10.22, 50,9.80, 60,8.98, 70,7.70, 80,5.91, 90,3.79,
    95,2.58, 99.24,1.52)}
  \scaleput(99.24,0.76){\arc(0,-0.76){180}}
  \scaleput(0,0){\curve(1.25,0, 99.24,0)}
  }
\setlength{\unitlength}{.5mm}
\linethickness{0.7mm}
\begin{center}
text\\
  \begin{picture}(100,20)
    \RAFsixE
  \end{picture}
  \\The RAF 6E has a flat undersurface.
\\
  \begin{picture}(120,30)(-20,0)
  \renewcommand{\xscale}{0.9781}
  \renewcommand{\xscaley}{0.2079}
  \renewcommand{\yscale}{0.9781}
  \renewcommand{\yscalex}{-0.2079}
  \put(0,20){\RAFsixE}
  \thicklines
  \put(-20,5){\vector(1,0){20}}
  \end{picture}
\\
  The RAF 6E has maximum lift at angles of attack over 12$^\circ$.
  \end{center}


\setlength{\unitlength}{0.4pt}
\linethickness{0.7mm}
\begin{picture}(400,110)(-10,0)
  \tagcurve(80,0, 0,0, 40,100, 80,0, 0,0)
  \closecurve(150,0, 190,100, 230,0)
  \curve(300,0, 340,100, 380,0)
\end{picture}



\setlength{\unitlength}{1.8mm}
\begin{picture}(40,30)
  \thicklines
  \multiput(20,5)(20,12){2} {\line(0,-1){2}\line(-5,3){20}}
  \multiput(20,5)(-20,12){2} {\line(5,3){20}}
  \put(20,3){\line(5,3){20}}
  \put(20,3){\line(-5,3){20}}
  \put(0,15){\line(0,1){2}}
  \linethickness{1pt}
  \put(20,5) {
   \renewcommand{\xscale}{1}
   \renewcommand{\xscaley}{-1}
   \renewcommand{\yscale}{0.6}
   \renewcommand{\yscalex}{0.6}
   \scaleput(10,10){\bigcircle{10}}
   \put(0,-2){%
      \scaleput(10,10){\arc(5,0){121}}
      \scaleput(10,10){\arc(5,0){-31}}}}
\end{picture}


  \begin{tabular}{|l||c|c|c|}
    \hline
    Protocol & Proof & 
    \begin{tabular}{c}max \# of \\ ``back'' steps \end{tabular} 
   & \begin{tabular}{c}Time \\ ms.\end{tabular} \\
    \hline 
    Otway-Rees             & Yes & 0 & 0.35\\
  \end{tabular}

% this should yield \relax
\expandafter\let\expandafter\relax\ifnum0=0\fi
\edef\foo{\jobname} \xshow\foo
\edef\foo{\noexpand\jobname=\jobname} \xshow\foo
112233\noexpand\par\noexpand\jobname=\jobname

{\tracingall
\count0=7 
\count1=7
\count0=2\ifnum\count0=\count1 3\fi4
}
{\tracingall
\count0=7 
\count13=0
\count0=2\ifnum\count0=\count13\fi4
}
{\tracingall
\def\relax{25}
\count0=7 
\count13=7
\count0=2\ifnum\count0=\count13\fi4
}

{\def\relax{0}\edef\foo{\ifnum0=0\fi}}

%%%%%%%%% TESTS
{
\if00 \else \BUG\fi
\if01 \BUG\fi
\if\par\relax \else \BUG\fi
\if0\par \BUG\fi
\count0=1000
\if\romannumeral\count0m \else \BUG\fi 
\if\romannumeral\count0n \BUG\fi
\if\number\count0  \BUG\fi
\count0=1100
\if\number\count0 \else \BUG\fi
\catcode `[=13 \catcode`]=13 \def[{*}
\if\noexpand[\noexpand] \BUG \fi
\if[* \else \BUG \fi
\if\noexpand[* \BUG\fi
\def\foo{01}\def\xfoo{00}
\if\xfoo \else\BUG\fi
\if\foo\par \BUG\fi 
\if\noexpand\foo\par \else\BUG\fi 
\if\par\noexpand\foo \else\BUG\fi 
\if0\noexpand\foo \BUG\fi \if\par\else \BUG\fi \if\else \BUG\fi \if\fi
}
\begingroup
\catcode `^^^^0123=13 \def^^^^0123{$} % $ emacs
\def\foo{\uccode`\~=`A\uppercase{~}}
\def\xfoo{\uccode`\~=`A\uppercase{\gdef\xxfoo{~}}}

{\catcode`A 13\gdef A{B}}

\foo % typesets B
\xfoo% defines\xxfoo
\if\xxfoo B\else\bad\fi
\ifcat\noexpand~1\bad\fi
\ifcat^^^^0123$\else\bad\fi
\ifcat^^^^012344\bad\fi
\ifcat^^^^01234a\bad\fi
\ifcat\noexpand^^^^0123$\bad\fi % $ emacs
\ifcat\noexpand^^^^01234\bad\fi
\catcode`[=13 \catcode`]=13 \def[{*}%
 \ifcat\noexpand[\noexpand]\else\bad\fi
 \ifcat[*\else\bad\fi
 \ifcat\noexpand[*\bad\fi
 \def\test{%
   \ifcat\noexpand[\noexpand]\else\bad\fi
   \ifcat[*\else\bad\fi
   \ifcat\noexpand[*\bad\fi}%
 \test
 \let\b*\def\a{*}\def\c{/}%
 \if*\a\else\bad\fi
 \if\a\b\else\bad\fi
 \if\a\c\bad\fi
 \if\a\par\bad\fi
 \if\par\let\else\bad\fi
 \if\noexpand\cpar\let\else\bad\fi
\def\foo{W}\let\xfoo=W\catcode`W=12 \def\yfoo{W}%
\catcode`W=13 \if\noexpandW\foo\else \xbad\fi 
\if\noexpandW\xfoo\else \xbad\fi  
\if\foo\xfoo\else \xbad\fi  \ifcat\foo\xfoo\else \xbad\fi   
\if\yfoo\xfoo\else \xbad\fi  \ifcat\yfoo\xfoo \xbad\fi  \ifcat\yfoo=\else \xbad\fi   
x

\endgroup



{
\ifcat a0 \BUG\fi
\ifcat $^ \BUG\fi %$
\ifcat 01 \else \BUG\fi
\ifcat AB \else \BUG\fi
\count0=1000
\ifcat\romannumeral\count0m \BUG\fi 
\count0=1100
\ifcat\romannumeral\count0m \else \BUG\fi 
\count0=1
\ifcat\number\count0 3\else \BUG\fi
\ifcat\par\relax \else \BUG\fi
\ifcat0\par \BUG\fi
\catcode `[=13 \catcode`]=13 \def[{*}
\ifcat\noexpand[\noexpand] \else \BUG \fi
\ifcat[* \else \BUG \fi
\ifcat\noexpand[* \BUG\fi
}
{ \tracingall
\ifnum 0<1 \else\BUG\fi
\ifnum 0=1 \BUG\fi
\ifnum 0>1 \BUG\fi
\ifnum -3<-2 \else \BUG\fi
\ifnum -3=-2 \BUG\fi
\ifnum -3>-2  \BUG\fi
\def\equalsign{=}
\ifnum 1\ifnum2<3 4\else6\fi\equalsign 14 \else \BUG \fi
\ifnum 1\ifnum2>3 4\else6 \fi\equalsign 14 \BUG \fi
\ifnum 1\ifnum2>3 4\else6\fi\equalsign 14 \BUG \fi
}
{ \tracingall
\ifdim 0pt<1pt \else\BUG\fi
\ifdim 0pt=1pt \BUG\fi
\ifdim 0pt>1pt \BUG\fi
\ifdim -3pt<-2pt \else \BUG\fi
\ifdim -3pt=-2pt \BUG\fi
\ifdim -3pt>-2pt  \BUG\fi
\def\equalsign{=}
\ifdim 1\ifnum2<3 4\else6\fi pt\equalsign 14pt \else \BUG \fi
\ifdim 1\ifnum2>3 4\else6\fi pt\equalsign 14pt \BUG \fi

\ifodd 1 \else \BUG\fi
\ifodd -1 \else \BUG\fi
\ifodd 0 \BUG\fi
\ifodd 2 \BUG\fi
\ifodd -2 \BUG\fi
\ifodd 1\ifnum2<3 5 \else6 \fi \else \BUG \fi
\ifodd 1\ifnum2>3 5 \else6 \fi \BUG\fi
\ifnum4\ifodd1 2 \else 3\fi=42\else \BUG\fi
}

{
\ifvmode 1 \else 2 \fi \ifinner 3\fi
\vbox{\ifvmode 1 \else 2 \fi \ifinner 3\fi}
\ifhmode 1 \else 2 \fi \ifinner 3\fi
\hbox{\ifhmode 1 \else 2 \fi \ifinner 3\fi}
\ifmmode 1 \else 2 \fi $\ifmmode \ifinner 1 \fi \else 2 \fi$
$$\ifmmode 1 \else 2 \fi\ifinner 3\fi$$

{\makeatletter \ifvoid\voidb@x \else \bad \fi}
\ifvoid0 \else \bad \fi
\setbox0=\hbox{x} 
\ifvoid0 \bad \fi
\setbox0=\xbox{foo}{bar} 
\ifvoid0 \bad \fi
\setbox0=\hbox{} 
\ifvoid0 \else \bad \fi
\setbox0=\xbox{}{} 
\ifvoid0 \else \bad \fi
\setbox0=\xbox{foo}{} 
\ifvoid0 \bad \fi \setbox1\box0
\ifvoid0 \else \bad \fi
}
{
\chardef\xx=48 \chardef\yy=`0
\ifx 01 \BUG \fi
\ifx aa \else \BUG \fi
\ifx {} \BUG \fi
\ifx\xx\yy \else \BUG \fi
\countdef\xx17 \countdef\yy 17
\ifx\xx\yy \else \BUG \fi
\ifx\par\relax \BUG\fi \if\par\relax \else \BUG\fi
\let\endgraf\par 
\ifx\endgraf\relax \BUG\fi
\ifx\endgraf\par \else \BUG\fi

\def\Xa{\Xc}\def\Xb{\Xd} \def\Xc{\Xe}\def\Xd{\Xe}
\def\Xe{a}
\ifx\Xa\Xb \BUG \fi \ifx\Xb\Xa \BUG \fi
\ifx\Xa\Xc \BUG \fi \ifx\Xc\Xa \BUG \fi
\ifx\Xa\Xd \BUG \fi \ifx\Xd\Xa \BUG \fi
\ifx\Xa\Xe \BUG \fi \ifx\Xe\Xa \BUG \fi
\ifx\Xb\Xc \BUG \fi \ifx\Xc\Xb \BUG \fi
\ifx\Xb\Xd \BUG \fi \ifx\Xd\Xb \BUG \fi
\ifx\Xb\Xe \BUG \fi \ifx\Xe\Xb \BUG \fi
\ifx\Xc\Xd \else\BUG \fi \ifx\Xd\Xc \else \BUG \fi
\ifx\Xc\Xe \BUG \fi \ifx\Xe\Xc \BUG \fi
\ifx\Xd\Xe \BUG \fi \ifx\Xe\Xd \BUG \fi
\ifx\Xa\Xa \else \BUG \fi
\ifx\Xb\Xb \else \BUG \fi
\ifx\Xc\Xc \else \BUG \fi
\ifx\Xd\Xd \else \BUG \fi
\ifx\Xe\Xe \else \BUG \fi
\expandafter\ifx\csname undefined \endcsname\relax \else \BUG \fi
\ifx \undeffined\Undefined \else \BUG \fi
}

$x=0123456789$
$\root 3 \of 4$ $\sqrt[3]{4} \char"321 \char"4321 \char"54321 \char"654321 \char"7654321$
XYZ 

\bibliography{torture+foot}

markX\mark{a}\topmark\mark{b}\botmark\mark{c}\firstmark
\mark{d}\splitfirstmark\mark{e}\splitbotmark Y
XYZ\space{ } \space T

{\makeatletter
\outer\def\newif#1{\count@\escapechar \escapechar\m@ne
  \expandafter\expandafter\expandafter
   \def\@if#1{true}{\let#1=\iftrue}%
  \expandafter\expandafter\expandafter
   \def\@if#1{false}{\let#1=\iffalse}%
  \@if#1{false}\escapechar\count@} % the condition starts out false
\def\@if#1#2{\csname\expandafter\if@\string#1#2\endcsname}
{\uccode`1=`i \uccode`2=`f \uppercase{\gdef\if@12{}}} % `if' is required
\tracingall
\newif \ifBAR
}


\xentity{2030}\xentity{2031}\xentity{2032}\xentity{2033}\\
\xentity{2034}\xentity{2035}\xentity{2036}\xentity{2037}\\
\xentity{2113}


\newenvironment{exercice}{\begin{xmlelement*}{exercice}}{\end{xmlelement*}}
OK
\begin{exercice}
Blabla

Blabla
\end{exercice}

\def\tla#1{#1}
\long\def\tlb#1{#1}
\outer\def\tlc#1{#1}
\long\outer\def\tld#1{#1}
\newcommand\tle[1]{#1}
\newcommand*\tlf[1]{#1}
\tla{}\tlb{}\tlc{}\tld{}\tle{}\tlf{}
%\show\tla \show\tlb \show\tlc \show\tld \show\tle \show\tlf
%\tle\par \tlf0\par

{
\DefineShortVerb{\|}
\def\verbatimfont#1{{#1}}
\def\verbprefix#1{A#1A}
\def\verbatimprefix#1{B#1B}
Test: \verb+foo+ and |bar|
\UndefineShortVerb{\|}
\begin{verbatim}
line1
line2
\end{verbatim}
}





\long\def\FirstOfTwo#1#2{#1}
\long\def\SecondOfTwo#1#2{#2}

\def\IfRange#1#2{%
  \ifcase0\ifnum\expandafter`\CurrentToken<`#1 1\fi
          \ifnum\expandafter`\CurrentToken>`#2 1\fi
         \relax
% * TeX reads the number that starts with "0" and expands
%   macros until it finds a token that is not expandable and
%   is not a digit, \relax in this case.
% * The backtick generates the "character number" of the
%   follwoing token.
% * Instead of the use of \CurrentToken you can use another
%   parameter, then the \expandafter before `\CurrentToken
%   is not necessary, too.
    \expandafter\FirstOfTwo
  \else
    \expandafter\SecondOfTwo
  \fi
}
\def\IfEqual#1{%
  \expandafter\ifx\CurrentToken#1%
    \expandafter\FirstOfTwo
  \else
    \expandafter\SecondOfTwo
  \fi
}

\def\YourTestMacro#1{%
  \def\CurrentToken{#1}%
  \IfRange{A}{M}{%
    \TestOutput{A-M}%
  }{%
    \IfEqual{N}{%
      \TestOutput{N}%
    }{%
      \IfEqual{O}{%
        \TestOutput{O}%
      }{%
        \IfRange{P}{Q}{%
          \TestOutput{P-Q}%
        }{%
          \TestOutput{UNKNOWN}%
        }%
      }%
    }%
  }%
}

\def\TestOutput#1{%
  \immediate\write-1{* \CurrentToken\space is in range #1}%
}

\immediate\openout 6=tout.tex
\immediate\write6{\def\noexpand\foo{\noexpand\endinput}}
\immediate\write6{\noexpand\foo^^J\noexpand\bar}
\immediate \closeout6 \relax
\input{tout}
123

\YourTestMacro{C}
\YourTestMacro{N}
\YourTestMacro{X}

\begin{equation}
0=\hspace{0.4
cm}1
\end{equation}




Unskip:a \space\unskip b \unskip c\space \unskip d\space\space \unskip\unskip e

\let\operatorname\relax
\DeclareMathOperator{\Xfoo}{\it bar}\DeclareMathOperator*{\Yfoo}{\it bar}
DMO\Xfoo OK\Yfoo ok

\tracingall
\def\First#1#2{#1}
\def\Second#1#2{#2}
\def\w\ww{}
ok
Ifempty \ifempty{}{1}{2}\ifempty{ }{1}{2}\ifempty{\w}{1}{2}%
\yifempty{}{1}{2}\yifempty{ }{1}{2}\yifempty{\w}{1}{2}%
\bifempty{}{1}{2}\bifempty{ }{1}{2}\bifempty{\w}{1}{2}%
\zifempty{}{1}{2}\zifempty{ }{1}{2}\zifempty{\w}{1}{2}%
\ifempty{}{\First}{\Second}{1}{2}%
\ifempty{ }{\First}{\Second}{1}{2}%
\ifempty{\w}{\First}{\Second}{1}{2}%
\yifempty{}{\First}{\Second}{1}{2}%
\yifempty{ }{\First}{\Second}{1}{2}%
\yifempty{\w}{\First}{\Second}{1}{2}%
\bifempty{}{\First}{\Second}{1}{2}%
\bifempty{ }{\First}{\Second}{1}{2}%
\bifempty{\w}{\First}{\Second}{1}{2}%
\zifempty{}{\First}{\Second}{1}{2}%
\zifempty{ }{\First}{\Second}{1}{2}%
\zifempty{\w}{\First}{\Second}{1}{2} %
\yifempty{x@22}{4}{2}%
\yifempty{xy@22}{4}{2}%
\yifempty{xy@22@33}{4}{2}%
.
\if\ifempty{}{1}{2}1\else\uerror\fi
\if\ifempty{\w}{1}{2}2\else\uerror\fi
\if\yifempty{}{1}{2}1\else\uerror\fi
\if\yifempty{\w}{1}{2}2\else\uerror\fi
\if\bifempty{}{1}{2}1\else\uerror\fi
\if\bifempty{\w}{1}{2}2\else\uerror\fi
\if\bifempty{ }{1}{2}1\else\uerror\fi
\if\yifempty{ }{1}{2}1\else\uerror\fi
\count\bifempty{\w}{1}{2}=1
\count\ifempty{\w}{1}{2}=1
\count\yifempty{\w}{1}{2}=1
\count\zifempty{\w}{1}{2}=1
Ifempty 122112112122122112112122 222.

%% funny tests by David Kastrup.
\iffalse\iftrue\fi \uerror\fi
\iftrue\iffalse\fi \else \uerror\fi
\expandafter\iffalse\iftrue\fi \else \uerror\fi
\expandafter\iftrue\iffalse\fi \else \uerror\fi


\leavevmode\xbox{date}{This is a date}
\Href{a}{b} \Href{a\~n}{b} \Href{b}{a\~n}
\footnote{\href{toto~titi}{a~b}}

\newenvironment{xgather} 
    {A\begin{equation}B\begin{array}{c}} 
    {\end{array}A\end{equation}D}
\begin{xgather} x\\y \end{xgather}


\long\def\foo#1{#1}
Gobble:x\foo{\gobble\par}\FirstOfOne{\gobble\par}x%
\kFirstOfTwo{\gobble\par}{\uerror}\kSecondOfTwo{\uerror}{\gobble\par}x

soul:\caps{ABC}\hl{x\ul{y}\textit{z\st{w}}}\ul{etc}

\newcounter{Ctr}
\begin{list}{(\theCtr)}{HEY!\usecounter{Ctr}}
\makeatletter
\item[a] bla bla
\item ble ble 
\item blu blu
\begin{enumerate}
\item e1
\item e1
\def\@itemlabel{[\theenumi]}
\item e3
\item
\begin{enumerate}\def\@itemlabel{[\theenumi,\theenumii]}
\item i1 
\item i2 
\item[foo] i3 
\item[foo] i5
\end{enumerate}
\item e5
\end{enumerate}
\item bli bli
\end{list}

{\language=0
\xmlelt{a}{-}\xmlelt{a}{--}\xmlelt{a}{'}\xmlelt{a}{`}\xmlelt{a}{<}%
\xmlelt{a}{>}%
\xmlelt{a}{--}\xmlelt{a}{---}\xmlelt{a}{''}\xmlelt{a}{``}\xmlelt{a}{<<}%
\xmlelt{a}{>>}
}
{\language=1
\xmlelt{a}{-}\xmlelt{a}{--}\xmlelt{a}{'}\xmlelt{a}{`}\xmlelt{a}{<}%
\xmlelt{a}{>}%
\xmlelt{a}{--}\xmlelt{a}{---}\xmlelt{a}{''}\xmlelt{a}{``}\xmlelt{a}{<<}%
\xmlelt{a}{>>}
}


\muskip0 = 18mu plus 36mu minus 1 fill
\skip0 = 10pt plus 20pt minus 1 fil
\edef\foo{%
\the\muskip0,%
\the\mutoglue\muskip0,%
\the\skip0,%
\the\gluetomu\skip0,%
\the\mutoglue\gluetomu\skip0,%
\the\glueshrink\skip0,%
\the\gluestretch\skip0,%
\the\glueshrinkorder\skip0,%
\the\gluestretchorder\skip0}

{\let\GDEF\gdef\let\XDEF\xdef\def\S{ }
 \catcode`m=12 \catcode`u=12 \catcode`p=12 \catcode`f=12  
 \catcode`i=12  \catcode`l=12 \catcode`n=12 \catcode`i=12 \catcode`s=12 
 \catcode`t=12 
 \GDEF\MU{mu}\GDEF\PT{pt}\GDEF\FIL{fil}\GDEF\FILL{fill}%
 \GDEF\PLUS{plus}\GDEF\MINUS{minus}
\XDEF\xoo{18.0\MU\S \PLUS\S 36.0\MU\S \MINUS\S 1.0\FILL,%
18.0\PT\S \PLUS\S 36.0\PT\S \MINUS\S 1.0\FILL,%
10.0\PT\S \PLUS\S 20.0\PT\S \MINUS\S 1.0\FIL,% 
10.0\MU\S \PLUS\S 20.0\MU\S \MINUS\S 1.0\FIL,%
10.0\PT\S \PLUS\S 20.0\PT\S \MINUS\S 1.0\FIL,%
1.0\PT,20.0\PT,1,0}}

\edef\yoo{\detokenize{18.0mu plus 36.0mu minus 1.0fill,%
18.0pt plus 36.0pt minus 1.0fill,%
10.0pt plus 20.0pt minus 1.0fil,% 
10.0mu plus 20.0mu minus 1.0fil,%
10.0pt plus 20.0pt minus 1.0fil,%
1.0pt,20.0pt,1,0}}

\xtesteq\foo\xoo
\xtesteq\yoo\xoo

\parshape 3 1pt 2pt 3pt 4pt 5pt 6pt
\ifnum\parshape=3 \else\bad\fi
\ifdim\parshapelength 1 = 2.0pt\else\bad\fi
\ifdim\parshapeindent 2 = 3.0pt\else\bad\fi
\ifdim\parshapedimen 4  = 4.0pt\else\bad\fi
\ifdim\parshapedimen 5  = 5.0pt\else\bad\fi
\ifdim\parshapedimen 6  = 6.0pt\else\bad\fi
\ifdim\parshapedimen 7  = 5.0pt\else\bad\fi
\ifdim\parshapedimen 0  = 0.0pt\else\bad\fi
\parshape 0
\ifdim\parshapedimen 1  = 0.0pt\else\bad\fi

\ifdim \dimexpr(2pt-5pt) *\numexpr 3-3*13/5\relax + 34pt/2=32pt
\else\bad\fi

\def\foo#1#2{\number#1
  \ifnum#1<#2, %
  \expandafter\foo
  \expandafter{\number\numexpr#1+1\expandafter}%
  \expandafter{\number#2\expandafter}%
  \fi}

\edef\Bar{\foo{7}{13}}
\def\xBar{7, 8, 9, 10, 11, 12, 13}
\ifx\Bar\xBar\else \bad\fi


\ifnum \numexpr \dimexpr 10sp=10\else \bad\fi
\skip0=\glueexpr 3pt plus \dimexpr2pt+3pt\relax minus \numexpr3fill \relax
%\showthe\skip0
\edef\foo{\the\skip0}
\edef\xfoo{\detokenize{3.0pt plus 5.0pt minus 3.0fill}}
\xtesteq\foo\xfoo
\muskip0=\muexpr 6mu plus 4mu minus 6 fill/2 
\edef\foo{\the\muskip0}
\edef\xfoo{\detokenize{3.0mu plus 2.0mu minus 3.0fill}}
\xtesteq\foo\xfoo

\ifdefined x \else \bad\fi
\ifdefined \undefined \bad\fi
\ifdefined \par\else\bad \fi
\ifcsname foo bar\endcsname\bad\fi
\ifcsname bar\endcsname\else\bad\fi
\csname foo bar\endcsname\ifcsname foo bar\endcsname\else\bad\fi
\unless\iffontchar\font 32 \bad\fi
\iffontchar\nullfont 32 \bad\fi

\def\foo{A}
\edef\FooA{\foo\unexpanded\relax{\foo\foo}\relax{\foo\foo}}
\testeq\FooA{A\foo\foo\relax{AA}}

{
{\catcode`~12\xdef\B{~}}
\def\A{a}
\def~{a}
\if\noexpand~\A \bad\fi
\if\noexpand~\B \else \bad\fi
\if~\A \else\bad\fi
}



% test \@car\@cdr
\makeatletter
\newcommand\tfoo[1][x]{\def\foo{#1}}
\tfoo \testeq\foo{x}
\tfoo[3] \testeq\foo{3}
\@testopt\tfoo{44} \testeq\foo{44} 
\@testopt\tfoo{{44}} \testeq\foo{{44}} 
\@testopt\tfoo{44}[55] \testeq\foo{55} 
\@testopt\tfoo{44}[{55}] \testeq\foo{55} 
\@testopt\tfoo{44}[{{55}}] \testeq\foo{{55}} 
\@testopt{\def\bar{3}\tfoo}{44}[55] \testeq\foo{55} \testeq\bar{3} 
\@testopt{\def\bar{4}\tfoo}{44} \testeq\foo{44} \testeq\bar{4} 

\edef\foo{\@car123\@nil}\testeq\foo{1}
\edef\foo{\@car {1}23\@nil}\testeq\foo{1}
\edef\foo{\@car {134}23\@nil}\testeq\foo{134}
\edef\foo{\@cdr123\@nil}\testeq\foo{23}
\edef\foo{\@cdr {1}23\@nil}\testeq\foo{23}
\edef\foo{\@cdr {134} 23\@nil}\testeq\foo{ 23}
\edef\foo{\@cdr {134}\@nil}\testeq\foo{}
\edef\foo{\@cdr {134}{x}\@nil}\testeq\foo{x}
\edef\foo{\@cdr {134}{{x}}\@nil}\testeq\foo{{x}}
\edef\foo{\@cdr {134}{{x}} \@nil}\testeq\foo{{{x}} }
\def\@elt{12}\@cons\foo{\@car123\@nil}\@cons\foo{4}\testeq\@elt{12}
\testeq\foo{{{x}} \@elt1\@elt4}
0\@gobble1\@gobbletwo12\@gobblefour1234%
\@gobble\par\@gobbletwo\par\par\@gobblefour\par\par\par\par0
\def\ABC[#1]#2{\def\fooi{#1#2}}\def\foo{\@testopt\ABC{x}}\foo[2]{3}%
\testeq\fooi{23}\foo{3}\testeq\fooi{x3}

{\def\foo{A}\@cons\foo{\@car123\@nil}\@cons\foo{4}} % cons in a group
\testeq\foo{A\@elt1\@elt4}

\makeatother

{
\makeatletter
\long\def\test#1{\def\res{#1}\ifx\foo\res\else \ERROR\fi}
\edef\foo{\@car 123\@nil} \test{1}
\edef\foo{\@car {1}23\@nil} \test{1}
\edef\foo{\@car {123}{456}{7}\@nil}\test{123}
\edef\foo{\@cdr 123\@nil} \test{23}
\edef\foo{\@cdr {134}{x}\@nil}\test{x}
\edef\foo{\@cdr {134}{{x}}\@nil}\test{{x}}
\def\foo{\@nil} \ifx\foo\@nnil\else \ERROR\fi
\let\foo\@nnil \test{\@nil}
\toks@={abc\foo}\addto@hook\toks@{x\bar}
\expandafter\def\expandafter\foo\expandafter{\the\toks@} \test{abc\foo x\bar}
\g@addto@macro\foo{y\gee} \test{abc\foo x\bar y\gee}
\def\xx{456}
\def\foo{123}\@cons\foo{\xx78}\test{123\@elt45678}
\toks@={abc\foo}\addto@hook\toks@{x\par}
\expandafter\def\expandafter\foo\expandafter{\the\toks@} \test{abc\foo x\par}
}

\makeatletter 
\def\Tin#1#2#3#4{\in@{#1}{#2}\ifin@#3\else #4\fi}
\def\RM#1#2{\@expandtwoargs\@removeelement{#1}{#2}#2}
\let\Tfor\@tfor\let\BreakTfor\@break@tfor\let\Lfor\@for
\let\Xforn\tralics@for@n
\let\Xforo\tralics@for@o
\let\Xforeo\tralics@for@eo
\let\Xforen\tralics@for@en
\let\Xbreak\tralics@for@break
\edef\foo{\zap@space 1 2 345 \@empty 6 7\strip@prefix 1134>89}
\testeq\foo{123456 789}
{\def\@empty{t}\edef\foo{\zap@space0 \@empty}\xtesteq\foo{0}} 
\count0=11 
\edef\foo{\hexnumber@1 \hexnumber@3 \hexnumber@ 10 \hexnumber@{\count0}\hexnumber@15 }
\testeq\foo{13ABF}

\def\remove@fl@space#1#2{%
\expandafter\def\expandafter #1\expandafter {\zap@fl@space{#2}}}
\remove@fl@space\foo{ \empty\space Hello  World }
\testeq\foo{\empty\space Hello  World}
\makeatother
{\everyxbox{Vest}
\setbox0=\xbox{foo}{1\xbox{bar}{2} %
   \XMLaddatt[\XMLlastid]{x}{1}\XMLaddatt[\XMLcurrentid]{y}{2}\XMLaddatt{y}{22}3}
\box0
}
{\everyxbox{West}
\setbox0=\xbox{foo}{1\xbox{bar}{2} %
   \XMLaddatt*[\XMLlastid]{x}{1}\XMLaddatt*[\XMLcurrentid]{y}{2}\XMLaddatt*{y}{4}3}
\box0
}



A\clearpage B\cleardoublepage C\newpage D \hrulefill E \dotfill F
\samepage G
\pagenumbering{arabic} \pagestyle{mypagestyle}\thispagestyle{plain}

\Tin4{12345}{}{\bad}
\Tin6{12345}{\bad}{}
\Tin{12}{1x2x3x4x5x12}{}{\bad}
\Tin{123}{1x121221\bad2121121212121212x4x1235x\bad12}{}{\bad}
\Tin{123}{1x1212212121121212121212x4x125x12}{\bad}{}
\Tin{1B3}{1x1B1BB1B1B11B1B1B1B1B1Bx4x1B3{}5x1B}{}{\bad}
\Tin{1B 3}{1x1B 1B B 1B 1B 11B 1B 1B 1B 1B 1B x4x1B 35x1B}{}{\bad}

\def\foo{A,B
C,D,E F}\RM{D}\foo\testeq\foo{A,B C,E F}\RM{D}\foo\testeq\foo{A,B C,E F}
\RM{B}\foo\testeq\foo{A,B C,E F}
\RM{B C}\foo\testeq\foo{A,E F}
\RM{A}\foo\testeq\foo{E F}
\RM{E F}\foo\testeq\foo{}

\def\List{}%% list not expanded here
\Tfor\foo:=12345\do{\edef\List{\List\foo}\if\foo4\BreakTfor\fi}
\Tfor\foo:=12 345\do{\edef\List{\List\foo}}
\Tfor\foo:=6\do{\edef\List{\List\foo}}
\Tfor\foo:=78\do{\edef\List{\List\foo}}
\Tfor\foo:=\do{\edef\List{\List\foo}}
\testeq\List{123412345678}

\def\List{}\def\thelist{12,3,4,5,6} % list is expanded here
\Lfor\foo:=\thelist\do{\edef\List{\List\foo}\if\foo4\BreakTfor\fi}
\Lfor\foo:=1,23,45\do{\edef\List{\List\foo}}
\Lfor\foo:=6\do{\edef\List{\List\foo}}
\Lfor\foo:=7,8\do{\edef\List{\List\foo}}
\Lfor\foo:=\do{\edef\List{\List\foo}}
\testeq\List{123412345678}

\def\List{}\def\thelist{12,13,14,15,16} 
\Xforn{1,2,3,4}\foo{\if4\foo\Xbreak\else\edef\List{\List+\foo}\fi}
\Xforn{1,2,3,4}\foo{\if1\foo\Xbreak\else\edef\List{\List+\foo}\fi}
\Xforn{}\foo{\edef\List{\List?\foo}}
\Xforen{}\foo{\edef\List{\List-\foo}}
\Xforo\thelist\foo{\edef\List{\List+\foo}}
\Xforo\empty\foo{\edef\List{\List?\foo}}
\Xforeo\empty\foo{\edef\List{\List=\foo}}
\Xforeo\empty\foo{\edef\List{\List=\foo}}
\testeq\List{+1+2+3-+12+13+14+15+16==}


\def\expandA{\expandB}
\protected\def\expandB{\expandC}
\def\expandC{\count0=0}
\count0=17
\edef\foo{\expandA\expandB\expandC}
\def\xfoo{\expandB\expandB\count0=0}
%\show\foo\show\xfoo
\ifx\foo\xfoo\else \bad\fi

\expandA \makeatletter
\def\xfoo#1#2{\def\testa{x#1#2}}
\def\yfoo#1#2{\def\testb{y#1#2}}
\def\testA{xua}\def\testB{yvb}
\def\test#1{\ifnum\count0=#1 \@afterelsefi\xfoo u \else\@afterfi\yfoo v\fi} 
\test0a \test1b 
\ifx\testa\testA\else\bad\fi
\ifx\testb\testB\else\bad\fi
\def\testc#1{%
   \ifnum\count0=#1 %
   \@afterelsefi \fct u 
   \else\@afterfi\fct v\fi} 
\def\fct#1#2{}\edef\foo{\testc023}\testeq\foo{3}
\makeatother


Reference to figure: \ref{fig:fig1}
Reference to first subfigure: \ref{fig:fig1a}
Reference to second subfigure: \ref{fig:fig1b}

\csname tralics@use@subfigure\endcsname=0
\begin{figure}[htbp]
    \centering
    \subfigure[first subfigure caption]{
           \includegraphics{figure1a.eps}
       \label{fig:fig1au}
    }
    \subfigure[second subfigure caption]{
           \includegraphics[width=3cm]{figure1b.eps}
       \label{fig:fig1bu}
    }
    \caption{main figure caption}
    \label{fig:fig1u}
\end{figure}


\csname tralics@use@subfigure\endcsname=1
\begin{figure}[htbp]
  \XMLaddatt{FOO}{BAR}
    \centering
    \subfigure[first subfigure caption]{
           \includegraphics{figure1a.eps}
       \label{fig:fig1a}
    }
    \hspace{2cm}
    \subfigure[second subfigure caption]{
           \includegraphics[width=3cm]{figure1b.eps}
       \label{fig:fig1b}
    }
\par
    \subfigure{\includegraphics{figure1c}}
    \caption{main figure caption}
    \label{fig:fig1}
\end{figure}

{\makeatletter
\edef\xoo{\the\bigskipamount,\the\medskipamount,\the\smallskipamount,\the\hideskip}
\edef\yoo{\detokenize{12.0pt plus 4.0pt minus 4.0pt,%
6.0pt plus 2.0pt minus 2.0pt,%
3.0pt plus 1.0pt minus 1.0pt,%
-1000.0pt plus 1.0fill}}
\xtesteq\yoo\xoo
\topsep=1ptminus2pt \partopsep=1ptminus2pt \@tempskipa=1ptminus2pt \@tempskipb=1ptminus2pt \@flushglue=1ptminus2pt \listparindent=1ptminus2pt
\hideskip=1ptminus2pt \z@skip=1ptminus2pt \skip@=1ptminus2pt \normalbaselineskip=1ptminus2pt \normallineskip=1ptminus2pt \smallskipamount=1ptminus2pt
\medskipamount=1ptminus2pt \bigskipamount=1ptminus2pt \floatsep=1ptminus2pt \textfloatsep=1ptminus2pt \intextsep=1ptminus2pt
\dblfloatsep=1ptminus2pt \dbltextfloatsep=1ptminus2pt

}

%% Testing that these are dimensions
{\makeatletter
\advance\maxdimen by-"3FFFFFFFsp \ifdim\maxdimen=0pt\else\bad\fi
\textwidth=3cm\linewidth=3cm\columnwidth=3cm
\paperheight=3cm \paperwidth=3cm \headheight=3cm \headsep=3cm
\footskip=3cm\jot=3cm\marginparwidth=3cm \marginparsep=3cm \marginparpush=3cm
\tabcolsep=3cm\arraycolsep=3cm\footnotesep=3cm\doublerulesep=3cm
\arrayrulewidth=3cm\@tempdima=3cm\@tempdimb=3cm \@tempdimc=3cm\topmargin=3cm
\dimen@=4pt\ifdim\dimen@=\dimen0 \else\bad\fi
\dimen@i=5pt\ifdim\dimen@i=\dimen1 \else\bad\fi
\dimen@ii=6pt\ifdim\dimen@ii=\dimen4 \else\bad\fi
}

{\makeatletter
\setcounter{bottomnumber}{3}\setcounter{topnumber}{3}
\setcounter{dbltopnumber}{3}\setcounter{totalnumber}{3}
\setcounter{footnote}{17}\setcounter{mpfootnote}{17}
\@tempcnta=3 \@tempcntb=4 
\interfootnotelinepenalty=5 \interdisplaylinepenalty=6
\setbox\@tempboxa\xbox{foo}{bar}\setbox\voidb@x\xbox{foo}{bar}
\@temptokena=\relax{foo\bar}\@spaces
\expandafter\def\expandafter\foo\expandafter{\@spaces}
\testeq\foo{\space\space\space\space}
\testeq\fmtname{Tralics}
\testeq\encodingdefault{T1}
\testeq\familydefault{cmr}
\testeq\seriesdefault{m}
\testeq\shapedefault{n}
\def\test{\bgroup \@sanitize\ifnum\catcode`\$=12 \else\bad\fi
  \def\do##1{\catcode`##1=3\relax}\dospecials \ifnum\catcode`\}=3\else\bad\fi
  \egroup}\test
\newread\NRA\newwrite\NRB\newmuskip\NRC\newlanguage\NRD
\NRC=3mu plus 4mu
\count\NRA=0 \count\NRB=0 \count\NRD=0
\testeq\baselinestretch{1}
\testeq\arraystretch{1}
}

\offinterlineskip \nointerlineskip \frenchspacing \nonfrenchspacing
\showoverfull \loggingoutput \showoutput \nofiles \sloppy \fussy
\onecolumn \twocolumn \flushbottom \raggedbottom \normalmarginpar 
\reversemarginpar \normalbaselines \removelastskip 
\edef\srelax{\ifnum0=0\fi}
\edef\foo{\ifnum0=7\expandafter\fi\expandafter*\expandafter\iftrue\fi}
\xtesteq\foo{*\srelax}
\makeatletter

\edef\foo{\ifnum0=7\expandafter\fi\expandafter*\expandafter\iftrue\expandafter\@gobble\fi\relax}
\testeq\foo{*\relax}
\edef\foo{\ifnum0=0\expandafter\fi\expandafter*\expandafter\iftrue\fi}
\xtesteq\foo{\srelax*\srelax}\marks0{OK}
\edef\foo#1{#\iftrue12\fi #\iffalse12\fi1}\def\yoo#1{#12#1}\ytesteq\foo\yoo

\def\xfoo#1#2{\def\foo{{#2}{#1}}}%% note the order
\Scanupdown\xfoo{a}{b}{etc}\testeq\foo{{a}{b}}
\Scanupdown\xfoo{a}{b}{^{uv}}\testeq\foo{{uv}{b}}
\Scanupdown\xfoo{a}{b}{_{uv}}\testeq\foo{{a}{uv}}
\Scanupdown\xfoo{a}{b}{^{x}_{uv}}\testeq\foo{{x}{uv}}
\Scanupdown\xfoo{a}{b}{'^{x}_{uv}}\testeq\foo{{\prime x}{uv}}
\Scanupdown\xfoo{a}{b}{_{uv}'^{x}}\testeq\foo{{\prime x}{uv}}


%%%%%%%% x-------------------x


\begin{table}\centering
$\text{a\quad b\qquad c\!d\,e\;f\:g h~i}
a\quad b\qquad c\!d\,e\;f\:g h~i$
\caption{Infinite and finite discrete time signal processing.
\label{m:inffin}}
\end{table}
\begin{table}\centering \xbox{foo}{bar} \end{table}

\def\boldmath{\mathversion{bold}}
\def\unboldmath{\mathversion{normal}}
\newcommand{\bxi}{{\mbox{\boldmath{$\xi$}}}}
\newcommand{\uxi}{\mbox{\unboldmath$\xi$}}


\ref{exA}\ref{exB}\ref{exA2}\ref{exB2}
\begin{Exa}
\label{exA}
 This is a sentence typeset \par \label{exB}in the theorem environment \Lenv{Exa}.
\end{Exa}

\ChangeElementName{theorem}{ theorem}
\begin{Exa}
\label{exA2}
 This is a sentence typeset \par \label{exB2}in the theorem environment \Lenv{Exa}.
\end{Exa}

\ChangeElementName{theorem}{theorem}
\begin{Exa}x \end{Exa}
\ChangeElementName{theorem}{}
\begin{Exa}x \end{Exa}
\ChangeElementName{theorem}{theorem}
\begin{Exa}x \end{Exa}
\ChangeElementName{theorem}{}
\begin{Exa}x \end{Exa}



$\noexpand\undefinded A\xi\bxi_1 B\uxi_1$
\boldmath
$A\xi\bxi_1 B\uxi_1$
\unboldmath
$\hbox{{\ensuremath\alpha}}$

\makeatletter
\def\Edef#1{\expandafter\def\expandafter#1\expandafter}
\Edef\fooa{\tralics@split{L@}\A\B{u=v,w,, U = V}}
\def\foob{\A {L@u}{v}\B {L@w}\A {L@U}{V}}
\ytesteq\fooa\foob

\centerline{A\it B}\leftline{C\it D}\rightline{E\it F}\rlap{G\it H}\llap{I\it J}
\marginpar{a\bf B}c\samepage
\noexpand\bad\rule{2pt}{3pt}
\rule[1pt]{3pt+5pt}{17PT}
\vglue 1pt plus 2pt  minus 3pt
\vglue 1pt plus 2pt  minus 3pt
\vglue 1pt plus 2pt  minus 3pt
\hglue 10pt plus 2pt  minus 3pt

\renewcommand{\labelenumi}{\Roman{enumi}.}
\renewcommand{\labelenumii}{\Alph{enumii})}
\renewcommand{\labelenumiii}{(\alph{enumiii})}
\renewcommand{\labelenumiv}{\roman{enumiv}.}
\makeatletter
\def\enumi@hook{
\AddAttToCurrent{list-counter}{Roman}}
\def\enumii@hook{\AddAttToCurrent{list-counter}{Alph}}
\def\enumiii@hook{\AddAttToCurrent{list-counter}{alph}}
\def\enumiv@hook{\AddAttToCurrent{list-counter}{roman}}
%\section{crazy nested Enumerated List}
\begin{enumerate}
\item one
\item two
\item three
    \begin{enumerate}
    \item three.one
    \item three.two
    \item three.three
        \begin{enumerate}
        \item three.three.one
        \item three.three.two
        \item three.three.three
            \begin{enumerate}
            \item three.three.three.one
            \item three.three.three.two
            \item three.three.three.three \ref{xx} \label{xx}
            \end{enumerate}
        \end{enumerate}
    \end{enumerate}
\end{enumerate}

\begin{table}
\def\IC#1{\includegraphics[width=#1cm]{x}}
\begin{tabular}{cc}
\IC{1}&\IC{2}\\
\IC{3}&\IC{4}\\
\end{tabular}
\caption[opt arg for caption]{a table with a caption}
\end{table}

border matrix $\bordermatrix{
   &C&I&C'\cr
 C&1&0&0\cr
I&b&&-b&0\cr
C&a'&1-a\cr}
$

\begin{subequations}
\begin{equation} A\end{equation}
\begin{equation} B \end{equation}
\end{subequations}
where
\begin{subequations}
\begin{equation} C \end{equation}
\begin{equation} D \end{equation}
\end{subequations}

\edef\foo{\multispan3x}\testeq\foo{\omit\span\omit\span\omit x}
\def\SBU{Upper }
\def\SBL{Lower }
\def\SBM{Mixed }\ref{Cd1}\ref{Cd2}
\omitcite{not-a-citation}\omitcite{tor1}\omitcite{tor1}%
a: \MakeTextUppercase{\SBU \NoCaseChange{MiXeD} \NoCaseChange{\SBM} \SBU }\\
b: \MakeTextUppercase{\SBU $\rm Bar \SBM$  \(Yy\), \ensuremath{Zz} \SBU}\\
c: \MakeTextUppercase{Text \cite{mathml2} and \cite[Page 12]{mathml2} \SBU}
d: \MakeTextUppercase{\ref{label-A} \label{Cd1}  \SBU}\\

a: \MakeTextLowercase{\SBL \NoCaseChange{MiXeD} \NoCaseChange{\SBM} \SBL }\\
b: \MakeTextLowercase{\SBL $\rm Bar \SBM$  \(Yy\), \ensuremath{Zz} \SBL}\\
c: \MakeTextUppercase{Text \cite{mathml2} and \cite[Page 12]{mathml2} \SBL}
d: \MakeTextLowercase{\ref{label-A} \label{Cd2}  \SBL}\\
%
%
 
\typeout{\thisistheend} 
\end{document}

fin du document

