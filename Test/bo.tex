\documentclass[AIF]{cedram}

\begin{document}

\title{Fourier coefficients for simple $L^\infty$ functions}
\alttitle{Coefficients Fourier pour fonctions  $L^\infty$ simples}
\author{\firstname{Donald} \middlename{E.} \lastname{Knuth}} 

\address{\TeX\ Users Group \\P.O. Box 869\\
Santa Barbara, CA 93102-0869 USA}

\email{d.e.knuth@somewhere.on.the.net} 

\subjclass{11M26,  11M36, 11S40}

\keywords{simple $L^\infty$ functions, lambda function}

\altkeywords{fonctions  $L^\infty$ simples, fonction lambda}

\daterecieved{2004-06-14}%{14 juin 2004}
\dateaccepted{2004-12-09}%{9 d�cembre 2004}

\begin{abstract}
  This is an abstract with a beautiful inline formula % Comment!
  $\lambda_n(\pi) = \frac{N}{2} n \log n + C_1(\pi) n + 
   O(\sqrt{n}\log{n})$, where $C_1(\pi)$ is a real-valued constant.
\end{abstract}


\begin{altabstract}
  Mon r�sum� avec ma formule
  $\lambda_n(\pi) = \frac{N}{2} n \log n + C_1(\pi) n + 
   O(\sqrt{n}\log{n})$, o� $C_1(\pi)$ est une constante r�elle.
\end{altabstract}

\maketitle

\section{Introduction}

The content of the document is unimportant.
We have a simple math formula $\alpha=\beta$
and two references \cite{Ba03} and \cite{BPY1}


\bibliography{bo}

\end{document}
