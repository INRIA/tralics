% tralics configuration file 'test'
% This file tests that a lot of XML element and attribute names
% can be changed

\documentclass{book}
\usepackage{curves,ra,soul}
\begin{document}

\notrivialmath=7

\newcounter{bib}
\everybibitem{\stepcounter{bib}%
\edef\foo{%
bk=\XMLgetatt[\the\XMLlastid]{bib-key},%
BK=\XMLgetatt[\the\XMLlastid]{Bibkey},%
N=\XMLgetatt[\the\XMLlastid]{},%
S=\XMLgetatt{}.
}
\xbox{bib}{\thebib\XMLaddatt*{}{BiB}}\xbox{}{\foo}}

%%% test divisions
\frontmatter
\chapter{Chapter one}  OK
\mainmatter
\part{First Part of document}
Some text here
\part{Second Part}
\chapter{Chapter two}
\section{Section one}
\subsection{Subsection one}
\subsubsection{Subsubsection one}
\paragraph{First paragraph}
\subparagraph{First sub paragraph}
Normal text\label{a}
\backmatter
We are now in the backmatter.\ref{a}\pageref{a}

%%% test tables 
\newcolumntype{+} {>{B}l <{D}|}

\begin{tabular*}{3cm}[t]{|ll|rr|cc|}
\hline a&b&c&d&e&f\\
A&\multicolumn{3}{+}{C}&E&F\\
\multicolumn{2}{|l}{ab}&c&d&e&f\\[1cm]
\cline{1-3}\cline{6-6}\\\hline
\end{tabular*}

%%% Test Graphics
\def\IC#1{\includegraphics[width=#1cm]{Logo-INRIA-couleur}}
\IC{1}
\fbox{\includegraphics[clip=, angle=90, scale=3, %
       width=10cm, height=3mm]{A}}

%%% Test Float

\begin{figure}
\subfigure[~one]{\IC{2}} \subfigure{\IC{3}} \par
\subfigure[~two]{\IC{4}} \subfigure{\IC{1.3}} 
\caption{Caption of the figure}
\end{figure}
\begin{table}[b]
\begin{tabular}[t]{cc}
a&b
\end{tabular}
\caption{ok}
\end{table}

a\footnote{B}\footnote{C\par D}
\caption{a caption}
\subfigure[~one]{\IC{2}} 
\noindent \mbox{\xbox{a}{b}} \rotatebox{30}{x}
\subfigure[a]{b}

\begin{tabular}{c}a\end{tabular}
\caption{A}
\begin{table}[htbp]
\begin{tabular}{c}a\end{tabular}
\caption{B}
\end{table}
\begin{table}[htbp]
\includegraphics{a}
\caption{C}
\end{table}
\begin{figure}[htbp]
\includegraphics{a}
\caption[ok]{D}
\end{figure}
\caption{a}\caption[b]{c}

%% Boxes
\mbox{fo\xbox{a}{b}}
\scalebox{0.3}{foo}
\raisebox{0.3}{foo}
\framebox{foo}
\framebox[2cm]{foo}
\makebox(1,2)[bl]{foo\xbox{a}{b}}
\rotatebox{45}{\it x}
\fbox{foo}\scalebox{0.3}{foo}\framebox(10,10){foo}\framebox[10pt][11]{foo}


%%% Picture
\begin{picture}(100,100)
\qbezier(0,0)(10,30)(50,30)
\put(15,20){\circle{6}}
\put(15,20){\circle*{6}}
 \scaleput(10,10){\arc(5,0){121}}
  \multiput(20,5)(-20,12){2} {\line(5,3){20}}
\put(10,32){\vector(-2,1){10}}
\put(15,20){\oval(30,10)}
\dashline{3}[0.7](0,18)(63,18)
\drawline some text here% well
\dottedline{2}(0,10)(70,10)
\put(30,20){\circle{6}}
\put(15,20){\bigcircle{2}}
\closecurve(150,0, 190,100, 230,0)
\curve[17](300,0, 340,100, 380,0)
\tagcurve(80,0, 0,0, 40,100, 80,0, 0,0)
\thicklines
\thinlines
\linethickness{.5pt}
\end{picture}

%%% Lists 
\begin{itemize}
\makeatletter
\@item [A]b
\@@item [A]b
\end{itemize}

\begin{list}{}{}
\item[item label] item value
\begin{description}
\item[first item] okok
\item[second item]
\begin{itemize}
\item First
\item Second
\begin{enumerate}
\item some item
\item another one
\end{enumerate}
\end{itemize}
\end{description}
\end{list}

%%% Theorems
\newtheorem{xCor}{Corollary}
\begin{xCor}[Important] this is easy \end{xCor}

%%% Fonts
\ul{a}\caps{b}\hl{c}\so{d}\st{e}
\textsuperscript{f}\textsubscript{g}
\oldstylenums{h}\overline{i}\underline{j}

{\Huge a}{\huge b}{\LARGE c}{\Large d}{\large e}{\normalsize f}{\small
  g}{\footnotesize h}{\scriptsize i}{\tiny j}

\textbf{a}\textsf{b}\texttt{c}\textsc{d}\textsl{e}\textit{f}
\textup{x}\textmd{x}\textrm{x} %% currently these do not work
{\upshape a}{\itshape b}{\slshape c}{\scshape d}
{\mdseries e}{\bfseries f}{\ttfamily h}{\sffamily h}{\rmfamily i}
Normal text

%%% Bibliography
\makeatletter {
\leavevmode\@setmode 4
\citation{A}{cite:B}{bid2}{D}{E}
\bauthors{\bpers {a}{}{c}{d}}
\beditors{\bpers[AA] {a}{b}{c}{d}}
\endcitation
\@setmode 1}\par

\def\cite@prenote{cf}
\def\cite@@type{fx}
\cite[p 30]{toto}

\xbox{Bibitem}{\XMLsolvecite{toto}{key1}}
OKOK \bibitem [toto] {titi}{tata}
\ChangeElementName{bibitem}{Bibitem}
\ChangeElementName{natcite}{NatCite}
\ChangeElementName*{bibkey}{Bibkey}
\tracingall
\bibitem [toto] {titi2}{tata}
\natcite{titi2}

%%% Raweb specific stuff 
No everything tested here
\def\X{\pers{a}{b}} \def\Y{\pers{c}{d}}
\begin{participant} \X\Y \end{participant}

\begin{xmlelement*}{catperso}\xbox{head}{xx}
%\pers {jean}{dupond}{PHD}{Cnrs}[Ok][not yet]
\end{xmlelement*}

\begin{glossaire}
\glo{aa}{bb}
\glo{cc}{dd}
\end{glossaire}

%%% Other 
Text
\setbox0\xbox{foo}{bar}%
\leaders\copy0\hfil\cleaders\copy0\hfil\xleaders\copy0\hfil
\noindent Word\index{foo}
\index{foo!a@barb|gee}
\index{foo!b@bara}%
\index{foo}
\begin{motscle} first, second,  {(max,+)}\end{motscle}
\url{foo}
\tableofcontents
\makeindex
{\@nomathml=-1 $xy$
\def\[{$$\formulaattribute{textype}{Bdisplay}}  \[ x \] 
\def\({$\formulaattribute{textype}{Binline}}  \( x \) }
\def\unused{$$ and $} % Emacs



a\\[2pt]b
\begin{center}A\end{center}
\begin{quote}B\end{quote}
\begin{quotation}C\end{quotation}
\begin{flushleft}D\end{flushleft}
\begin{flushright}E\end{flushright}

\begin{tabular}{|r|l|c}
\hline a&b&\multicolumn{1}{c|}{x}\\[2pt]
\end{tabular}
$\begin{array}{|r|l|}
a&b&\multicolumn{1}{c|}{x}\\[2pt]
\end{array}$

\begin{minipage}[t][2cm][s]{3cm}
ok
\end{minipage}

\theorembodyfont{\sl}
\theoremstyle{break}
\newtheorem{Cor}{Corollary}
\theoremstyle{plain}
%\newcounter{section}
\setcounter{section}{17}
\newtheorem{Exa}{Example}[section]
{\theorembodyfont{\rmfamily}\newtheorem{Rem}{Remark}}
\theoremstyle{marginbreak}
\newtheorem{Lem}[Cor]{lemma}
\theoremstyle{change}
\theorembodyfont{\small\itshape} \newtheorem{Def}[Cor]{Definition}
\theoremheaderfont{\scshape}
\def\Lenv#1{\texttt{#1}}
\begin{Cor}
 This is a sentence typeset in the theorem environment \Lenv{Cor}.
\end{Cor}
\ref{exA}\ref{exB}
\begin{Exa}
\label{exA}
 This is a sentence typeset \par \label{exB}in the theorem environment \Lenv{Exa}.
\end{Exa}
\begin{Rem}
 This is a sentence typeset in the theorem environment \Lenv{Rem}.
\end{Rem}
\begin{Lem}[Ben User]
 This is a sentence typeset in the theorem environment \Lenv{Lem}.
\end{Lem}
\begin{Def}[Very Impressive definition]
 This is a sentence typeset in the theorem environment \Lenv{Def}.
\end{Def}

\setlength{\unitlength}{1.8mm}
\begin{picture}(40,30)
  \thicklines
  \multiput(20,5)(20,12){2} {\line(0,-1){2}\line(-5,3){20}}
  \multiput(20,5)(-20,12){2} {\line(5,3){20}}
  \put(20,3){\line(5,3){20}}
  \put(20,3){\line(-5,3){20}}
  \put(0,15){\line(0,1){2}}
  \linethickness{1pt}
  \put(20,5) {
   \renewcommand{\xscale}{1}%
   \renewcommand{\xscaley}{-1}%
   \renewcommand{\yscale}{0.6}%
   \renewcommand{\yscalex}{0.6}%
   \scaleput(10,10){\bigcircle{10}}
   \put(0,-2){%
      \scaleput(10,10){\arc(5,0){121}}
      \scaleput(10,10){\arc(5,0){-31}}}}
\end{picture}\hspace{1.5cm}
\setlength{\unitlength}{1mm}
\newcounter{cms}
\begin{picture}(50,39)
\put(0,7){\makebox(0,0)[bl]{cm}}
\multiput*(10,7)(10,0){5}{\addtocounter{cms}{1}\makebox(0,0)[b]{\arabic{cms}}}
\put(15,20){\circle{6}}
\put(30,20){\circle{6}}
\put(15,20){\circle*{2}}
\put(30,20){\circle*{2}}
\put(10,24){\framebox(25,8){car}}
\put(10,32){\vector(-2,1){10}}
\multiput(1,0)(1,0){49}{\line(0,1){2.5}}
\multiput(5,0)(10,0){5}{\line(0,1){3.5}}
\thicklines
\put(0,0){\line(1,0){50}}
\multiput(0,0)(10,0){6}{\line(0,1){5}}
\end{picture}

On the fly changes are possible, for instance
a\allowbreak b
\ChangeElementName{allowbreak}{AllowBreak}
a\allowbreak b

You can change attributes also
\begin{quote}B\end{quote}
\ChangeElementName*{rend}{rendering}
\ChangeElementName*{quote}{quotation}
\begin{quote}B\end{quote}

\section[Section 1]{Section two}
\subsection[Subsetion 2]{Subsection two}

% Wagner test
A \begin{table} $x$ \end{table}
B \begin{table} \begin{tabular}{c} x \end{tabular} \end{table}
C \begin{tabular}{c} x \end{tabular} 
A \begin{table}[ht] $x$ \end{table}
B \begin{table*} \begin{tabular}{c} x \end{tabular} \end{table*}
C \begin{tabular}{c} x \end{tabular} 
\begin{figure} text \end{figure}
\begin{figure}[!h] \includegraphics{x.ps} \end{figure}
\begin{figure}[!h] \includegraphics{x.ps}\hspace{2cm}\includegraphics{x.ps} \end{figure}
\begin{figure}[!h] \includegraphics{x.ps} text\includegraphics{x.ps} \end{figure}


Same test as in torture file
\let\IC=\includegraphics
\def\FILE{Logo-INRIA-couleur}
\IC[angle=0,width=3cm,clip=]{\FILE}
\IC[angle=20,width=.5\textwidth,height=.3\textheight]{\FILE.ducon}
\IC[angle=0,width=\columnwidth,height=\textheight]{\FILE.foo_bar}
{\language=1 a:c
\IC[angle=0, =foo,,width=3cm,scale=1,scale=2,clip]{../../a_b:c}
}


%% More table tests

\begin{tabular*}{3cm}[t]{|ll|rr|cc|}
\hlinee{}{}{}{3pt} line1&b&c&d&e&f\\ 
\hlinee{}{}{}{6pt} \hlinee{}{}{}{7pt} line2&b&c&d&e&f\\ % 7wins
A&\multicolumn{3}{+}{C}&E&F\\ \hlinee{}{}{}{4pt} 
\multicolumn{2}{|l}{line4}&c&d&e&f\\[1cm]
\cline{1-3}\cline{6-6}\\\hlinee{}{}{}{6pt}%
\end{tabular*}


\begin{tabular*}{3cm}[t]{|ll|rr|cc|}
\hlinee{}{}{}{3pt} line1&b&c&d&e&f\\ 
\hlinee{}{}{}{6pt} \hlinee{}{}{}{7pt} line2&b&c&d&e&f\\ % 7wins
A&\multicolumn{3}{+}{C}&E&F\\ \hlinee{}{}{}{4pt} 
\multicolumn{2}{|l}{line4}&c&d&e&f\\[1cm]
\hlinee{1-3}{}{}{13pt}\hlinee{6-6}{}{}{66pt}\\\hlinee{}{}{}{6pt}%
\end{tabular*}


\begin{tabular*}{3cm}[t]{|ll|rr|cc|}
\hlinee{}{1.1pt}{1.2pt}{3pt} line1&b&c&d&e&f\\ 
\hlinee{}{1.5pt}{}{6pt} \hlinee{}{}{1.6pt}{7pt} line2&b&c&d&e&f\\ % 7wins on line1
A&\multicolumn{3}{+}{C}&E&F\\ \hlinee{}{}{}{4pt} 
\multicolumn{2}{|l}{line4}&c&d&e&f\\[1cm]
\hlinee{1-3}{1.8pt}{}{13pt}\hlinee{6-6}{}{1.9pt}{66pt}\\\hlinee{}{}{}{6pt}%
\end{tabular*}


\end{document}

%%% Local Variables: 
%%% mode: latex
%%% TeX-master: t
%%% End: 

