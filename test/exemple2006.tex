% Example of an activity report, much simplified

\documentclass{ra2006}
\usepackage{amsfonts}
\usepackage{amsmath}
\usepackage{subfigure}
\usepackage{array}
\let\href\htmladdnormallink
\theme{num}
\isproject{Yes}
\projet{exemple}{Apics}{Analysis and Problems of Inverse type in Control and Signal processing}
\UR{\URSophia}

\def\reachable{controllable} % ok ?
\def\corresp{manager}
\def\CC{{\mathbb C}}

\declaretopic{A}{Topic one}
\declaretopic{B}{Topic two}
\declaretopic{C}{Topic three}

\begin{document}

\begin{filecontents+}{exemple\_all2006.bib}
@Book{anap,
  editor =       {Fournier, J.-D. and Grimm, J. and Leblond, J. and
 Partington, J. R.},
  title =        {Harmonic Analysis and Rational Approximation;
Their R�les in Signals, Control and Dynamical Systems},
  publisher =    {Springer Verlag},
  year =         {2006},
  volume =       {327},
  series =       {Lecture Notes in Control and Information Sciences},
  isbn =         {3-540-30922-5}
}

@PhdThesis{th-david,
  author =       {Avanessoff, David},
  title =        {Lin�arisation dynamique des syst�mes non lin�aires et
                  param�trage de l'ensemble des solutions},
  school =       {Univ. de Nice - Sophia Antipolis},
  year =         2006,
  month =        jun,
 }
@InProceedings{ref1,
  author =       { Charlot, Gr�goire},
  title =        {Stability of Nonlinear switched systems in the plane},
  booktitle =    {44th IEEE Conf. },
  year =         2005,
  pages={3285--3290},
  address =      {Seville, Spain},
  month =        dec
}
@InProceedings{ref2,
  author =       {Boscain, Ugo},
  title =        {Stability of Non},
  year =         2006,
  pages={3285--3290},
  address =      {Seville, Spain},
  month =        dec
}
@InProceedings{ref3,
  author =       {Sigalotti, Mario},
  title =        {Stability of Nonlinear switched systems in the plane},
  booktitle =    {44th IEEE Conf. },
  year =         2005,
  pages={3285--3290},
  address =      {Seville, Spain},
  month =        dec
}

\end{filecontents+}

\maketitle
\nocite{*}

\begin{module}{composition}{en-tete}{}

\begin{catperso}{Team Leader}
\pers{Laurent}{Baratchart}{Scientist}{Inria}[DR Inria]
\end{catperso}

% 
\begin{catperso}{Deputy Team Leader}
\pers{Jean-Baptiste}{Pomet}{Scientist}{Inria}[CR Inria]
\end{catperso}

\begin{catperso}{Administrative Assistant}
\pers{France}{Limouzis}{Assistant}{Inria}[AI Inria, partial time in the team]
\end{catperso}

\begin{catperso}{Staff Member}
\pers{Jos�}{Grimm}{Scientist}{Inria}[CR Inria]
\pers{Juliette}{Leblond}{Scientist}{Inria}[DR Inria (since September, CR INRIA b
efore)]
\pers{Martine}{Olivi}{Scientist}{Inria}[CR Inria]
\pers{Fabien}{Seyfert}{Scientist}{Inria}[CR Inria]
\end{catperso}

\end{module}


\begin{module}{presentation}{presentation}{}
\begin{moreinfo}
The Apics Team is a Project Team since January 2005. 
\end{moreinfo}


The Team develops constructive methods for modeling, identification and 
control of dynamical systems.

\subsubsection*{Research Themes}
\begin{itemize}
\item Meromorphic approximation in the complex domain.
\item Inverse potential problems in 3-D and analysis of harmonic fields.
\item Control and structure analysis of non-linear systems.
 \end{itemize}

\subsubsection*{International and industrial partners}

\begin{itemize}
\item Industrial collaborations with Alcatel-Alenia-Space.
\item Exchanges with UST (Villeneuve d'Asq).
\item The project is involved in a NATO Collaborative Linkage Grant (with.
\end{itemize}
\end{module}


\begin{module}{fondements}{identif}{Identification and deconvolution}
Let us first introduce the subject of Identification in some generality.
Let us turn to work of the Apics Team\footnote{and of the former
  MIAOU-project}  can be partly recast from the data.

We shall explain in more detail the above steps in the sub-paragraphs
to come. 

\subsubsection{Analytic approximation of incomplete boundary data}
\label{dida-mero}
\begin{participants}
\pers{Laurent}{Baratchart},
\pers{Jos�}{Grimm},
\pers{Juliette}{Leblond},
\pers{Jean-Paul}{Marmorat}[CMA, �cole des Mines],
\pers{Jonathan}{Partington},
\pers{Fabien}{Seyfert}
\end{participants}

\begin{motscle}
meromorphic approximation, frequency-domain identification,
extremal problems, {alg�bre �l�mentaire $(\max,+)$}  
\end{motscle}

A prototypical Problem is:

{\sl ($P$)~~Let $p \geq 1$, $N \geq 0$, $K$ be an arc of the unit circle $T$, 
  $f \in L^p(K)$, $\psi \in L^p(T \setminus K)$ and $M>0$;
  find a function  $g \in H^p + R_N$ such that 
  $\|g - \psi\|_{L^p(T \setminus K)} \leq M$ and such that $g - f$ 
  is of minimal norm in  $L^p(K)$ under this constraint.}

In order to impose pointwise constraints in the frequency domain
 one may wish to express
the gauge constraint on $T\setminus K$ in a more subtle manner, depending on
the frequency:

{\sl ($P'$)~~Let $p \geq 1$, $N \geq 0$, $K$ be an arc of the unit circle
$T$, $f \in L^p (K)$, $\psi \in L^p(T \setminus K)$ and
$M \in L^p(T \setminus K)$;
find a function 
$g \in H^p + R_N$ such that $|g - \psi|\leq M$ a.e.\ on 
$T \setminus K$ and such that  $g - f$ is of minimal norm in
$L^p(K)$ under this constraint.}

Deeply linked with Problem $(P)$,  is the following completion Problem:


{\sl ($P''$)~~Let
$p \geq 1$, $N \geq 0$, $K$ an arc of the unit circle $T$, 
$f \in L^p(K)$,  $\psi \in L^p(T \setminus K)$ and
$M >0$; find a function  
$h \in L^p(T \setminus K)$ such that 
$\|h - \psi\|_{L^p(T\setminus K)} \leq M$, and such that the distance to 
$H^p + R_N$ of the concatenated function $f\vee h$
is minimal in $L^p(T)$ under this constraint.}

A version of this problem where the constraint depends on the frequency is:

{\sl ($P'''$)~~Let $p \geq 1$,
$N \geq 0$, $K$ an arc the unit circle $T$, $f \in L^p(K)$,  
$\psi \in L^p(T \setminus K)$ and
$M \in L^p(T \setminus K)$; find a function  $h \in L^p(T
\setminus K)$ such that 
$|h - \psi|\leq M$ a.e.\ on $T \setminus K$, and such that the distance to
$H^p + R_N$ of the concatenated function $f\vee h$
is minimal in $L^p(T)$ under this constraint.}

Let us mention that Problem $(P'')$ reduces to Problem 
$(P)$ that in turn reduces, although implicitly, 
to an extremal Problem without
constraint, (i.e., a Problem of type  $(P)$ where $K=T$) that is
denoted conventionally by  $(P_0)$. In the case where $p=\infty$, 
Problems  $(P')$ and $(P''')$ can viewed as special cases of 
$(P)$ and $(P'')$ respectively, but if  $p<\infty$ the situation
is different.

where the constraint on the approximant is expressed in terms of
its real and imaginary parts while the criterion takes only its real part 
into account:

{\sl Let $p \geq 1$, $K$ be an arc of the unit circle $T$, 
  $f \in L^p(K)$, $\psi \in L^p(T \setminus K)$, and $\alpha, \beta, M>0$;
  find a function  $g \in H^p$ such that 
  $\alpha \, \|\mbox{\rm Re} ({g - \psi})\|_{L^p(T \setminus K)} + \beta
\, \|\mbox{\rm Im} ({g - \psi})\|_{L^p(T \setminus K)} \leq M$ and
such that $\mbox{\rm Re} (g - f)$    is of minimal norm in  $L^p(K)$
under this constraint.} 

see sections \moduleref{APICS}{domaine}{dom-fissures} and
\moduleref{APICS}{resultats}{fissures}, where data and physical prior
information concern  real (or imaginary)  parts of analytic functions.

This allows one to 
\begin{enumerate}
\item extend  the index theorem to the case $2\leq p\leq\infty$
\item study asymptotic errors with
\item characterize the asymptotic (cf. section \ref{didactique-poles}). 
\end{enumerate}
In connection with the second and third items above,see section \ref{AHH}.

\subsubsection{Scalar rational approximation}
\label{didactique-approx-rat-scal}
\begin{participants}
\pers{Laurent}{Baratchart},
\pers{Martine}{Olivi},
\pers{Edward}{Saff},
\pers{Herbert}{Stahl}[TFH Berlin],
\pers{Maxim}{Yattselev}
\end{participants}

\begin{motscle}
rational approximation, critical point, orthogonal polynomials
\end{motscle}
Rational approximation is the second step mentioned
in section~\moduleref{APICS}{fondements}{identif}. 
The Problem can be stated as:

{\sl Let $1\leq p\leq\infty$, $f\in H^p$ and $n$ an integer; 
find a rational function without poles in the unit disk, and of
degree at most $n$ that is nearest possible to $f$ in  $H^p$.}
In this way we are led to consider
minimizing a criterion of the form:
\begin{equation}
\label{crit}
\left\|f - \frac{p_m}{q_n} \right\|_{L^2(d \mu)} 
\end{equation}
where, by definition, 
\[
\|g\|_{L^2(d \mu)}^2=\frac{1}{2\pi} \int_{-\pi}^{\pi}|g(e^{i\theta})|^2
d\mu(\theta),
\]


\[
\min \left\||f| - \left|\frac{p_n}{q_n}\right| \right\|_{L^p(T)}.
\]

\[
\left\||f|^2 - \left|\frac{p_n}{q_n}\right|^2 \right\|_{L^\infty(T)}
<\varepsilon,
\]

\paragraph{OK}
\label{didactique-approx-rat-mat}
If one introduces now as a new variable the rational matrix $R$ defined by
\[
R=\left(\begin{array}{cc}
L            &  H \\
0            &   I_m
\end{array} \right)^{-1}
\]
and if $T$ stands for the first block-row, 

\begin{equation}
\label{defLL}
\|T\|_{\Lambda}^2={\bf Tr}\left\{\frac{1}{2\pi}
\int_{0}^{2\pi}T(e^{i\theta})\,
d\Lambda(\theta)\,T^*(e^{i\theta})\right\},
\end{equation}
\end{module}

\begin{module}{fondements}{nl}{Structure and control of non-linear systems}

In order to control a system, one generally relies on a model.

\subsubsection{Feedback control and optimal control}
\label{nl-stab}

Stabilization by continuous state feedback---or output feedback, that is,


\subsubsection{Transformations and equivalences of non-linear systems and models} 
\label{nl-trans}


\paragraph{Dynamic linearization.}
The problem of dynamic linearization,.

\paragraph{Topological Equivalence}

In what precedes, we have not taken into account the degree of
\emph{smoothness} of the transformations under consideration.\end{module}

\begin{module}{domaine}{chapeau}{Introduction}
The botton line of the team's activity is twofold,
\end{module}

\begin{module}{domaine}{dom-fissures}{Geometric inverse problems 
for the Laplacian}

Localizing cracks, pointwise sources or occlusions in a two-dimensional.



\end{module}


\begin{module}{domaine}{resonn}{Identification and design of resonant systems}

One of the best training ground for the research of the team in function
theory is the identification and design of physical systems. 

\begin{figure}
\begin{center}
\includegraphics{miaou_transf}
\end{center}
\caption{Transducer model.}
\label{trans}
\end{figure}

\begin{figure}
\begin{center}
\includegraphics{miaou_coup}
\end{center}
\caption{Configuration of the filter}
\label{filtrescnes}
\end{figure}\end{module}

\begin{module}{domaine}{spatial}{Spatial mechanics}
The use of satellites in telecommunication networks motivates.
\end{module}


\begin{module}{domaine}{optique}{Non-linear optics}
The increased capacity of numerical channels in information
technology is a major
industrial challenge.
\end{module}


\begin{module}{domaine}{plat}{Transformations and equivalence of non-linear systems}


The works presented in module~\ref{nl-trans} lie upstream.
\end{module}


\begin{module}{logiciels}{logi-tralics}{The Tralics software}
\label{RARL2}
\begin{participant}
\pers{Jos�}{Grimm}[\corresp]
\end{participant}
The development of a \LaTeX\ to XML translator, named Tralics was continued.
\end{module}

\begin{module}{logiciels}{RARL2}{The RARL2 software}
\begin{participants}
\pers{Jean-Paul}{Marmorat},
\pers{Martine}{Olivi}[\corresp]
\end{participants}
 \label{didactique-poles}

RARL2 (R�alisation interne et Approximation Rationnelle L2) is a software for
rational approximation (see module \ref{didactique-approx-rat-mat}). Its web
page is
\htmladdnormallink{\url{http://www-sop.inria.fr/miaou/RARL2/rarl2.html}}
{http://www-sop.inria.fr/miaou/RARL2/rarl2.html}.
It is germane to the arl2 function of hyperion
\end{module}

\begin{module}{logiciels}{RGC}{The RGC software}
The identification of  filters modeled 
see section~\moduleref{APICS}{resultats}{Couplages-Algebrique}.
\end{module}

\begin{module}{logiciels}{PRESTO-HF}{PRESTO-HF}
\begin{participant}
\pers{Fabien}{Seyfert}
\end{participant}

PRESTO-HF: a toolbox dedicated to lowpass parameter identification for
hyperfrequency filters
\htmladdnormallink{\url{http://www-sop.inria.fr/miaou/Fabien.Seyfert/Presto_web_page/presto_pres.html}}
{http://www-sop.inria.fr/miaou/Fabien.Seyfert/Presto_web_page/presto_pres.html}
The `miaou' should be replaced by `apics' here.
\end{module}


\begin{module}{logiciels}{logi-endymion}{The Endymion software}
\label{endymion}
\begin{participant}
\pers{Jos�}{Grimm}[\corresp]
\end{participant}
We have started the development of Endymion, a software licensed
under the CeCILL license version two, see
\href{http://www.cecill.info}{http://www.cecill.info/}.
\end{module}

\begin{module}[A]{resultats}{RAjose}{Tools for producing the Activity Report
    (this document)}

The great novelty in the RAWEB2002 (Scientific Annex to the Annual Activity
Report of Inria), was the use of XML as intermediate language, and the
possibility of bypassing \LaTeX.  for the example we get
\verb+${\#119987 _y=lim_{x\#8594 0}sin^2{(x)}}$+.

\end{module}

\begin{module}[A]{resultats}{tralics}{Tralics: a Latex to XML Translator}
The \textit{Tralics} software is a C++ written \LaTeX\ to XML translator

\end{module}

\begin{module}[B]{resultats}{fissures}{Inverse Problems for 2D and 3D
  elliptic operators}

\subsubsection{Sources recovery in 2D and 3D}
\label{AHH}
The fact that 2D harmonic functions are real parts is also considered.

\subsubsection{Application to EEG inverse problems}

In 3D, epileptic regions in the cortex are often 
linked  to a number of important related issues.


\subsubsection{Cauchy problems in 2D and 3D}

Solving Cauchy problems on an annulus can be extended.

\subsubsection{More general geometries}

We also started  to be developed.

\subsubsection{Others elliptic operators}
Within the post-doctoral stay of E. Sincich, we began
the University of Nice.

\subsubsection{Application to  magnetic dipoles recovery}

The magnetic field produced by a magnetic dipole $\vec {m}$
located at a point ${\vec r}'$ is
\begin{equation}
\vec B(\vec r) = \frac{\mu_0}{4\pi}\left\{
\frac{3\vec m({\vec r}')\cdot(\vec r-{\vec r}')}{|\vec r-{\vec r}'|^5}
(\vec r-{\vec r}') - \frac{\vec m({\vec r}')}{|\vec r-{\vec r}'|^3}
\right\}. \label{W102}
\end{equation}

\[ 
B_z(x,y,z) = \frac{\mu_0}{4\pi}\lambda_k
\frac{2z^2-(x-x_k)^2-(y-y_k)^2}{[(x-x_k)^2+(y-y_k)^2+z^2]^{5/2}}
\]

\begin{equation}
C_z(x,y,z) = \frac{\mu_0}{4\pi^2a^2}\sum_k\lambda_k
\int_{D(0,a)}
\frac{2z^2-(x-\alpha-x_k)^2-(y-\beta-y_k)^2}
{\left[(x-\alpha-x_k)^2+(y-\beta-y_k)^2+z^2\right]^{5/2}} d\alpha d\beta.
\label{equa3}
\end{equation}
\end{module}


\begin{module}[C]{resultats}{martine1}{Parametrizations  of matrix-valued
    lossless functions} 
\label{Schur-realisations}
 The possibility to fertilize the pure algebraic LMI approach
with the rich and vast topic of Schur analysis has been pointed out and
deserve to be further investigated. 
\end{module}

\begin{module}[C]{resultats}{martine2}{The mathematics of Surface Acoustic Wave
    filters} 
\end{module}


\begin{module}[B]{resultats}{jul2}{Rational and Meromorphic Approximation}
The results  have been exploited .
\end{module}


\begin{module}[B]{resultats}{poles}{Behavior of poles}
\label{Rpoles}

This rather unexpected algorithm is currently being explored in 
details by E. Mina.
\end{module}


\begin{module}[C]{resultats}{ExtFab}{Analytic extension under pointwise constraints}
Such regularity condition should greatly impinge on the
numerical practice of the problem.
\end{module}

\begin{module}[B]{resultats}{Couplages-Algebrique}{Exhaustive determination
of constrained realizations corresponding to a transfer function}

\begin{eqnarray}
\label{ep}
p \in \CC^{r_1},\,\, E_{\sigma_1}(p)&=\{q \in \CC^{r_1},
\pi_{\sigma_1}(q)=\pi_{\sigma_1}(p)\} \\
p \in \CC^{r_2},\,\, E_{\sigma_1,\sigma_2}(p)&=\{q \in \CC^{r_1},
\pi_{\sigma_1}(q)=\pi_{\sigma_2}(p)\}
\end{eqnarray}
\end{module}

\begin{module}[A]{resultats}{synthese}{Zolotarev problem and multi-band filter design}

\label{zolo}
T

\end{module}


\begin{module}[B]{resultats}{omux}{Frequency Approximation and OMUX design}
\label{secOMUXc}
This is one reason for analysing the optimization problem further.
\end{module}



\begin{module}[A]{resultats}{mario-optim}{On the structure of optimal trajectories}

The results on the local regularity of trajectories in optimal control
obtained previously have been published.
\end{module}

\begin{module}[A]{resultats}{alex1}{Feedback for low thrust orbital transfer}
\label{secorbitet}
   The study concerns the control of a satellite.
\end{module}


\begin{module}{resultats}{flat}{Local linearization (or flatness) of control systems}

 a workable formulation of the question is now available.
\end{module}

\begin{module}{resultats}{dubbins}{Controllability for a general Dubins 
problem}
Controllability results for systems with drift are usually obtained.

\smallskip

The main object of our research  is given by Dubins-like systems 
In \footcite{ref1} we proved that $0\ne1$. This has been presented in 
\cite{ref2}  and \refercite{ref3}.
\end{module}



\begin{module}{contrats}{cnes}{Contracts CNES-IRCOM-INRIA}
Contracts \no 04/CNES/1728/00-DCT094
 see module \ref{filtrescnes},
\end{module}



\begin{module}{contrats}{aspi-c}{Contract Alcatel Space (Cannes)}
Contract \no 1 01 E 0726.
\end{module}

\begin{module}{international}{nat}{Scientific Committees}
L. Baratchart is a member of the editorial board
\end{module}

\begin{module}{international}{nationale}{National Actions}
Together with project-teams Caiman and Odyss�e
\end{module}

\begin{module}{international}{cee}{Actions  Funded by the EC}
The team is the recipient see
\htmladdnormallink{\url{http://www.ladseb.pd.cnr.it/control/ercim/control.html}}{http://www.ladseb.pd.cnr.it/control/ercim/control.html}.
\end{module}

\begin{module}{international}{monde}{Extra-european International Actions}
\textbf{NATO CLG} (Collaborative Linkage Grant), PST.CLG.979703, 
``Constructive approximation and inverse diffusion problems'', with
Vanderbilt Univ. (Nashville, USA) and LAMSIN-ENIT (Tunis, Tu.), 2003-2005.

\textbf{EPSRC}  grant (EP/C004418) ``Constrained approximation in
function spaces, with applications'', with Leeds Univ. (UK) and
Univ. Lyon I, 2005-2006.

\textbf{STIC-INRIA} and \textbf{AireD�veloppement} grants with
LAMSIN-ENIT (Tunis, Tu.), ``Probl�mes 
inverses du Laplacien et approximation constructive des fonctions'',

\textbf{NSF EMS21} RTG students exchange program (with
Vanderbilt University).
\end{module}



\begin{module}{international}{accueilx}{The Apics Seminar}

The following scientists gave a talk at the seminar:

\end{module}





\begin{module}{diffusion}{dif-ens}{Teaching}

\begin{description}
\item [Courses] \ 
  \begin{itemize}
  \item L. Baratchart, DEA G�om�trie et Analyse, LATP-CMI, Univ. de Provence
  \item M. Olivi, Math�matiques pour l'ing�nieur 
  \end{itemize}
  
\item [Trainees] \ 
  \begin{itemize}
  \item  Jonathan Chetboun (ENPC)
  \item Cristina Paduret
    \textit{R�solution de probl�mes inverses}
  \end{itemize}

\item[Ph.D. Students] \ 
  \begin{itemize}
  \item Alex Bombrun, �~Commande optimale de satellites~� (optimal etc)
  \item Imen Fellah, ``Data completion in inverse problems'',  co-tutelle.
\item Vincent Lunot,  �~Probl�mes  � la synth�se d'OMUX �, 
\item Moncef Mahjoub,    ``Compl�tion de donn�es g�om�triques.'' co-tutelle.
\item  Erwin Mina Diaz, ``Asymptotic properties of urves.''
\item  Maxim Yattselev, ``Meromorphic orthogonality.''

\end{itemize}
\item[defended Ph.D. thesis] \ 
 \begin{itemize}
  \item David Avanessoff, �~Lin�arisation dynamique des solutions~� (dynamic
 trajectories). June 8, 2005.\cite{th-david}
\end{itemize}
\item[Jurys] \ 
%
\item L.~Baratchart sat on 
\item J.~Leblond has been sitting
\item F.~Seyfert has been sitting 
\item J.-B. Pomet has been sitting 
\end{description}


\end{module}

\begin{module}{diffusion}{dif-anim}{Community service}

L. Baratchart was a member of the  ``bureau'' of the CP
(Comit� des Projets) of INRIA-Sophia Antipolis untill July.
He is a member of the ``commission de sp�cialistes'' (section 25) of the
Universit� de Provence.

J.~Leblond and J. Grimm are co-editors of  the
proceedings (to appear in 2006) of the 
CNRS-INRIA summer school ``Harmonic analysis and rational approximation: their
r\^oles in signals, control and dynamical systems theory''
(Porquerolles, 2003)
\htmladdnormallink{\url{http://www-sop.inria.fr/apics/anap03/index.en.html}}
{http://www-sop.inria.fr/apics/anap03/index.en.html} \footcite{anap}. 
\end{module}

\begin{module}{diffusion}{dif-conf}{Conferences and workshops}

A. Bombrun, B. Atfeh and L. Baratchart have presented a communication at 
CMFT2005 (Computational Methods and Function
Theory), Joensuu, Finland (June).
 \refercite{th-david}
\end{module}

\loadbiblio
\end{document}

