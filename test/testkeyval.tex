\documentclass[Cu,Cv,Cw,foo=E,opF={\foobar,gee},unused,Unused=U]{article}
\usepackage{checkend}
\ChangeElementName{warning}{Warning}

%%% This tests the special filecontents environment
\begin{filecontents+}{testkeyval.plt}
\ProvidesPackage{testkeyval}{2008/02/15 Package for testing keyval}

\DeclareOptionX{landscape}{\landscapefalse}

\DeclareOptionX{keya}{%
\edef\vara{prefix=\XKV@prefix, fams=\XKV@fams, this fam=\XKV@tfam, 
header=\XKV@header,this key=\XKV@tkey, na=\XKV@na,val=#1}}
\DeclareOptionX{keyb}{%
\edef\varb{prefix=\XKV@prefix, fams=\XKV@fams, this fam=\XKV@tfam, 
header=\XKV@header,this key=\XKV@tkey, na=\XKV@na,val=#1}}

\DeclareOptionX{opA}{\def\opA{#1}}
\DeclareOptionX{opB}{\def\opB{#1}}
\DeclareOptionX{opC}{\def\opC{#1}}
\DeclareOptionX{opD}{\def\opD{#1}}
\DeclareOptionX{opF}{\def\opF{#1}}
\DeclareOptionX{Cw}{}
\let\opA\relax 
\let\opB\relax 
\let\opC\relax 
\let\opD\relax 
\let\opF\relax 

\def\unknownoptions{}

\ExecuteOptionsX{keya,keyb=1}

\DeclareOptionX*{%
  \typeout{xkeyval: Unknown option `\CurrentOption'}%
  \edef\unknownoptions{\unknownoptions\CurrentOption.}
}

\ProcessOptionsX \relax
\end{filecontents+}

\begin{filecontents+}{options.plt}
\ProvidesPackage{options}[2008/02/29 Package for testing package]
\newcounter{xyz}
\DeclareOption{foo=A}{\addtocounter{xyz}{1}}
\DeclareOption{foo=B}{\addtocounter{xyz}{10}}
\DeclareOption{foo=C}{\addtocounter{xyz}{100}}
\DeclareOption{foo=D}{\addtocounter{xyz}{1000}}
\DeclareOption{foo=E}{\addtocounter{xyz}{10000}}
\ExecuteOptions{foo=A,foo=D}
\ProcessOptions \relax
\end{filecontents+}


%% See comment below about encoding
\csname input@encoding\endcsname=1
\begin{filecontents+}{testclass.plt}
% This file is a model
\ProvidesPackage{testclass}[2008/02/19 Package for testing package]

%We assume \unknownoptions set by another package
\def\addfoo#1{\edef\unknownoptions{\unknownoptions#1}}

\AtEndOfPackage{\addfoo{x}}
\AtEndOfPackage{\addfoo{y}}

\@ifclassloaded{article}{}{\addfoo{badclass.}}
\@ifpackageloaded{testkeyval}{}{\addfoo{badpack.}}
\@ifpackageloaded{article}{\addfoo{badclass2.}}{}
\@ifclassloaded{testkeyval}{\addfoo{badpack2.}}{}

\@ifclasslater{article}{2000/00/00}{}{\addfoo{badclass3.}}
\@ifclasslater{article}{2020/00/00}{\addfoo{badclass4.}}{}
\@ifpackagelater{testclass}{2008/00/00}{}{\addfoo{badclass5.}}
\@ifpackagelater{xkeyval}{2020/00/00}{\addfoo{badpack3.}}{}
\@ifpackagelater{xkeyval}{2008/00/00}{}{\addfoo{badpack4.}}
\@ifpackagelater{testclass}{\the\numexpr\the\year+10\relax/00/00}{\addfoo{badpack6a.}}{}
\@ifpackagelater{testclass}{\the\numexpr\the\year-10\relax/00/00}{}{\addfoo{badpack5a.}}
\@ifpackagewith{testclass}{La,Lb,Lc}{}{\addfoo{badpack5.}}
\@ifpackagewith{testclass}{Lc,Lb,La}{}{\addfoo{badpack6.}}
\@ifpackagewith{testclass}{Ld}{\addfoo{badpack7.}}{}
%%\@ifpackagewith{testkeyval}{opW=5}{}{\addfoo{badpack9.}}
\@ifpackagewith{testkeyval}{opW=6}{\addfoo{badpack10.}}{}
\@ifclasswith{article}{Cuu}{\addfoo{badclass8.}}{}
\@ifclasswith{article}{Cu}{}{\addfoo{badclass7.}}

%% Note about encodings: this file is read in latin2
%% is saved as UTF8, so the stream coontains José. 

\def\NameA{Jos�}% 
\input@encoding=1
\def\NameB{Jos�}% 
\def\xNameA{Jos^^e9}% 
\def\xNameB{Jos^^c3^^a9}% 
\ifx\NameA\xNameA\else \bad \fi
\ifx\NameB\xNameB\else \bad \fi
% Next line should produce an error. the line number should be
% that of the real tex file, not the virtual testclass.plt
\ifx\NameA\NameB\else\Error \fi

\DeclareOption{Lc}{\addfoo{Lc}}
\DeclareOption{La}{\addfoo{La}}
\DeclareOption{Cv}{\addfoo{Cv}}
\DeclareOption{Cu}{\addfoo{Cu}}

\DeclareOption*{% 
 \addfoo{D=\CurrentOption.}
}

\DeclareOption
\ExecuteOptions{keya,keyb=1}

\ProcessOptions* \relax
\AtEndOfPackage{\addfoo{z}}
\end{filecontents+}

\begin{filecontents+}{testkv.bib}
@Article{louarn88,
  author =       {Philippe Louarn},
  title =        {Une exp\'erience d'utilisation de {LaTeX} : le {R}apport d'act
ivit\'e de {l'INRIA}},
  journal =      {Cahiers Gutenberg},
  year =         {1988},
  number =       {0},
  pages =        {17-24},
  month =        {apr},
  OPTnote =      {},
  annote =       {http://www.gutenberg.eu.org/publications/cahiers/25-cahiers0.h
tml}
}
\end{filecontents+}

%% Is the usepackage working ?
\usepackage{xkeyval}
\usepackage[opA,opB=C,epW=5,opC=\foo,opE]{testkeyval}
\usepackage[La,Lb,Lc]{testclass}
\usepackage[foo=A,foo=C]{options}
\begin{document}


\def\fooA{a
b}
\def\fooB{a b}
\ifx\fooA\fooB\else\bad\fi

\makeatletter
\newif\ifOK
\def\cp@store#1#2#3#4\@nnil{\def#1{#3}\def#2{#4}}
\def\cp@split#1#2{\expandafter\cp@store\expandafter#1\expandafter#2#2\@nnil}
\def\cp@compare#1#2{\cp@split\fooA#1\cp@split\fooB#2\ifx\fooA\fooB\OKtrue\else\OKfalse\fi
\ifx#1#2\typeout{SAME}\show#1 \OKfalse\fi
}






\newcommand{\makeCommandwithOption}[1]
{
\define@choicekey+[Prefix]{#1}{CommandStyle}%
[\expandafter\csname Prefix@#1@CommandStyle\endcsname]
 {none, bold, italic, roman}
 {
   % CommandStyle: ##1, \csname Prefix@#1@CommandStyle\endcsname. \\
  }{
    \PackageError{Prefix@#1@CommandStyle}{Invalid CommandStyle Option}
%
    {*CommandStyle can only be `none', `bold', `italic',`roman'}%
  }
}
\makeCommandwithOption{myCommand}

% The tester function
\def\isfoo#1{\ifx\foo#1\else\show#1\show\foo\bad\fi}
\def\ISfoo#1{\def\tmp{#1}\ifx\foo\tmp\else\show\tmp\show\foo\bad\fi}
\makeatletter

%% Command set by the package
\unknownoptions


\define@key{fam}{keyA}{}\ifcsname KV@fam@keyA\endcsname\else \bad\fi
\define@key[]{xx}{keyA}{}\ifcsname xx@keyA\endcsname\else \bad\fi
\define@key{}{keyA}{}\ifcsname KV@keyA\endcsname\else \bad\fi
\define@key[]{}{keyA}{}\ifcsname keyA\endcsname\else \bad\fi
\define@key[my]{fam}{keyA}[]{}\ifcsname my@fam@keyA\endcsname\else \bad\fi
\ifcsname my@fam@keyA@default\endcsname\else \bad\fi
\define@key[my]{}{keyA}{}\ifcsname my@keyA\endcsname\else \bad\fi
\ifcsname my@keyA@default\endcsname\bad\fi

\define@key{family}{keyA}{The input is #1}
\define@key{family}{keyB}[none]{The input is #1}
\def\foo#1{The input is #1} \isfoo\KV@family@keyA
\def\foo{\KV@family@keyB{none}} \isfoo\KV@family@keyB@default
\define@key{family}{keyB}[none]{\edef\bar{#1}}
\KV@family@keyB{toto}
\def\foo{toto}\isfoo\bar
\define@key{family}{keyB}[none]{\edef\bar{#1}}
\KV@family@keyB@default
\def\foo{none}\isfoo\bar



\define@cmdkey[xKV]{fam}{keyA}[x]{code #1}
\def\foo{\xKV@fam@keyA {x}}\isfoo\xKV@fam@keyA@default
\def\foo#1{\def \cmdxKV@fam@keyA {#1}code #1}\isfoo\xKV@fam@keyA
\define@cmdkey[xKV]{fam}{keyA}[y]{\edef\bar{x#1}}
\xKV@fam@keyA{toto}
\def\foo{toto}\isfoo\cmdxKV@fam@keyA
\def\foo{xtoto}\isfoo\bar
\xKV@fam@keyA@default
\def\foo{y}\isfoo\cmdxKV@fam@keyA
\def\foo{xy}\isfoo\bar


\define@cmdkey[xKV]{fam}[MP@]{keyB}[y]{code #1}
\def\foo#1{\def \MP@keyB {#1}code #1}\isfoo\xKV@fam@keyB
\def\foo{\xKV@fam@keyB {y}}\isfoo\xKV@fam@keyB@default
\define@cmdkey[xKV]{fam}[MP@]{keyC}{code #1}
\def\foo#1{\def \MP@keyC {#1}code #1}\isfoo\xKV@fam@keyC

\define@cmdkeys[xKV]{fam}[MP@]{keyD,keyE}[z]
\def\foo#1{\def \MP@keyD {#1}}\isfoo\xKV@fam@keyD
\def\foo#1{\def \MP@keyE {#1}}\isfoo\xKV@fam@keyE
\def\foo{\xKV@fam@keyE {z}}\isfoo\xKV@fam@keyE@default

\define@cmdkeys[xKV]{fam}[MP@]{keyF,keyG}
\def\foo#1{\def \MP@keyF {#1}}\isfoo\xKV@fam@keyF
\def\foo#1{\def \MP@keyG {#1}}\isfoo\xKV@fam@keyG
\let\foo\undefined\isfoo\xKV@fam@keyF@default

\define@choicekey*[KV]{fam}{keyC}[\val\nr]{left,center,right}[w]
{\ifcase\nr\relax\raggegright\or\centering\or\raggedleft\fi}
\def\foo{\KV@fam@keyC {w}}\isfoo\KV@fam@keyC@default

\define@choicekey*[KV]{fam}{keyC}[\val\nr]{a,b}{#1}
\def\foo#1{\XKV@cc*[\val \nr ]{#1}{a,b}{#1}}
\isfoo\KV@fam@keyC
\define@choicekey*[KV]{fam}{keyC}[\val\nr]{a,b}{\edef\bar{w#1}}
\KV@fam@keyC{A} 
\def\foo{a}\isfoo\val
\def\foo{0}\isfoo\nr
\def\foo{wA}\isfoo\bar
\KV@fam@keyC{B} 
\def\foo{b}\isfoo\val
\def\foo{1}\isfoo\nr
\def\foo{wB}\isfoo\bar

\define@choicekey+[KV]{fam}{keyC}[\val\nr]{a,b}{\edef\bar{w#1}}{\edef\bar{W#1}}
\KV@fam@keyC{A} 
\def\foo{A}\isfoo\val
\def\foo{-1}\isfoo\nr
\def\foo{WA}\isfoo\bar
\KV@fam@keyC{B} 
\def\foo{B}\isfoo\val
\def\foo{-1}\isfoo\nr
\def\foo{WB}\isfoo\bar

\define@choicekey[KV]{fam}{keyC}[\val\nr]{a,b}{#1}
\def\foo#1{\XKV@cc[\val \nr ]{#1}{a,b}{#1}}
\isfoo\KV@fam@keyC

\define@choicekey[KV]{fam}{keyC}{a,b}{#1}
\def\foo#1{\XKV@cc[]{#1}{a,b}{#1}}
\isfoo\KV@fam@keyC

\define@choicekey*+[KV]{fam}{keyC}[\val\nr]{a,b}{#1}{=#1}
\def\foo#1{\XKV@cc*+[\val \nr ]{#1}{a,b}{#1}{=#1}}
\isfoo\KV@fam@keyC

\define@boolkeys{fam}[my@]{frameA,frameB}
\define@boolkey{fam}[my@]{frame}{A#1}
\define@boolkey+{fam}{shadow}{B#1}{C#1}
\def\foo#1{\XKV@cc*[\XKV@resa ]{#1}{true,false}{\csname
    my@frame\XKV@resa \endcsname A#1}}
\isfoo\KV@fam@frame
\def\foo#1{\XKV@cc*[\XKV@resa ]{#1}{true,false}{\csname
    my@frameA\XKV@resa \endcsname}}
\isfoo\KV@fam@frameA

\def\foo#1{\XKV@cc*+[\XKV@resa ]{#1}{true,false}{\csname KV@fam@shadow\XKV@resa \endcsname B#1}{C#1}}
\isfoo\KV@fam@shadow           


\define@key{fam}{key}{%
I will first check your input, please wait.\\
\XKV@cc*+[\val]{#1}{true,false}{%
   The input \val\ was correct, we proceed.\\
}{%
   The input \val\ was incorrect and was ignored.\\
}%
  I finished the input check.
}


\KV@fam@key{True}
\KV@fam@key{NotTrue}


\define@boolkey{fam}{A}{\xdef\foo{\ifKV@fam@A Atrue\else Afalse\fi}}
\define@boolkeys{fam}{B,C}

\setkeys{fam}{A=true,B=false,C=True}
\def\Test{Atrue}
{
\ifx\foo\Test \ifKV@fam@B\else \ifKV@fam@C \let\bad\relax\fi\fi\fi
\bad}

\define@key{Fam}{keya}[ax]{au#1v}
\define@key{Fam}{keyb}[bx]{bu#1v}



\setkeys{Fam}{keya,keyb=3}
\setkeys{Fam}{\savevalue{keya},\gsavevalue{keyb}=3}
\setkeys{Fam}{ keya , keyb   = 43 , ,}

%% ----------------------------------------------------------------------
\define@key[my]{familya}{keya}{\def\foo{fam=A,key=A,value=#1}}
\define@key[my]{familya}{keyb}{\def\foo{fam=A,key=B,value=#1}}
\define@key[my]{familyb}{keyb}{\def\foo{fam=B,key=B,value=#1}}
\define@key[my]{familya}{keyc}{\def\foo{fam=A,key=C,value=#1}}

\define@boolkey{fam}{frame}

\define@key[my]{dfamilya}{keya}{fam=A,key=A,value=#1}
\define@key[my]{dfamilya}{keyb}{fam=A,key=A,value=#1}
\disable@keys[my]{dfamilya}{keya,keyb}

\key@ifundefined[my]{familya,familyb}{keya}{\bad}{}
\key@ifundefined[my]{xx,yy,familya,familyb}{keya}{\bad}{}
\def\expt{familya}\ifx\XKV@tfam\expt\else\bad\fi
\key@ifundefined[my]{xx,yy}{keya}{}{\bad}
\def\expt{yy}\ifx\XKV@tfam\expt\else\bad\fi
\key@ifundefined[my]{}{keya}{}{\bad}
\def\expt{}\ifx\XKV@tfam\expt\else\bad\fi

\setkeys[my]{dfamilya,dfamilyb}{keya=xxxxxxxxxxxxxx}

\define@key[X]{familya}{keya}{a#1a}
\define@key[X]{familyb}{keya}{b#1a}
\define@key[X]{familyb}{keyb}{b#1b}
\define@key[X]{familyc}{keye}{c#1e}
\define@key[X]{familyc}{keyf}{c#1f}
\define@key[X]{familyb}{keyc}[CV]{c#1c}

XXA\setkeys[X]{familya,familyb}{keya=A,keyb=B,keyd=D}
\setkeys[X]{familya}{keya=A}
\setkeys[X]{familyb}{keyb,keyc}
XXB\setkeys[X]{familyb}{keyc=a\setkeys[X]{familya}{keya=~and b},keyb=~and c}
\setkeys*[X]{familyb}{keyc,keyd,keye} \def\foo{keyd,keye}\isfoo\XKV@rm
\setkeys*[X]{familya,familyb}[keya,keyd]{keyc,keyd,keye=1, keyf= 2, keyg=3}
\def\foo{keye=1,keyf= 2,keyg=3}\isfoo\XKV@rm
XX\setrmkeys*[X]{familyd}C\setrmkeys[X]{familyc}[keyg]
\setkeys*+[X]{familyb}{keyc,keyd,keye} \def\foo{keyd,keye}\isfoo\XKV@rm
\setkeys*+[X]{familya,familyb}[keya,keyd]{keyc,keyd,keye=1, keyf= 2, keyg=3}
\def\foo{keye=1,keyf= 2,keyg=3}\isfoo\XKV@rm
XX\setrmkeys*+[X]{familyd}C\setrmkeys[X]{familyc}[keyg]
XXD\setkeys[X]{familya,familyb}{keya=A}E\setkeys+[X]{familya,familyb}{keya=A}

\setkeys[my]{familya,familyb}{keya=test}\ISfoo{fam=A,key=A,value=test}
\setkeys[my]{familya,familyb}{keyb=test}\ISfoo{fam=A,key=B,value=test}
\setkeys[my]{familyb,familya}{keyb=test}\ISfoo{fam=B,key=B,value=test}
\setkeys[my]{familyc,familya}{keyb=test}\ISfoo{fam=A,key=B,value=test}
\setkeys[my]{familya,familyb}{keya = test}\ISfoo{fam=A,key=A,value=test}
\setkeys[my]{familya,familyb}{keya = {{{test=x}}}}\ISfoo{fam=A,key=A,value={test=x}}

\define@key[X]{familya}{keyc}{%
\edef\var{prefix=\XKV@prefix, fams=\XKV@fams, this fam=\XKV@tfam, 
header=\XKV@header,this key=\XKV@tkey, na=\XKV@na}}
\setkeys*[X]{familya,familyb}[keya,keyd]{keyc=x,keyd,keye=1, keyf= 2, keyg=3}

\def\foo{prefix=X@, fams=familya,familyb, %
this fam=familya, header=X@familya@,this key=keyc, na=keya,keyd}
\isfoo\var

{
\setkeys[my]{familya}{\savevalue{keya}=test1}\ISfoo{fam=A,key=A,value=test1}
\def\foo{test1}\isfoo\XKV@my@familya@keya@value
\setkeys[my]{familya}{\gsavevalue{keya}=test2}\ISfoo{fam=A,key=A,value=test2}
\def\foo{test2}\isfoo\XKV@my@familya@keya@value
\isfoo\XKV@my@familya@keya@value
\setkeys[my]{familya}{\savevalue{keya}=test3}\ISfoo{fam=A,key=A,value=test3}
\def\foo{test3}\isfoo\XKV@my@familya@keya@value
}
\def\foo{test2}\isfoo\XKV@my@familya@keya@value
\define@key{familya}{keya}{\def\foo{fam=A,key=A,value=#1}}
\setkeys{familya}{\savevalue{keya}=test4}
\def\foo{test4}\isfoo\XKV@KV@familya@keya@value

{
\savekeys[my]{familya}{keya , keyc}
\def\foo{keya,keyc}\isfoo\XKV@my@familya@save
\gsavekeys[my]{familya}{keyb,keyc}
\def\foo{keya,keyc,keyb}\isfoo\XKV@my@familya@save
\savekeys[my]{familya}{keyd}
\def\foo{keya,keyc,keyb,keyd}\isfoo\XKV@my@familya@save
}
\def\foo{keya,keyc,keyb}\isfoo\XKV@my@familya@save
{
  \unsavekeys[my]{familya}
  \let\foo\undefined\isfoo\XKV@my@familya@save
}
\def\foo{keya,keyc,keyb}\isfoo\XKV@my@familya@save
{
  \gunsavekeys[my]{familya}
  \let\foo\undefined\isfoo\XKV@my@familya@save
}
\let\foo\undefined\isfoo\XKV@my@familya@save



\savekeys[my]{familya}{keya,keyc,keybb,keyd}
{
  \gdelsavekeys[my]{familya}{keyc,keye}
  \def\foo{keya,keybb,keyd}\isfoo\XKV@my@familya@save
  \gdelsavekeys[my]{familya}{keyb}
  \def\foo{keya,keybb,keyd}\isfoo\XKV@my@familya@save
  \gdelsavekeys[my]{familya}{keybbb}
  \def\foo{keya,keybb,keyd}\isfoo\XKV@my@familya@save
  \delsavekeys[my]{familya}{keybb,keye}
  \def\foo{keya,keyd}\isfoo\XKV@my@familya@save
  \delsavekeys[my]{familya}{keya,keye}
  \def\foo{keyd}\isfoo\XKV@my@familya@save
  \delsavekeys[my]{familya}{keyd,keye}
  \def\foo{}\isfoo\XKV@my@familya@save
}
\def\foo{keya,keybb,keyd}\isfoo\XKV@my@familya@save

\unsavekeys[my]{familya}
\savekeys[my]{familya}{keya,\global{keyc}}
\setkeys[my]{familya}{keyc=test5}
{
 \def\foo{keya,\global{keyc}}\isfoo\XKV@my@familya@save
 \savekeys[my]{familya}{\global{keya},keyc,keyd,\global{keye}}
 \def\foo{\global{keya},keyc,keyd,\global{keye}}\isfoo\XKV@my@familya@save
 \setkeys[my]{familya}{keya=test6}
 \setkeys[my]{familya}{keyc=test7}
 \def\foo{test7}\isfoo\XKV@my@familya@keyc@value
 \def\foo{test6}\isfoo\XKV@my@familya@keya@value
 \delsavekeys[my]{familya}{keyd,keye} 
 \def\foo{\global{keya},keyc}\isfoo\XKV@my@familya@save
}
\def\foo{test5}\isfoo\XKV@my@familya@keyc@value
\def\foo{test6}\isfoo\XKV@my@familya@keya@value
\def\foo{keya,\global{keyc}}\isfoo\XKV@my@familya@save

\define@key{familye}{keya}{}
\define@key{familye}{keyb}{}
\define@key{familye}{keyc}{}
\savekeys{familye}{keya,keyc}
{
\setkeys{familye}{keya=test4}
\def\foo{test4}\isfoo\XKV@KV@familye@keya@value
\setkeys{familye}{\savevalue{keya}=test5}
\def\foo{test5}\isfoo\XKV@KV@familye@keya@value
\setkeys{familye}{\gsavevalue{keyc}=test6}
\setkeys{familye}{keyc=test7}
\def\foo{test7}\isfoo\XKV@KV@familye@keyc@value
}
\def\foo{test6}\isfoo\XKV@KV@familye@keyc@value

%% pointers
\unsavekeys[my]{familya}
\setkeys[my]{familya,familyb}{keya=test}\ISfoo{fam=A,key=A,value=test}
\setkeys[my]{familya}{keya=\usevalue test1} % error
\setkeys[my]{familya}{keya=\usevalue{unknown}} % error
\setkeys[my]{familya}{\savevalue{keya}=test1}\ISfoo{fam=A,key=A,value=test1}
\setkeys[my]{familya}{keyb=x\usevalue{keya}y}\ISfoo{fam=A,key=B,value=xtest1y}
\setkeys[my]{familya}{\savevalue{keya}=test2}\ISfoo{fam=A,key=A,value=test2}
\setkeys[my]{familya}{\savevalue{keyb}=x\usevalue{keya}y}\ISfoo{fam=A,key=B,value=xtest2y}
\setkeys[my]{familya}{keyc=X\usevalue{keyb}Y}\ISfoo{fam=A,key=C,value=Xxtest2yY}
\setkeys[my]{familya}{\savevalue{keya}=\usevalue{keya}} % loop

\setkeys[my]{familya}{\savevalue{keya}=test3}\ISfoo{fam=A,key=A,value=test3}
\setkeys[my]{familya}{\savevalue{keyb}=\usevalue{keya}}\ISfoo{fam=A,key=B,value=test3}
\setkeys[my]{familya}{\savevalue{keya}=x\usevalue{keyb}y} % loop
\ISfoo{fam=A,key=A,value=}

\define@key{fam}{keya}{\def\foo{keya: #1}}
\define@key{fam}{keyb}[\usevalue{keya}Q]{\def\foo{keyb: #1}}
\define@key{fam}{keyc}[\usevalue{keyb}R]{\def\foo{keyc: #1}}
\setkeys{fam}{\savevalue{keya}=test}\ISfoo{keya: test}
\setkeys{fam}{\savevalue{keyb}}\ISfoo{keyb: testQ}
\setkeys{fam}{keyc}\ISfoo{keyc: testQR}

% preset
\presetkeys[pre]{fama}{keya, keyb=c, \savevalue{keyc}}{Keya, Keyb=c, \savevalue{Keyc}}
\def\foo{keya,keyb=c,\savevalue{keyc}}\isfoo\XKV@pre@fama@preseth
\def\foo{Keya,Keyb=c,\savevalue{Keyc}}\isfoo\XKV@pre@fama@presett
{
  \gpresetkeys[pre]{fama}{keya=1,keyd}{Keya=2,Keyd}
\def\foo{keya=1,keyb=c,\savevalue{keyc},keyd}\isfoo\XKV@pre@fama@preseth
\def\foo{Keya=2,Keyb=c,\savevalue{Keyc},Keyd}\isfoo\XKV@pre@fama@presett
  \presetkeys[pre]{fama}{keyc}{Keyc}
\def\foo{keya=1,keyb=c,keyc,keyd}\isfoo\XKV@pre@fama@preseth
\def\foo{Keya=2,Keyb=c,Keyc,Keyd}\isfoo\XKV@pre@fama@presett
}
\def\foo{keya=1,keyb=c,\savevalue{keyc},keyd}\isfoo\XKV@pre@fama@preseth
\def\foo{Keya=2,Keyb=c,\savevalue{Keyc},Keyd}\isfoo\XKV@pre@fama@presett
{
  \gdelpresetkeys[pre]{fama}{keya}{Keya}
\def\foo{keyb=c,\savevalue{keyc},keyd}\isfoo\XKV@pre@fama@preseth
\def\foo{Keyb=c,\savevalue{Keyc},Keyd}\isfoo\XKV@pre@fama@presett
  \delpresetkeys[pre]{fama}{keyb,keyw}{Keyb,Keyw}
\def\foo{\savevalue{keyc},keyd}\isfoo\XKV@pre@fama@preseth
\def\foo{\savevalue{Keyc},Keyd}\isfoo\XKV@pre@fama@presett
}
\def\foo{keyb=c,\savevalue{keyc},keyd}\isfoo\XKV@pre@fama@preseth
\def\foo{Keyb=c,\savevalue{Keyc},Keyd}\isfoo\XKV@pre@fama@presett
{
\unpresetkeys[pre]{fama}
\let\foo\undefined\isfoo\XKV@pre@fama@preseth
\let\foo\undefined\isfoo\XKV@pre@fama@presett
}
\def\foo{keyb=c,\savevalue{keyc},keyd}\isfoo\XKV@pre@fama@preseth
\def\foo{Keyb=c,\savevalue{Keyc},Keyd}\isfoo\XKV@pre@fama@presett
{
\gunpresetkeys[pre]{fama}
}
\let\foo\undefined\isfoo\XKV@pre@fama@preseth
\let\foo\undefined\isfoo\XKV@pre@fama@presett

\savekeys[my]{familya}{keya}
\presetkeys[my]{familya}{keya=blue}{keyb=\usevalue{keya}}
\setkeys[my]{familya}{keya=red}



\define@key[my2]{familya}{keya}{\edef\foo{\foo keya: #1}}
\define@key[my2]{familya}{keyb}{\edef\foo{\foo keyb: #1}}
\define@key[my2]{familya}{keyc}{\edef\foo{\foo keyc: #1}}
\savekeys[my2]{familya}{keya}
\presetkeys[my2]{familya}{keya=blue}{keyb=\usevalue{keya}}
\def\foo{}
\setkeys[my2]{familya}{keya=red}\ISfoo{keya: redkeyb: red}
\def\foo{}
\setkeys[my2]{familya}{keyc=green}\ISfoo{keya: bluekeyc: greenkeyb: blue}
\def\foo{}
\setkeys[my2]{familya}{keyb=green}\ISfoo{keya: bluekeyb: green}
\def\foo{}
\setkeys[my2]{familya}{keya=green}\ISfoo{keya: greenkeyb: green}

%% options
\DeclareOptionX[my]<foobar>{landscape}{\landscapetrue}
\def\foo#1{\landscapetrue}\isfoo\my@foobar@landscape
\def\foo{\my@foobar@landscape{}}\isfoo\my@foobar@landscape@default
\DeclareOptionX[my]<foobar>{portrait}[false]{pt=#1}
\def\foo#1{pt=#1}\isfoo\my@foobar@portrait
\def\foo{\my@foobar@portrait{false}}\isfoo\my@foobar@portrait@default

%% options from 
\def\foo#1{\landscapefalse}\expandafter\isfoo\csname KV@testkeyval.sty@landscape\endcsname

\def\foo{prefix=KV@, fams=testkeyval.sty, this fam=testkeyval.sty, header=KV@testkeyval.sty@,this key=keya, na=,val=}
\isfoo\vara
\def\foo{prefix=KV@, fams=testkeyval.sty, this fam=testkeyval.sty, header=KV@testkeyval.sty@,this key=keyb, na=,val=1}
\isfoo\varb

\def\foo{}\isfoo\opA
\def\foo{C}\isfoo\opB
\def\foo{\foo}\isfoo\opC
\def\foo{\foobar,gee}\isfoo\opF
\let\foo\relax\isfoo\opD {\setkeys[xxx]{yyy}{}\setkeys{Gin}{width=20pt}\includegraphics{G}}

\def\xNameA{Jos�}% 
\def\xNameB{José}% 
\ifx\NameA\xNameA\else \bad \fi
\ifx\NameB\xNameB\else \bad \fi


\bibliography{testkv}
\cite{louarn88}

\section{Testing errors and the like}
This is a test
\PackageError{mypack}{error line1\MessageBreak line2}{unused}
\PackageWarning{mypack}{warning line1\MessageBreak line2}
\PackageWarningNoLine{mypack}{warning nl line1\MessageBreak line2}
\PackageInfo{mypack}{info line1\MessageBreak line2}
\ClassError{myclass}{error line1\MessageBreak line2}{unused}
\ClassWarning{myclass}{warning line1\MessageBreak line2}
\ClassWarningNoLine{myclass}{warning nl line1\MessageBreak line2}
\ClassInfo{myclass}{info line1\MessageBreak line2}

\GenericInfo{info}{info bar}
\GenericWarning{PREFIX}{warning bar\MessageBreak second line}
\GenericError{PREFIX}{error bar\MessageBreak  second line}{FOO}{BAR}

\section{Elements}\label{here}
\ChangeElementName{node}{Node}
\ChangeElementName{nodeconnect}{NodeConnect}
\ChangeElementName{anodeconnect}{ANodeConnect}
\ChangeElementName{barnodeconnect}{BarNodeConnect}
\ChangeElementName{abarnodeconnect}{ABarNodeConnect}
\ChangeElementName{nodecurve}{NodeCurve}
\ChangeElementName{anodecurve}{ANodeCurve}
\ChangeElementName{nodecircle}{NodeCircle}
\ChangeElementName{nodetriangle}{NodeTriangle}
\ChangeElementName{nodeoval}{NodeOval}
\ChangeElementName{nodebox}{NodeBox}

\ChangeElementName*{nameA}{NameA}
\ChangeElementName*{nameB}{NameB}
\ChangeElementName*{posA}{PosA}
\ChangeElementName*{posB}{PosB}
\ChangeElementName*{depth}{Depth}
\ChangeElementName*{depthA}{DepthA}
\ChangeElementName*{depthB}{DepthB}
\ChangeElementName*{name}{Name}
\ChangeElementName*{pos}{Pos}
\ChangeElementName*{xpos}{Xpos}
\ChangeElementName*{ypos}{Ypos}

\ChangeElementName{keywords}{Keywords}
\ChangeElementName{term}{Term}
\ChangeElementName*{page}{Page}
\ChangeElementName{ref}{Ref}

\begin{motscle}
latex, xml, translator.
\end{motscle}

\section{Trees}
We test also trees\par % Check that a new paragraph is created
\node{a}{Value of node A}
\par % Should a paragraph start here ?
\nodepoint{b}. 
\nodepoint{c}[3pt].
\nodepoint{d}[4pt][5pt].
\nodeconnect{a}{b}
\nodeconnect[tl]{a}[r]{c}
\anodeconnect{a}{b}
\anodeconnect[tl]{a}[r]{c}
\barnodeconnect[3pt]{a}{d}
\barnodeconnect{a}{d}
\abarnodeconnect[3pt]{a}{d}
\abarnodeconnect{a}{d}
\nodecurve{a}{b}{2pt}.
\nodecurve[l]{a}[r]{b}{2pt}[3pt]
\anodecurve{a}{b}{2pt}.
\anodecurve[l]{a}[r]{b}{2pt}[3pt]
\nodetriangle{a}{b}
\nodebox{a}
\nodeoval{a}
\nodecircle[3pt]{a}
\nodecircle{a}

\pageref{here}
\ChangeElementName*{page}{NewPage}
\ChangeElementName{ref}{NewRef}
\pageref{here}

\show\unknownoptions

\expandafter\showthe\csname c@xyz\endcsname
\ifnum\value{xyz}=11102\else \bad\fi

Now checking runaway
\def\nohope#1\bad{}\nohope
This line is ignored
\def\@car{} this is ignored too

Second test
\@car there is a missing%\@nil
so that this line is ignored

Line of text.

\providecommand\ABC{x} \long\def\foo{x} \ifx\foo\ABC\else\bad\fi
\providecommand\ABC{xy} \long\def\foo{x} \ifx\foo\ABC\else\bad\fi
\providecommand*\ABc{x} \def\foo{x} \ifx\foo\ABc\else\bad\fi
\CheckCommand\par{} % this fails
\def\foo{}\CheckCommand*\foo{} % this OK
\long\def\foo{}\CheckCommand\foo{} % this OK
\def\foo{}\CheckCommand\foo{} % this fails
\def\foo{}\CheckCommand\foo{} % this fails
\renewcommand\foo[2][x]{#1#2}\CheckCommand\foo[2][x]{#1#2} % OK
\renewcommand\foo[2][xx]{#1#2}\CheckCommand\foo[2][x]{#1#2} % fails


\begin{it}{ \begingroup

%outer check for some commands, tokens after \par are not read
\@addtoreset{ab}{c\par x
\newcounter{foo}[a\par x
\nocite[a\par {a1}x
\XMLsolvecite[10][a \par yx
\cite{a2\par x
\cite@one{a \par {a3}{b}x
\cite@one{year}{a4 \par {b}x
\bibliographystyle{aa\par x
\bibliography{\par x
\cititem{aa\par {val}x%  this makes two errors
\XMLsolvecite{a \par x
\XMLaddatt{a}{b\par x
\XMLaddatt{b\par {a} x
\newtheorem{tha}[page\par {c} x
\newtheorem{thb}{b}[page\par x

\par \bibitem{a1}X\bibitem{a2}X\bibitem{a3}X\bibitem{a4}X%
\end{document}
